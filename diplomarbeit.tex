\documentclass[11pt,a4paper,openright,oneside]{report}

\usepackage{tabularx}
\usepackage{longtable}
\usepackage{ltxtable}
\usepackage{colortbl}
\usepackage{booktabs}

% UTF8-Zeichencode 
\usepackage[utf8]{inputenc}

% Dokument in deutsch
\usepackage{ngerman}

%\usepackage[ngerman]{babel}
%\usepackage{blindtext}
%\usepackage{fontspec}% provides font selecting commands
%\usepackage{xunicode}% provides unicode character macros
\usepackage{xltxtra} % provides some fixes/extras 
%\usepackage{mathpazo}
\defaultfontfeatures{Mapping=tex-text}
\setmainfont{Cambria}
\setsansfont{Calibri}
\setmonofont{Consolas}

% Verknüpfungen in der PDF
\usepackage{hyperref}

%%%%%%%%%%%%%%%%%%%%%%%%%%%%%%%%%%%%%%%%%%%%%%%%%%%%%%%%%%%%%%
% Quellcode
\usepackage{listings}
\lstdefinelanguage{blub}{sensitive=false,morekeywords={for,each,mod,if,to,else,while,end,then,Procedure,until,new,in,do}}
\lstset{%inputencoding=utf8/latin1,
	basicstyle=\ttfamily\small,
	keywordstyle=\color{blue},
	commentstyle=\color{darkgreen},
	numbers=none,
	breaklines=true,
	showstringspaces=false,
	tabsize=4,
	captionpos=b,
	float=htp,
	frame=TB,
	language=blub,
	mathescape=true,
	morecomment=[l]{//},
}



%%%%%%%%%%%%%%%%%%%%%%%%%%%%%%%%%%%%%%%%%%%%%%%%%%%%%%%%%%%%
% (Farbige) Rahmen
\usepackage{framed}
\setlength{\fboxrule}{.5pt}
\setlength{\fboxsep}{4pt}


\newenvironment{DefFrame}{
    \def\FrameCommand{\fcolorbox{clLight80Blue!33}{clLight80Blue!33}}
    \MakeFramed{\advance\hsize-\width \FrameRestore}
  }
  {\endMakeFramed}

\newenvironment{ToDoFrame}{
    \def\FrameCommand{\fcolorbox{clRed}{clLight40Red!33}}
    \MakeFramed{\advance\hsize-\width \FrameRestore}
  }
  {\endMakeFramed}

\newenvironment{TheoFrame}{
    \def\FrameCommand{\fcolorbox{clDark25Green}{clLight40Green!33}}
    \MakeFramed{\advance\hsize-\width \FrameRestore}
  }
  {\endMakeFramed}

\newenvironment{LemFrame}{
    \def\FrameCommand{\fcolorbox{clLight40Green!33}{clLight40Green!33}}
    \MakeFramed{\advance\hsize-\width \FrameRestore}
  }
  {\endMakeFramed}


%%%%%%%%%%%%%%%%%%%%%%%%%%%%%%%%%%%%%%%%%%%%%%%%%%%%%%%%
% Blöcke für Definitionen, Sätze und Beweise
%\usepackage[framed,standard,hyperref]{ntheorem}
\usepackage[framed,hyperref]{ntheorem}

\theoremstyle{plain}
\theoremheaderfont{\rmfamily\bfseries}
\theorembodyfont{\normalfont}
\theoremseparator{}
\theoremprework{}
\theorempreskipamount 0pt
\theorempostskipamount 0pt

\newtheorem*{myProof}{Beweis}
\newtheorem{DefBox}{Definition}[chapter]
\newtheorem{TheoBox}{Satz}
\newtheorem{LemBox}{Lemma}

\newenvironment{mydef}%[1][]%
{\begin{DefFrame}\begin{DefBox}%[#1]
}%
{\end{DefBox}\end{DefFrame}}
 
\newenvironment{myTheo}%
{\begin{TheoFrame}\begin{TheoBox}}%
{\end{TheoBox}\end{TheoFrame}}
 
\newenvironment{Lemma}%
{\begin{LemFrame}\begin{LemBox}}%
{\end{LemBox}\end{LemFrame}} 



\makeatletter % Anpssung der Theorem-Konstrukte: \endtrivlist                          wird früher verwendet, um den erzeugten Platz am Ende des Theorems zu entfernen. Dient primär dazu, damit es in einem gefärbten Frame besser aussieht.
\gdef\@thm#1#2#3{%
   \if@thmmarks
     \stepcounter{end\InTheoType ctr}%
   \fi
   \renewcommand{\InTheoType}{#1}%
   \if@thmmarks
     \stepcounter{curr#1ctr}%
     \setcounter{end#1ctr}{0}%
   \fi
   \refstepcounter{#2}%
   \theorem@prework
   \thm@topsepadd \theorempostskipamount   % cf. latex.ltx: \@trivlist
   \ifvmode \advance\thm@topsepadd\partopsep\fi
   \trivlist                          
   \@topsep \theorempreskipamount
   \@topsepadd \thm@topsepadd        % used by \@endparenv
   \advance\linewidth -\theorem@indent
   \advance\@totalleftmargin \theorem@indent
   \parshape \@ne \@totalleftmargin \linewidth
   \@ifnextchar[{\@ythm{#1}{#2}{#3}}{\@xthm{#1}{#2}{#3}}}\endtrivlist %%%%%%%%%%%%%%%% neu
\gdef\@endtheorem{%
  %\endtrivlist                          %%%%%%%%%%%%%%%%%%%%%%%%%%%%%%%%%%%%%%%%%%%%%%%%%%%
  \csname\InTheoType @postwork\endcsname
  }

\makeatother


%%%%%%%%%%%%%%%%%%%%%%%%%%%%%%%%%%%%%%%%%%%%%%%%%%%
% Bilder und Abbildungen

\usepackage{graphicx} % Bilder
\usepackage{subfig} % Abbildungen aus mehreren Bildern

% Rahmen für Abbildungen in einen Befehl kapseln
\newcommand{\floatimage}[3]{\begin{figure}[htb]
\centering
#3
\caption{#2}
\label{#1}
\end{figure}}

% Mathe Symbole
\usepackage{amssymbExp}
\usepackage{amsmath}
\DeclareRobustCommand{\qed}{%
  \ifmmode \square%\mathqed
  \else
    \leavevmode\unskip\penalty9999 \hbox{}\nobreak\hfill
    \quad\hbox{$\square$}%
  \fi
}

% ToDo-Befehl
\newcommand{\todo}[1]{
    \begin{ToDoFrame}\itshape
        {\rmfamily\normalsize\bfseries ToDo:} #1
    \end{ToDoFrame}
}

% Abstand nach einem Absatz
\newcommand{\parspace}{\parskip 10pt}

%%%%%%%%%%%%%%%%%%%%%%%%%%%%%%%%%%%%%%%%%%%%%%%%%%%%%%%%%%%%%%%%%
% Farben
\usepackage[table]{xcolor}
\definecolor{clDefBG}{rgb}{.776,.850,.941}
\definecolor{clSecTxt}{rgb}{.122,.286,.490}
\definecolor{clChapTxt}{rgb}{.090,.212,.365}%{.06,.14,.24}

\definecolor{clLight80Blue}{rgb}{.776,.850,.941}
\definecolor{clLight60Blue}{rgb}{.553,.584,.886}
\definecolor{clLight40Blue}{rgb}{.329,.553,.831}
\definecolor{clBlue}{rgb}{.122,.286,.490}
\definecolor{clDark25Blue}{rgb}{.090,.212,.365}
\definecolor{clDark50Blue}{rgb}{.059,.141,.243}

\definecolor{clLight80Green}{rgb}{.886,.894,.906}
\definecolor{clLight60Green}{rgb}{.843,.890,.737}
\definecolor{clLight40Green}{rgb}{.765,.839,.608}
\definecolor{clGreen}{rgb}{.608,.733,.349}
\definecolor{clDark25Green}{rgb}{.463,.573,.235}
\definecolor{clDark50Green}{rgb}{.310,.380,.157}

%\definecolor{clLight80Green}{rgb}{.886,.894,.906}
%\definecolor{clLight60Green}{rgb}{.843,.890,.737}
\definecolor{clLight40Red}{rgb}{.851,.588,.580}
\definecolor{clRed}{rgb}{.753,.314,.302}
%\definecolor{clDark25Green}{rgb}{.463,.573,.235}
%\definecolor{clDark50Green}{rgb}{.310,.380,.157}

% Schrift
%\usepackage{fontenc} 
%\usepackage{lmodern}
%\usepackage{textcomp}

% Computer Modern Sans Serif als Standard-Schrift 
\renewcommand*\familydefault{\sfdefault} %% Only if the base font of the document is to be sans serif

\makeatletter
\renewcommand{\section}{\@startsection {section}{1}{\z@}%
                                   {-3.5ex \@plus -1ex \@minus -.2ex}%
                                   {2.3ex \@plus.2ex}%
                                   {\rmfamily\Large\bfseries\color{clBlue}}}
\renewcommand{\subsection}{\@startsection{subsection}{2}{\z@}%
                                     {-3.25ex\@plus -1ex \@minus -.2ex}%
                                     {1.5ex \@plus .2ex}%
                                     {\rmfamily\large\bfseries\color{clBlue}}}
\renewcommand{\subsubsection}{\@startsection{subsubsection}{3}{\z@}%
                                     {-3.25ex\@plus -1ex \@minus -.2ex}%
                                     {1.5ex \@plus .2ex}%
                                     {\rmfamily\normalsize\bfseries\color{clBlue}}}


%---------------------------------------------
% \Chapter ändern
%
\def\@makechapterhead#1{%
    {\parspace
     \parbox[t][25\p@]{\linewidth}
     {\parindent \z@ \raggedright \rmfamily
     \ifnum \c@secnumdepth >\m@ne \vspace{\fill}
         \normalsize \bfseries \textcolor{clChapTxt}{\@chapapp\space \thechapter}
     \fi}
     \par\nobreak  \vskip 5\p@
     \interlinepenalty\@M   \raggedright \rmfamily
     \Huge \bfseries \textcolor{clChapTxt}{#1}\par\nobreak
     \vskip 40\p@
    }}
\def\@makeschapterhead#1{%
    {\parspace
     \parbox[t][25\p@]{\linewidth}{}
     \parindent \z@ \raggedright
     \par\nobreak  \vskip 5\p@
     \interlinepenalty\@M
     \Huge \rmfamily \bfseries  \textcolor{clChapTxt}{#1}\par\nobreak
     \vskip 40\p@
  }}
%---------------------------------------------
  
%---------------------------------------------
% Anderen Abstract
%
\newenvironment{abstractFrame}
{
  \null\vfil
  \@beginparpenalty\@lowpenalty
}
{\par\null}

\newenvironment{deAbstract}
{
  \selectlanguage{ngerman}
  \begin{center}%
    {\bfseries\rmfamily \abstractname\vspace{-.5em}\vspace{\z@}}%
  \end{center}% 
}
{\par\vfil}

\newenvironment{enAbstract}
{%
  \selectlanguage{english}
  \begin{center}%
    {\bfseries\rmfamily \abstractname\vspace{-.5em}\vspace{\z@}}%
  \end{center}% 
}%
{\par\vfil}
%---------------------------------------------


\renewcommand{\tableofcontents}{%
    \if@twocolumn
      \@restonecoltrue\onecolumn
    \else
      \@restonecolfalse
    \fi
    \chapter*{\contentsname
        \@mkboth{%
           \MakeUppercase\contentsname}{\MakeUppercase\contentsname}}%
    {\parskip 2pt\@starttoc{toc}}%
    \if@restonecol\twocolumn\fi
    }
    
\renewcommand{\listoffigures}{%
    \if@twocolumn
      \@restonecoltrue\onecolumn
    \else
      \@restonecolfalse
    \fi
    \chapter*{\listfigurename}%
      \@mkboth{\MakeUppercase\listfigurename}%
              {\MakeUppercase\listfigurename}%
    {\parskip 2pt\@starttoc{lof}}%
    \if@restonecol\twocolumn\fi
    }

\makeatother

% Zeilenabstand
\linespread{1.15}
\parspace

% Einrücken bei neuen Absatz
\parindent 0pt

% Linksbündig
\raggedright

% Abstand zwischen Überschrift und nachfolgendem Absatz veringern
\usepackage{titlesec}
\titlespacing{\section}{0pt}{*2}{*-1.3}
\titlespacing{\subsection}{0pt}{*2}{*-1.3}
\titlespacing{\subsubsection}{0pt}{*2}{*-1.3}

% URLs in den Quellen erkennen
\usepackage{url}

% Literatur- und Abbildungsverzeichnis mit ins Inhaltsverzeichnis
\usepackage[nottoc]{tocbibind}

% Literaturverzeichnis
\usepackage[numbers]{natbib}
\bibliographystyle{natdin}



 

% Verknüpfungen in der PDF
\usepackage{url}
\usepackage{hyperref}




%%%%%%%%%%%%%%%%%%%%%%%%%%%%%%%%%%%%%%%%%%%%%%%%%%%%%%%%%%%%%%
% Quellcode
\usepackage{listings}
\lstdefinelanguage{blub}{sensitive=false,morekeywords={for,each,mod,if,to,else,while,end,then,Procedure,Function,until,new,in,do,and,or}}
\lstset{%inputencoding=utf8/latin1,
	basicstyle=\ttfamily\small,
	keywordstyle=\color{blue},
	commentstyle=\color{darkgreen},
	numbers=none,
	breaklines=true,
	showstringspaces=false,
	tabsize=4,
	captionpos=b,
	float=htp,
	language=blub,
%	backgroundcolor=\color{clLight80Orange},
	mathescape=true,
	frame=single,
	rulecolor=\color{clBlue},
	morecomment=[l]{//},
}



%%%%%%%%%%%%%%%%%%%%%%%%%%%%%%%%%%%%%%%%%%%%%%%%%%%%%%%%%%%%
% (Farbige) Rahmen
\usepackage{framed}
\setlength{\fboxrule}{.5pt}
\setlength{\fboxsep}{4pt}


\newenvironment{CodeFrame}{
    \def\FrameCommand{\fcolorbox{clBlue}{white!0}}
    \MakeFramed{\advance\hsize-\width \FrameRestore}
  }
  {\endMakeFramed}

\newenvironment{ProofFrame}{
    \setlength{\fboxrule}{0pt}
    \setlength{\fboxsep}{0pt}
    \def\FrameCommand{\fcolorbox{white!0}{white!0}}
    \MakeFramed{\advance\hsize-\width \FrameRestore}
  }
  {\endMakeFramed}

\newenvironment{DefFrame}{
    \def\FrameCommand{\fcolorbox{clLight80Blue!33}{clLight80Blue!33}}
    \MakeFramed{\advance\hsize-\width \FrameRestore}
  }
  {\endMakeFramed}

\newenvironment{AlgoFrame}{
    \def\FrameCommand{\fcolorbox{clLight40Orange}{clLight80Orange!50}}
    \MakeFramed{\advance\hsize-\width \FrameRestore}
  }
  {\endMakeFramed}

\newenvironment{ToDoFrame}{
    \def\FrameCommand{\fcolorbox{clRed}{clLight40Red!33}}
    \MakeFramed{\advance\hsize-\width \FrameRestore}
  }
  {\endMakeFramed}

\newenvironment{TheoFrame}{
    \def\FrameCommand{\fcolorbox{clDark25Green}{clLight40Green!33}}
    \MakeFramed{\advance\hsize-\width \FrameRestore}
  }
  {\endMakeFramed}

\newenvironment{LemFrame}{
    \def\FrameCommand{\fcolorbox{clLight40Green!33}{clLight40Green!33}}
    \MakeFramed{\advance\hsize-\width \FrameRestore}
  }
  {\endMakeFramed}


%%%%%%%%%%%%%%%%%%%%%%%%%%%%%%%%%%%%%%%%%%%%%%%%%%%%%%%%
% Blöcke für Definitionen, Sätze und Beweise
\usepackage[framed,standard,hyperref]{ntheorem}

\theoremstyle{plain}
\theoremheaderfont{\rmfamily\bfseries\boldmath}
\theorembodyfont{\normalfont}
\theoremseparator{}
\theoremprework{}
\theorempreskipamount 0pt
\theorempostskipamount 0pt

\newtheorem{TheoBox}{Satz}[chapter]
\newtheorem{LemBox}{Lemma}[chapter]
\newtheorem*{LemBoxS}{Lemma}

\newtheorem*{myProof}{Beweis}

\theoremstyle{break}
\theoremheaderfont{\rmfamily\bfseries\boldmath}
\theorembodyfont{\normalfont}
\theoremseparator{}
\theoremprework{\vspace*{11pt}}
\theorempreskipamount 0pt
\theorempostskipamount 0pt

\theoremstyle{break}
\theoremheaderfont{\rmfamily\bfseries\boldmath}
\theorembodyfont{\normalfont}
\theoremseparator{}
\theoremprework{}
\theorempreskipamount 0pt
\theorempostskipamount 0pt

\newtheorem{DefBox}{Definition}[chapter]
\newtheorem{AlgoBox}{Algorithmus}[chapter]

\newenvironment{mydef}%[1][]%
{\begin{DefFrame}\begin{DefBox}%[#1]
}%
{\end{DefBox}\end{DefFrame}}
 
\newenvironment{Algorithm}%[1][]%
{\begin{AlgoFrame}\begin{AlgoBox}%[#1]
}%
{\end{AlgoBox}\end{AlgoFrame}}
 
\renewenvironment{Theorem}%
{\begin{TheoFrame}\begin{TheoBox}}%
{\end{TheoBox}\end{TheoFrame}} 

\renewenvironment{Lemma}%
{\begin{LemFrame}\begin{LemBox}}%
{\end{LemBox}\end{LemFrame}} 

\newenvironment{LemmaS}%
{\begin{LemFrame}\begin{LemBoxS}}%
{\end{LemBoxS}\end{LemFrame}} 

\renewenvironment{Proof}%
{\begin{myProof}\ \nopagebreak\\}%
{\end{myProof}} 

\makeatletter % Anpssung der Theorem-Konstrukte: \endtrivlist                          wird früher verwendet, um den erzeugten Platz am Ende des Theorems zu entfernen. Dient primär dazu, damit es in einem gefärbten Frame besser aussieht.
\gdef\@thm#1#2#3{%
   \if@thmmarks
     \stepcounter{end\InTheoType ctr}%
   \fi
   \renewcommand{\InTheoType}{#1}%
   \if@thmmarks
     \stepcounter{curr#1ctr}%
     \setcounter{end#1ctr}{0}%
   \fi
   \refstepcounter{#2}%
   \theorem@prework
   \thm@topsepadd \theorempostskipamount   % cf. latex.ltx: \@trivlist
   \ifvmode \advance\thm@topsepadd\partopsep\fi
   \trivlist                          
   \@topsep \theorempreskipamount
   \@topsepadd \thm@topsepadd        % used by \@endparenv
   \advance\linewidth -\theorem@indent
   \advance\@totalleftmargin \theorem@indent
   \parshape \@ne \@totalleftmargin \linewidth
   \@ifnextchar[{\@ythm{#1}{#2}{#3}}{\@xthm{#1}{#2}{#3}}}\endtrivlist %%%%%%%%%%%%%%%% neu
\gdef\@endtheorem{%
  %\endtrivlist                          %%%%%%%%%%%%%%%%%%%%%%%%%%%%%%%%%%%%%%%%%%%%%%%%%%%
  \csname\InTheoType @postwork\endcsname
  }

\makeatother


%%%%%%%%%%%%%%%%%%%%%%%%%%%%%%%%%%%%%%%%%%%%%%%%%%%
% Bilder und Abbildungen

\usepackage[font=footnotesize,labelfont=bf,justification=raggedright,hang]{caption}
%\renewcommand{\captionfont}{\footnotesize\sffamily}
%\renewcommand{\captionlabelfont}{\bfseries}
\usepackage{graphicx} % Bilder
\usepackage[format=default,justification=raggedright]{subfig} % Abbildungen aus mehreren Bildern

% Rahmen für Abbildungen in einen Befehl kapseln
\newcommand{\floatimage}[3]{\begin{figure}[htb]
\centering
#3
\caption{#2}
\label{#1}
\end{figure}}

% Mathe Symbole
\usepackage{amssymb}
\usepackage{amsmath}
\DeclareRobustCommand{\qed}{%
  \ifmmode \square%\mathqed
  \else
    \leavevmode\newline\unskip\penalty9999 \hbox{}\nobreak\hfill
    \quad\hbox{$\square$}%
  \fi
}

% ToDo-Befehl
\newcommand{\todo}[1]{
    \begin{ToDoFrame}\itshape
        {\rmfamily\normalsize\bfseries ToDo:} #1
    \end{ToDoFrame}
}

% Abstand nach einem Absatz
\newcommand{\parspace}{\setlength{\parskip}{11pt}}

%%%%%%%%%%%%%%%%%%%%%%%%%%%%%%%%%%%%%%%%%%%%%%%%%%%%%%%%%%%%%%%%%
% Farben
\usepackage[table]{xcolor}
\definecolor{clDefBG}{rgb}{.776,.850,.941}
\definecolor{clSecTxt}{rgb}{.122,.286,.490}
\definecolor{clChapTxt}{rgb}{.090,.212,.365}%{.06,.14,.24}

\definecolor{clLight80Blue}{rgb}{.776,.850,.941}
\definecolor{clLight60Blue}{rgb}{.553,.584,.886}
\definecolor{clLight40Blue}{rgb}{.329,.553,.831}
\definecolor{clBlue}{rgb}{.122,.286,.490}
\definecolor{clDark25Blue}{rgb}{.090,.212,.365}
\definecolor{clDark50Blue}{rgb}{.059,.141,.243}

\definecolor{clLight80Green}{rgb}{.886,.894,.906}
\definecolor{clLight60Green}{rgb}{.843,.890,.737}
\definecolor{clLight40Green}{rgb}{.765,.839,.608}
\definecolor{clGreen}{rgb}{.608,.733,.349}
\definecolor{clDark25Green}{rgb}{.463,.573,.235}
\definecolor{clDark50Green}{rgb}{.310,.380,.157}

%\definecolor{clLight80Green}{rgb}{.886,.894,.906}
%\definecolor{clLight60Green}{rgb}{.843,.890,.737}
\definecolor{clLight40Red}{rgb}{.851,.588,.580}
\definecolor{clRed}{rgb}{.753,.314,.302}
%\definecolor{clDark25Green}{rgb}{.463,.573,.235}
%\definecolor{clDark50Green}{rgb}{.310,.380,.157}

\definecolor{clViolet}{HTML}{8064A2}

\definecolor{clLight80Orange}{HTML}{FDEADA}
\definecolor{clLight60Orange}{HTML}{FBD5B5}
\definecolor{clLight40Orange}{HTML}{FAC08F}
\definecolor{clOrange}{HTML}{F79646}
\definecolor{clDark25Orange}{HTML}{E36C09}
\definecolor{clDark50Orange}{HTML}{974806}

\definecolor{clAqua}{HTML}{4BACC6}

% Schrift
\usepackage[T1]{fontenc} 
\usepackage{lmodern}
\usepackage{textcomp}

% Computer Modern Sans Serif als Standard-Schrift 
%\renewcommand*\familydefault{\sfdefault} %% Only if the base font of the document is to be sans serif

\makeatletter
\renewcommand{\section}{\@startsection {section}{1}{\z@}%
                                   {-3.5ex \@plus -1ex \@minus -.2ex}%
                                   {2.3ex \@plus.2ex}%
                                   {\rmfamily\Large\bfseries\boldmath\color{clBlue}}}
\renewcommand{\subsection}{\@startsection{subsection}{2}{\z@}%
                                     {-3.25ex\@plus -1ex \@minus -.2ex}%
                                     {1.5ex \@plus .2ex}%
                                     {\rmfamily\large\bfseries\boldmath\color{clBlue}}}
\renewcommand{\subsubsection}{\@startsection{subsubsection}{3}{\z@}%
                                     {-3.25ex\@plus -1ex \@minus -.2ex}%
                                     {1.5ex \@plus .2ex}%
                                     {\rmfamily\normalsize\bfseries\boldmath\color{clBlue}}}
\renewcommand\paragraph{\@startsection{paragraph}{4}{\z@}%
                                    {3.25ex \@plus1ex \@minus.2ex}%
                                    {-1em}%
                                    {\rmfamily\normalsize\boldmath\bfseries}}




%---------------------------------------------
% \Part ändern
%

\renewcommand\part{%
  \if@openright
    \cleardoublepage
  \else
    \clearpage
  \fi
  \thispagestyle{empty}%
  \if@twocolumn
    \onecolumn
    \@tempswatrue
  \else
    \@tempswafalse
  \fi
  \null\vspace*{\fill}
  \secdef\@part\@spart}

\def\@part[#1]#2{%
    \ifnum \c@secnumdepth >-2\relax
      \refstepcounter{part}%
      \addcontentsline{toc}{part}{\thepart\hspace{1em}#1}%
    \else
      \addcontentsline{toc}{part}{#1}%
    \fi
    \markboth{}{}%
    {\centering
     \interlinepenalty \@M
     \normalfont\raggedleft
     \color{clChapTxt}\rule{\linewidth}{1pt}
     \ifnum \c@secnumdepth >-2\relax
       \normalsize\bfseries\boldmath \partname\nobreakspace\thepart
       \\
       %\vskip 20\p@
     \fi
     \Huge \bfseries\boldmath #2\par
     \rule[\baselineskip]{\linewidth}{1pt}
     }%
    \@endpart}
    
\def\@spart#1{%
    {\centering
     \interlinepenalty \@M
     \normalfont\raggedleft
     \color{clChapTxt}\rule{\linewidth}{1pt}
     \Huge \bfseries\boldmath #1\par
     \rule[\baselineskip]{\linewidth}{1pt}}%
    \@endpart}
    
\def\@endpart{
    \vspace*{\fill}
    \vspace*{\fill}
    \vspace*{\fill}
    \newpage
              \if@twoside
               \if@openright
                \null
                \thispagestyle{empty}%
                \newpage
               \fi
              \fi
              \if@tempswa
                \twocolumn
              \fi}
              

%---------------------------------------------
% \Chapter ändern
%
\renewcommand\chapter{\if@openright\cleardoublepage\else\clearpage\fi
                    \thispagestyle{plain}%
                    \global\@topnum\z@
                    \@afterindentfalse
                    \secdef\@chapter\@schapter}
\def\@makechapterhead#1{%
    {\parspace
     \parbox[t][25\p@]{\linewidth}
     {\parindent \z@ \raggedright \rmfamily
     \ifnum \c@secnumdepth >\m@ne \vspace{\fill}
         \normalsize \bfseries \textcolor{clChapTxt}{\@chapapp\space \thechapter}
     \fi}
     \par\nobreak  \vskip 5\p@
     \interlinepenalty\@M   \raggedright \rmfamily
     \Huge \bfseries\boldmath \textcolor{clChapTxt}{#1}\par\nobreak
     \vskip 40\p@
    }}
\def\@makeschapterhead#1{%
    {\parspace
     \parbox[t][25\p@]{\linewidth}{}
     \parindent \z@ \raggedright
     \par\nobreak  \vskip 5\p@
     \interlinepenalty\@M
     \Huge \rmfamily \bfseries\boldmath  \textcolor{clChapTxt}{#1}\par\nobreak
     \vskip 40\p@
  }}



%---------------------------------------------
% Kopfzeile

\setlength{\headheight}{16.5pt}

\if@twoside
  \def\ps@headings{%
    \def\@oddfoot{\tikz{
        \node[inner xsep=0pt, anchor=base west] (chapNode) at (0,0) {};
        \node[inner xsep=0pt, anchor=base east] (pageNode) at (\textwidth,0) {\thepage};
        %\draw[clChapTxt,semithick] ($(current bounding box.north west)+(0,1pt)$) -- ($(current bounding box.north east)+(0,0pt)$);
        %\draw[clChapTxt,semithick] (0,1em) -- ($(\linewidth,1em)-(0.0pt,0)$);
        \draw[clChapTxt,semithick] (0.5\textwidth,1em) -- (current bounding box.north west)
                                   (0.5\textwidth,1em) -- (current bounding box.north east);
    }}
    \def\@evenfoot{\tikz{
        \node[inner xsep=0pt, anchor=base west] (chapNode) at (0,0) {\thepage};
        \node[inner xsep=0pt, anchor=base east] (pageNode) at (\textwidth,0) {};
        %\draw[clChapTxt,semithick] ($(current bounding box.north west)+(0,1pt)$) -- ($(current bounding box.north east)+(0,0pt)$);
        %\draw[clChapTxt,semithick] (0,1em) -- ($(\linewidth,1em)-(0.0pt,0)$);
        \draw[clChapTxt,semithick] (0.5\textwidth,1em) -- (current bounding box.north west)
                                   (0.5\textwidth,1em) -- (current bounding box.north east);
    }}
      
   %   \def\@evenhead{\thepage\hfil\slshape\leftmark}%
    \def\@evenhead{\tikz{
        \node[inner xsep=0pt, anchor=base west] (chapNode) at (0,0) {};
        \node[inner xsep=0pt, anchor=base east] (pageNode) at (\textwidth,0) {\rmfamily\small\textcolor{clChapTxt}\leftmark};
        \draw[clChapTxt,semithick] (current bounding box.south west) -- (current bounding box.south east);
    }}%
    \def\@oddhead{\tikz{
        \node[inner xsep=0pt, anchor=base west] (chapNode) at (0,0) {\rmfamily\small\textcolor{clChapTxt}\rightmark};
        \node[inner xsep=0pt, anchor=base east] (pageNode) at (\textwidth,0) {};
        \draw[clChapTxt,semithick] (current bounding box.south west) -- (current bounding box.south east);
    }}%
      \let\@mkboth\markboth
    \def\chaptermark##1{%
      \markboth {{%
        \ifnum \c@secnumdepth >\m@ne
            \@chapapp\ \thechapter. \ %
        \fi
        ##1}}{}}%
    \def\sectionmark##1{%
      \markright {{%
        \ifnum \c@secnumdepth >\z@
          \thesection. \ %
        \fi
        ##1}}}}
\else
  \def\ps@headings{%
    %\let\@oddfoot\@empty
    \def\@oddfoot{\tikz{
        \node[inner xsep=0pt, anchor=base west] (chapNode) at (0,0) {};
        \node[inner xsep=0pt, anchor=base east] (pageNode) at (\textwidth,0) {\textcolor{black}\thepage};
        %\draw[clChapTxt,semithick] ($(current bounding box.north west)+(0,1pt)$) -- ($(current bounding box.north east)+(0,0pt)$);
        %\draw[clChapTxt,semithick] (0,1em) -- ($(\linewidth,1em)-(0.0pt,0)$);
        \draw[clChapTxt,semithick] (0.5\textwidth,1em) -- (current bounding box.north west)
                                   (0.5\textwidth,1em) -- (current bounding box.north east);
    }}
    \def\@oddhead{\tikz{
        \node[inner xsep=0pt, anchor=base west] (chapNode) at (0,0) {\rmfamily\small\textcolor{clChapTxt}\rightmark};
        \node[inner xsep=0pt, anchor=base east] (pageNode) at (\textwidth,0) {};
        \draw[clChapTxt,semithick] (current bounding box.south west) -- (current bounding box.south east);
    }}%
    \let\@mkboth\markboth
    \def\chaptermark##1{%
      \markright {%
        \ifnum \c@secnumdepth >\m@ne
            \@chapapp\ \thechapter. \ %
        \fi
        ##1}}}
\fi



\if@twoside
  \def\ps@plain{%
    %\let\@oddfoot\@empty
    \def\@evenfoot{\tikz{
        \node[inner xsep=0pt, anchor=base west] (chapNode) at (0,0) {\textcolor{black}\thepage};
        \node[inner xsep=0pt, anchor=base east] (pageNode) at (\textwidth,0) {};
        \draw[clChapTxt,semithick] (0.5\textwidth,1em) -- (current bounding box.north west)
                                   (0.5\textwidth,1em) -- (current bounding box.north east);
    }}
    \def\@evenhead{\tikz{
        \node[inner xsep=0pt, anchor=base west] (chapNode) at (0,0) {};
        \node[inner xsep=0pt, anchor=base east] (pageNode) at (\textwidth,0) {};
        \draw[clChapTxt,semithick] (current bounding box.south west) -- (current bounding box.south east);
    }}%
    \def\@oddfoot{\tikz{
        \node[inner xsep=0pt, anchor=base west] (chapNode) at (0,0) {};
        \node[inner xsep=0pt, anchor=base east] (pageNode) at (\textwidth,0) {\textcolor{black}\thepage};
        \draw[clChapTxt,semithick] (0.5\textwidth,1em) -- (current bounding box.north west)
                                   (0.5\textwidth,1em) -- (current bounding box.north east);
    }}
    \def\@oddhead{\tikz{
        \node[inner xsep=0pt, anchor=base west] (chapNode) at (0,0) {};
        \node[inner xsep=0pt, anchor=base east] (pageNode) at (\textwidth,0) {};
        \draw[clChapTxt,semithick] (current bounding box.south west) -- (current bounding box.south east);
    }}%
    \let\@mkboth\markboth
    \def\chaptermark##1{%
      \markright {%
        \ifnum \c@secnumdepth >\m@ne
            \@chapapp\ \thechapter. \ %
        \fi
        ##1}}}
\else
  \def\ps@plain{%
    %\let\@oddfoot\@empty
    \def\@oddfoot{\tikz{
        \node[inner xsep=0pt, anchor=base west] (chapNode) at (0,0) {};
        \node[inner xsep=0pt, anchor=base east] (pageNode) at (\textwidth,0) {\textcolor{black}\thepage};
        \draw[clChapTxt,semithick] (0.5\textwidth,1em) -- (current bounding box.north west)
                                   (0.5\textwidth,1em) -- (current bounding box.north east);
    }}
    \def\@oddhead{\tikz{
        \node[inner xsep=0pt, anchor=base west] (chapNode) at (0,0) {};
        \node[inner xsep=0pt, anchor=base east] (pageNode) at (\textwidth,0) {};
        \draw[clChapTxt,semithick] (current bounding box.south west) -- (current bounding box.south east);
    }}%
    \let\@mkboth\markboth
    \def\chaptermark##1{%
      \markright {%
        \ifnum \c@secnumdepth >\m@ne
            \@chapapp\ \thechapter. \ %
        \fi
        ##1}}}
\fi

%---------------------------------------------
% Inhaltsverzeichnis ändern, damit es auch Mathe-Symbole fett macht
%
\renewcommand*\l@part[2]{%
  \ifnum \c@tocdepth >-2\relax
    \addpenalty{-\@highpenalty}%
    \addvspace{2.25em \@plus\p@}%
    \setlength\@tempdima{3em}%
    \begingroup
      \parindent \z@ \rightskip \@pnumwidth
      \parfillskip -\@pnumwidth
      {\leavevmode
       \large \bfseries\boldmath\color{clBlue}\rule{\linewidth}{1pt} #1\hfil \hb@xt@\@pnumwidth{\hss #2}\\\rule[.5em]{\linewidth}{1pt}}\par \vspace*{-\baselineskip} %\par
       %\large \bfseries\boldmath\color{clBlue} #1\hfil \hb@xt@\@pnumwidth{\hss #2}}\par
       \nobreak
         \global\@nobreaktrue
         \everypar{\global\@nobreakfalse\everypar{}}%
    \endgroup
  \fi}
\renewcommand*\l@chapter[2]{%
  \ifnum \c@tocdepth >\m@ne
    \addpenalty{-\@highpenalty}%
    \vskip 1.0em \@plus\p@
    \setlength\@tempdima{1.5em}%
    \begingroup
      \parindent \z@ \rightskip \@pnumwidth
      \parfillskip -\@pnumwidth
      \leavevmode \bfseries\boldmath\color{clBlue}
      \advance\leftskip\@tempdima
      \hskip -\leftskip
      #1\nobreak\hfil \nobreak\hb@xt@\@pnumwidth{\hss #2}\par
      \penalty\@highpenalty
    \endgroup
  \fi}
%---------------------------------------------

  
%---------------------------------------------
% Anderen Abstract
%
\newenvironment{abstractFrame}
{
  \if@openright\cleardoublepage\else\clearpage\fi 
  \null\vfil
  \@beginparpenalty\@lowpenalty
}
{\par\null}

\newenvironment{deAbstract}
{
  \selectlanguage{ngerman}
  %\begin{center}%
  %  {\bfseries\rmfamily \abstractname\vspace{-.5em}\vspace{\z@}}%
  %\end{center}% 
  \subsection*{\abstractname}
}
{\par\vfil}

\newenvironment{enAbstract}
{%
  \selectlanguage{english}
  %\begin{center}%
  %  {\bfseries\rmfamily \abstractname\vspace{-.5em}\vspace{\z@}}%
  %\end{center}% 
  \subsection*{\abstractname}

}%
{\par\vfil}
%---------------------------------------------


\renewcommand{\tableofcontents}{%
    \if@twocolumn
      \@restonecoltrue\onecolumn
    \else
      \@restonecolfalse
    \fi
    \chapter*{\contentsname
        \@mkboth{%
           \contentsname}{\contentsname}}%
    {\setlength{\parskip}{2pt}\@starttoc{toc}}%
    \if@restonecol\twocolumn\fi
    }

\renewenvironment{thebibliography}[1]
     {\chapter*{\bibname\ blub}\addcontentsline{toc}{chapter}{\bibname\ blub}%
      \@mkboth{\bibname}{\bibname}%
      \list{\@biblabel{\@arabic\c@enumiv}}%
           {\settowidth\labelwidth{\@biblabel{#1}}%
            \leftmargin\labelwidth
            \advance\leftmargin\labelsep
            \@openbib@code
            \usecounter{enumiv}%
            \let\p@enumiv\@empty
            \renewcommand\theenumiv{\@arabic\c@enumiv}}%
      \sloppy
      \clubpenalty4000
      \@clubpenalty \clubpenalty
      \widowpenalty4000%
      \sfcode`\.\@m}
     {\def\@noitemerr
       {\@latex@warning{Empty `thebibliography' environment}}%
      \endlist}
    
\renewcommand{\listoffigures}{%
    \if@twocolumn
      \@restonecoltrue\onecolumn
    \else
      \@restonecolfalse
    \fi
    \chapter*{\listfigurename}\addcontentsline{toc}{chapter}{\listfigurename}%
      \@mkboth{\listfigurename}%
              {\listfigurename}%
    {\setlength{\parskip}{2pt}\@starttoc{lof}}%
    \if@restonecol\twocolumn\fi
    }
\renewcommand\listoftables{%
    \if@twocolumn
      \@restonecoltrue\onecolumn
    \else
      \@restonecolfalse
    \fi
    \chapter*{\listtablename}\addcontentsline{toc}{chapter}{\listfigurename}%
      \@mkboth{%
          \listtablename}%
         {\listtablename}%
    {\setlength{\parskip}{2pt}\@starttoc{lot}}%
    \if@restonecol\twocolumn\fi
    }

\renewcommand*\l@figure{\@dottedtocline{1}{0em}{2.3em}}
\renewcommand*\l@table{\@dottedtocline{1}{0em}{2.3em}}

\makeatother

% Zeilenabstand
\linespread{1.15}
\parspace

% Einrücken bei neuen Absatz
\parindent 0pt

% Linksbündig
\raggedright

% Abstand zwischen Überschrift und nachfolgendem Absatz veringern
\usepackage{titlesec}
\titlespacing{\section}{0pt}{*2}{*-1.3}
\titlespacing{\subsection}{0pt}{*2}{*-1.3}
\titlespacing{\subsubsection}{0pt}{*2}{*-1.3}

% URLs in den Quellen erkennen
%\usepackage{url}
%\renewcommand\UrlFont{\ttfamily}


% Literatur- und Abbildungsverzeichnis mit ins Inhaltsverzeichnis
%\usepackage[nottoc,notlof,notlot]{tocbibind}

% Literaturverzeichnis
\usepackage[numbers]{natbib}
\bibliographystyle{natdin}

\makeatletter
\renewenvironment{theindex}
               {
                %
                \begin{multicols}{2}[\chapter*{\indexname}\addcontentsline{toc}{chapter}{\indexname}]
                \@mkboth{\indexname}%
                        {\indexname}%
                \thispagestyle{plain}\parindent\z@
                \parskip\z@ \@plus .3\p@\relax
                \columnseprule \z@
                \columnsep 35\p@
                \let\item\@idxitem}
               {\end{multicols}}
               
\renewcommand\bibsection{%
      \chapter*{\bibname}\addcontentsline{toc}{chapter}{\bibname}
      \@mkboth{\bibname}{\bibname}%
    }
\makeatother



%\show\labelitemi
 \newlength{\enumItemWidth}
\renewenvironment{itemize}
    {\begin{list}
        {\settowidth{\enumItemWidth}{\labelenumi}\hspace*{-\enumItemWidth}\labelitemi}
        {\setlength{\topsep}{0.5em}
         \setlength{\itemsep}{0pt}
         \setlength{\parskip}{0pt}
         }
    }
    {\end{list}}
\renewenvironment{enumerate}
    {\begin{list}
        {\settowidth{\enumItemWidth}{\labelenumi}\hspace*{-\enumItemWidth}\labelenumi}
        {\usecounter{enumi}
         \setlength{\topsep}{0.5em}
         \setlength{\itemsep}{0pt}
         \setlength{\parskip}{0pt}
         }
    }
    {\end{list}}
\renewcommand{\theenumi}{(\roman{enumi})}
\renewcommand{\labelenumi}{\theenumi} 
\renewcommand{\theenumii}{(\alph{enumii})}
\renewcommand{\labelenumii}{\theenumii} 

\newenvironment{codeLine}
    {\begin{list}
        {\settowidth{\enumItemWidth}{(\arabic{enumi})}\hspace*{-\enumItemWidth}(\arabic{enumi})}
        {\usecounter{enumi}
         \setlength{\topsep}{0.5em}
         \setlength{\itemsep}{0pt}
         \setlength{\parskip}{0pt}
         }
    }
    {\end{list}}

\newenvironment{innerCodeLine}
    {\begin{list}
        {\settowidth{\enumItemWidth}{\labelenumii}\hspace*{-\enumItemWidth}\labelenumii}
        {\usecounter{enumii}
         \setlength{\topsep}{0.5em}
         \setlength{\itemsep}{0pt}
         \setlength{\parskip}{0pt}
         }
    }
    {\end{list}}

        

 
\input{tex-Dateien/Title}

\hypersetup{pdftex,unicode,
 pdfauthor={Arne Leitert},
 pdftitle={Das Dominating Induced Matching Problem für azyklische Hypergraphen},
 pdfkeywords={diploma thesis, dominating induced matching, acyclic hypergraphs, dually chordal, efficient domination, efficient edge domination},
 pdfcreator=pdflatex}

%\usepackage{gfsartemisia}
%eventuell für überschriften
\usepackage{calc}

\usepackage{inconsolata}
%\usepackage{bera}
%\usepackage[scaled]{berasans}




%% TikZ (http://www.texample.net/tikz/)
\usepackage{tikz}
\usetikzlibrary{shapes,positioning,fit,fadings,calc,decorations.pathmorphing}
\definecolor{darkgreen}{rgb}{0,0.5,0}
\pgfdeclarelayer{background}
\pgfdeclarelayer{lowerBackground}
\pgfsetlayers{lowerBackground,background,main}
\definecolor{bggreen}{rgb}{0.57,0.81,0.31}
\definecolor{bgblue}{rgb}{0.329,0.553,0.831}
\definecolor{bgred}{rgb}{0.753,0.314,0.302}

%%% Tikz-Stile:

% Basis für Knoten
\tikzstyle{nodeBase}=[minimum size=0.25cm,inner sep=0pt]

% Basis für sichtbare Knoten
\tikzstyle{visNode}=[nodeBase,thick,draw,fill=white]

% Normale Knoten
\tikzstyle{nN}=[visNode,circle]

% Kleine Knoten
\tikzstyle{sN}=[nN,minimum size=0.15cm]

% Unsichtbarer Knoten
\tikzstyle{hN}=[nodeBase,circle]

% Knoten einer Kante des Matchings
\tikzstyle{mN}=[visNode,minimum size=0.223cm]

% Label
\tikzstyle{lbl}=[font=\small]
\tikzstyle{llbl}=[left,lbl]
\tikzstyle{rlbl}=[lbl,right]
\tikzstyle{tlbl}=[lbl,above]
\tikzstyle{blbl}=[lbl,below]

% Spannbaum von alphaL Graphen
\tikzstyle{Tedge}=[very thick,clBlue,decoration={snake,amplitude=1},decorate]



\newcommand{\cirBase}[1]{\tikz[baseline,line width=.1ex]{\node [draw,anchor=base,inner sep=.2ex,circle] {#1};}}

\DeclareRobustCommand{\mathCir}[1]{%
  \ifmmode \,\cirBase{#1}\,
  \else
    \cirBase{#1}
  \fi
}
\newcommand{\join}{\mathCir{1}}
\newcommand{\cojoin}{\mathCir{0}}

\newcommand{\mC}{\mathcal{C}}
\newcommand{\mE}{\mathcal{E}}
\newcommand{\mF}{\mathcal{F}}
\newcommand{\mI}{\mathcal{I}}
\newcommand{\mK}{\mathcal{K}}
\newcommand{\mL}{\mathcal{L}}
\newcommand{\mN}{\mathcal{N}}
\newcommand{\mO}{\mathcal{O}}
\newcommand{\mP}{\mathcal{P}}
\newcommand{\mS}{\mathcal{S}}
\newcommand{\mU}{\mathcal{U}}
\newcommand{\mV}{\mathcal{V}}

\newcommand{\prR}{{\bfseries\boldmath$\Rightarrow$:\ }}
\newcommand{\prL}{{\bfseries\boldmath$\Leftarrow$:\ }}

\newcommand{\OBdA}{O.\,B.\,d.\,A.\ }
\newcommand{\oBdA}{o.\,B.\,d.\,A.\ }
\newcommand{\zB}{z.\,B.~}

% Mehrere Zeilen in einer Tabelle zusammenfassen
\usepackage{multirow}

% Mehrfachspalten
\usepackage{multicol}

%%%%%%%%%%%%%%%%%%%%%%%%%%%%%

\thesisType{Diplomarbeit}
%\title{\Huge Dominating Induced Matchings\\ für $\alpha$-azyklische Hypergraphen}
\title{Das Dominating Induced Matching Problem für azyklische Hypergraphen}
\author{Arne Leitert}
\matNo{6201646}
\studies{Diplom Informatik}
\eMail{arne.leitert@uni-rostock.de}
\chair{Lehrstuhl für Theoretische Informatik}
\university{Universität Rostock \\
            Fakultät für Informatik und Elektrotechnik}
            

\usepackage{makeidx}
\makeindex
%\usepackage{microtype}

\begin{document}

\pagestyle{headings}

% ersten Seiten mit römischen Zahlen nummerieren
\pagenumbering{roman}

% Titelseite
\maketitle

% Dokument ist unter CC
\makeatletter
\if@openright\cleardoublepage\else\clearpage\fi 
\makeatother%\pagestyle{plain}
%\setcounter{page}{2}
\vspace*{\fill}
%\null\vfil

\begin{center}
\includegraphics[width=0.3\linewidth,keepaspectratio]{bilder/cc_by-nc-sa_eu}

Diese Arbeit ist unter einem Creative Commons Attribution-""NonCommercial-""ShareAlike~3.0 Germany Lizenzvertrag lizenziert.

\url{http://creativecommons.org/licenses/by-nc-sa/3.0/de/}
\end{center}

%\vfil\null
\vspace*{\fill}
%\clearpage 

%\vspace*{\fill}

\begin{abstractFrame}

\begin{deAbstract}
Diese Arbeit beschäftigt sich mit dem Dominating Induced Matching Problem für azyklische Hypergraphen. Dabei wird nach einer Kantenmenge gesucht, so dass jede Kante mit genau einer Kante aus dieser Menge einen gemeinsamen Knoten besitzt. Nach der Behandlung der theoretischen Grundlagen wird gezeigt, dass sich das Problem auf das Weighted Independent Set Problem für das Quadrat des Linegraphen reduzieren lässt. Am Ende der Arbeit werden dann eingie Algorithmen vorgestellt, um das Problem für verschiedene Klassen von azyklischen Hypergraphen in Polynomialzeit zu lösen.
\end{deAbstract}

%\vfil

\begin{enAbstract}
This thesis deals with the Dominating Induced Matching problem for acyclic hypergraphs. This problem asks for a set of edges so that every edge shares a vertex with exactly one edge of this set. After dealing with the theoretical background this thesis will show that the problem can be reduced to the Weighted Independent Set problem on the square of the linegraph. At the end of the thesis some algorithms will be presentetd to solve the problem for several kinds of acyclic hypergraphs in polynomial time.
\end{enAbstract}

\end{abstractFrame}

% Inhaltsverzeichniss
\tableofcontents

% Nach Inhalt wieder arabische Zahlen benutzen und bei 1 beginnen.
\newpage
\pagenumbering{arabic}
\setcounter{page}{1}

\chapter{Einleitung}

Das \emph{Dominating Induced Matching Problem} sucht nach einer Teilmenge der Kanten eines Graphen, so dass jede Kante des Graphen entweder in dieser Teilmenge ist oder mit genau einer solchen Kante benachbart ist. Es wird auch als \emph{Efficient Edge Domination Problem} bezeichnet und ist im Allgemeinen NP-vollständig \cite{dimNPv}.

Die Besonderheit dieses Problems ist, dass es sowohl ein Pack- als auch ein Abdeckungsproblem darstellt. Es müssen genug Kanten gewählt werden, damit alle Kanten, die nicht zum Matching gehören mit einer Kante des Matchings verbunden sind. Zwischen den Kanten des Matchings müssen sich aber gleichzeitig immer mindestens zwei weitere Kanten befinden.

Das Problem wurde für einfache Graphen bereits umfangreich studiert (siehe beispielsweise \cite{eedHoleFree, eedP7Free, Cardoso2011, Lu2002227, ChinLungLu|ChuanYiTang1998203}). Diese Arbeit untersucht das Problem nun für Hypergraphen. Dabei stehen azyklische Hypergraphen im Vordergrund.

Diese Arbeit ist dafür in sechs Kapitel unterteilt. Nach der Einleitung befassen sich das zweite und dritte Kapitel mit Graphen und Hypergraphen. Dabei werden verschiedene Klassen vorgestellt und miteinander in Verbindung gebracht. Das Thema des vierten Kapitels sind dominierende Knoten- und Kantenmengen sowie die dazugehörigen Probleme (zu denen auch das Dominating Induced Matching Problem gehört). Basierend auf den Ergebnissen der vorherigen Kapitel zeigt dann das fünfte Kapitel algorithmische Ansätze zur Lösung des Problems. Abschließend werden im letzten Kapitel die Ergebnisse der Arbeit noch einmal zusammengefasst.

%\todo{Fertigstellen}
%
%\bigskip
%\bigskip
%
%Das Problem wurde zwar für einfache Graphen bereits umfangreich studiert
%
%hole free pol. \cite{eedHoleFree}\\
%p7-free lin \cite{eedP7Free} \\
%claw-free, convex pol \cite{Cardoso2011}\\
%(Anwendung oder so) \cite{Livingston1988} \\
%NP-complete on bipartite (planar bipartite) --- linear-time generalized series–parallel graphs and chordal graphs  \cite{Lu2002227}\\
%\cite{ChinLungLu|ChuanYiTang1998203} \\
%
%
%\bigskip
%\bigskip
%
%jedoch kaum für Hypergraphen.
%
%\bigskip
%\bigskip
%
%Diese Arbeit soll nun das Problem für die Klasse der azyklischen Hypergraphen untersuchen. Angestrebtes Ziel dabei ist es einen Polynomialzeit-Algorithmus zur Lösung des Problems zu finden oder zu zeigen, dass ein solcher Algorithmus nicht existiert.
%
%Das dazugehörige Entscheidungsproblem (Dominating Induced Matching
%Problem; kurz DIM) fragt, ob ein Graph ein solches Matching besitzt. Für
%Graphen wird es auch als Efficient Edge Domination Problem (EED)
%bezeichnet und ist NP-vollständig \cite{dimNPv}. Dies gilt auch für Hypergraphen, da
%jeder Graph auch ein Hypergraph ist.
%
%Das Besondere an DIM ist, dass es sowohl ein Pack- als auch ein
%Abdeckungsproblem darstellt. Es müssen genug Kanten gewählt werden,
%damit alle Kanten, die nicht zum Matching gehören mit einer des
%Matchings verbunden sind. Zwischen den Kanten des Matching müssen
%sich aber gleichzeitig immer mindestens zwei weitere Kanten befinden.
%Somit muss jede Kante genau ein mal gematcht werden.

\chapter{Graphen}\label{chp:Graphen}

Dieses Kapitel befasst sich mit den Grundlagen von Graphen. Es werden die in dieser Arbeit verwendeten Klassen definiert und in Verbindung gebracht. %Zusätzlich werden die Konzepte der Cliquenweite und der Potenz eines Graphen vorgestellt.

\section{Grundlagen}
Dieser Abschnitt enthält einige grundlegende Definitionen.

\begin{mydef}[Graph\index{Graph}]Ein Graph $G$ ist ein 2-Tupel $G=(V,E)$. Dabei ist
    $V$ eine endliche Menge von Knoten und $E \subseteq \{\{u,v\} \ |\ u,v \in V \wedge u \neq v\}$ die Menge der Kanten.

    Eine Kante $\{u,v\}$ wird abgekürzt durch $uv$.
\end{mydef}

\begin{mydef}[Teilgraph\index{Teilgraph}\label{def:Subgraph}]
    Es sei $T=(V_t,E_t)$ ein Graph. $T$ ist \emph{Teilgraph} des Graphen $G=(V,E)$ genau dann, wenn $V_t \subseteq V$ und $E_t \subseteq E$. Gilt für alle $u,v \in V_t$ zusätzlich $uv \in E \Rightarrow uv \in E_t$, dann ist $T$ ein \emph{induzierter Teilgraph}.

    Falls $T$ ein induzierter Teilgraph von $G$ ist, so wird $T$ auch als $G[V_t]$ bezeichnet.
\end{mydef}

\begin{mydef}
    Es seien $G$ und $F$ Graphen. $G$ ist \emph{$F$-frei} genau dann, wenn $F$ kein induzierter Teilgraph von $G$ ist.
\end{mydef}

\begin{mydef}[Nachbarschaft eines Knotens\index{Nachbarschaft}\index{$N[\cdot]$}\index{$N(\cdot)$}]
    Gegeben sei ein Graph $G=(V,E)$. Dann seien $N(v):=\{u\ |\ uv \in E\}$ die \emph{offene}
%    Nachbarschaft des Knotens $v$. Zusätzlich sei
und $N[v]:=N(v) \cup \{v\}$ die \emph{abgeschlossene Nachbarschaft} von $v$.
\end{mydef}

\section{Graphisomorphie}
Graphisomorphie bezeichnet vereinfacht gesagt, dass zwei Graphen gleich sind. Dies bedeutet bildlich gesprochen, dass man beide Graphen übereinander legen kann.

\begin{mydef}[Graphisomorphie\index{Graphisomorphie}\index{$\sim$|see{Isomorphie}}\index{Isomorphie!von Graphen}\label{def:Graphisomorphie}]
    Zwei Graphen $G_1 = (V_1,E_1)$ und $G_2 = (V_2,E_2)$ heißen \emph{isomorph} genau dann, wenn eine bijektive Abbildung $\varphi: V_1 \rightarrow V_2$ existiert, so dass für alle $u,v \in V_1$ gilt: \[ uv \in E_1 \Leftrightarrow \varphi(u)\varphi(v) \in E_2 \]
    
    Die Isomorphie zweier Graphen wird wie folgt dargestellt: $G_1 \sim G_2$
\end{mydef}

 
\section{Potenz eines Graphen}

Bei der Potenzierung eines Graphen wird dessen Kantendichte erhöht. Dazu wird für $G^i$ eine Kante zwischen zwei Knoten $u$ und $v$ eingefügt, wenn der Abstand $\delta$ von $u$ und $v$ in $G$ kleiner oder gleich $i$ ist.

\begin{mydef}[Potenz eines Graphen\index{Potenz eines Graphen}\index{$G^i$}]
    Es seien $G=(V,E)$ ein Graph und $\delta$ eine Funktion, die den Abstand zweier Knoten zueinander angibt. Dann sei $G^i=(V,E^i)$ wie folgt definiert:
    \[ E^i := \{uv \ |\ u,v \in V;\ u \neq v;\ \delta(u,v) \leq i\} \]
\end{mydef}

\begin{figure}[htbp]
    \centering
    \hspace*{\fill}
    \subfloat[\label{pic:bsp_GraphPower_G}]
    {
        \begin{tikzpicture}[thick]
        
            \node[nN] (m) at (0,0) {};
            
            \node[nN] (a) at (90:1) {};
            \node[nN] (b) at (-30:1) {};
            \node[nN] (c) at (210:1) {};
            
            \node[nN] (al) at ($(a.center)+(150:1)$) {};
            \node[nN] (ar) at ($(a.center)+(30:1)$) {};
            
            \node[nN] (bl) at ($(b.center)+(30:1)$) {};
            \node[nN] (br) at ($(b.center)+(-90:1)$) {};
            
            \node[nN] (cl) at ($(c.center)+(270:1)$) {};
            \node[nN] (cr) at ($(c.center)+(150:1)$) {};
            
            \foreach \n in {a,b,c}
            {
                \draw (m) -- (\n)
                      (\n l) -- (\n)
                      (\n r) -- (\n);
            }
        \end{tikzpicture}
    }
    \hspace*{\fill}
    \subfloat[\label{pic:bsp_GraphPower_GSquare}]
    {
        \begin{tikzpicture}[thick]
        
            \node[nN] (m) at (0,0) {};
            
            \node[nN] (a) at (90:1) {};
            \node[nN] (b) at (-30:1) {};
            \node[nN] (c) at (210:1) {};
            
            \node[nN] (al) at ($(a.center)+(150:1)$) {};
            \node[nN] (ar) at ($(a.center)+(30:1)$) {};
            
            \node[nN] (bl) at ($(b.center)+(30:1)$) {};
            \node[nN] (br) at ($(b.center)+(-90:1)$) {};
            
            \node[nN] (cl) at ($(c.center)+(270:1)$) {};
            \node[nN] (cr) at ($(c.center)+(150:1)$) {};
            
            \foreach \n in {a,b,c}
            {
                \draw[gray] (m) -- (\n)
                      (\n l) -- (\n)
                      (\n r) -- (\n);
                      
                \draw (m) -- (\n l)
                      (m) -- (\n r)
                      (\n l) -- (\n r);
            }
            
            \draw (a) -- (b) -- (c) -- (a);
        \end{tikzpicture}
    }
    \hspace*{\fill}
    \caption[Beispiel für die Potenz eines Graphen]{Beispiel für die Potenz eines Graphen: der Graph~\subref{pic:bsp_GraphPower_G} und dessen Quadrat~\subref{pic:bsp_GraphPower_GSquare}.}
    \label{pic:bsp_GraphPower}
\end{figure}


\section{Spezielle Graphen}
Bei den drei in diesem Abschnitt definierten Graphen handelt sich um individuelle Graphen. Die Abbildung~\ref{pic:bsp_HouseDominoGem} stellt alle drei dar.

\begin{mydef}[House\index{House}]
    Ein \emph{House} ist ein Graph $(V,E)$, der wie folgt definiert ist (Index Arithmetik modulo k):
    \begin{align*}
        V &=\{u,v_0,\ldots,v_3\} \\
        E &=\{v_iv_{i+1}\ | \ 0 \leq i \leq 3 \} \cup \{uv_1,uv_2\}
    \end{align*}
\end{mydef}

\begin{mydef}[Domino\index{Domino}]
    Ein \emph{Domino} ist ein Graph $(V,E)$, der wie folgt definiert ist:
    \begin{align*}
        V &=\{u_i,v_i \ |\ 1 \leq i \leq 3 \} \\
        E &=\{v_iv_{i+1} \ | \ 1 \leq i \leq 2 \} \cup \{u_iv_i \ | \ 1 \leq i \leq 3 \}
    \end{align*}
\end{mydef}

\begin{mydef}[Gem\index{Gem}]
    Ein \emph{Gem} ist ein Graph $(V,E)$, der wie folgt definiert ist:
    \begin{align*}
        V &=\{u,v_1,\ldots,v_4\} \text{ mit } \\
        E &=\{v_iv_{i+1}\ | \ 1 \leq i \leq 3 \} \cup \{uv_i \ | \ 1 \leq i \leq 4 \}
    \end{align*}
\end{mydef}

\floatimage{pic:bsp_HouseDominoGem}{Die Graphen House, Domino und Gem.}{
\hspace*{\fill}
\subfloat[House]{\begin{tikzpicture}
  [thick,node distance=.5cm]

  \foreach \x/\y in {45/a,135/b,225/c,315/d} {
     \node[nN] (\y) at (\x:0.75cm) {};
  }
 
  \node[nN] (e) at ($(45:0.75cm)+(120:1.06066cm)$) {};
  
  \foreach \x/\y in {a/b,b/c,c/d,d/a,a/e,b/e} {
      \draw (\x) -- (\y);
  }
  
 %\foreach \x in {45,135,-45,-135} {
 %    \node at (\x:1.1cm) {};
 %}
 
  \node[hN] at (-1.125,0) {};
  \node[hN] at (1.25,0) {};

\end{tikzpicture}}
\hspace*{\fill}
\subfloat[Domino]{\begin{tikzpicture}
  [thick,node distance=1.06066cm]

  \foreach \y in {1,2}
  {
  \foreach \x in {1,2,3}
  {
    \node[nN] (v\x\y) at (${\x-2}*(1.06066cm,0)+\y*(0,1.06066cm)$) {};
  }
  }
 
  \foreach \x in {1,2,3}
  {
    \draw (v\x1) -- (v\x2);
  }
  
  \foreach \x/\y in {1/2,2/3}
  {
    \draw (v\x1) -- (v\y1);
    \draw (v\x2) -- (v\y2);
  }
 
  %\foreach \x/\y in {a/b,b/c,c/d,d/a,a/c} {
  %    \draw (\x) -- (\y);
  %}

  %\foreach \x in {45,135,-45,-135} {
  %   \node at (\x:1.1cm) {};
  %}
 
 
  %\node[hN] at (-1.125,0) {};
  %\node[hN] at (1.25,0) {};
  
\end{tikzpicture}}
\hspace*{\fill}
\subfloat[Gem]{\begin{tikzpicture}
  [thick,node distance=.5cm]

        \def\len{1.25}
        
        \node[nN] (a) at (150:\len) {};
        \node[nN] (b) at (110:\len) {};
        \node[nN] (c) at (70:\len) {};
        \node[nN] (d) at (30:\len) {};
        \node[nN] (e) at (0,0) {};


  %\node[hN] at (-1.125,0) {};
  %\node[hN] at (1.25,0) {};

%  \node[nN] (a) at (-0.905,0) {};
%  \node[nN] (d) at  (0.905,0) {};
%  \node[nN] (b) [above right=of a] {};
%  \node[nN] (c) [above left=of d] {};
%  
%  \node[nN] (e) at (-90:0.65cm) {};
        
 %\foreach \x/\y in {0/a,60/b,120/c,180/d,-90/e} {
 %    \node[nN] (\y) at (\x:0.75cm) {};
 %}
 
 \foreach \x/\y in {a/b,b/c,c/d} {
     \draw (\x) -- (\y);
 }

 \foreach \x in {a,b,c,d} {
     \draw (\x) -- (e);
 }

 %\foreach \x in {45,135,-45,-135} {
 %    \node at (\x:1.1cm) {};
 %}
  
\end{tikzpicture}}
\hspace*{\fill}
}


\section{Einfache Klassen}
Die Klassen in diesem Abschnitt sind sehr einfache Klassen. Graphen innerhalb dieser Klassen unterscheiden sich lediglich durch ihre Größe voneinander. Die Abbildungen~\ref{pic:bsp_WCK} und \ref{pic:bsp_Sun} zeigen für jede Klasse einen Beispielgraphen der Größe~5.

\begin{mydef}[Kreis\index{Kreis}, $C_k$\index{$C_k$}]
    Ein \emph{sehnenloser Kreis} (engl. chordless cycle) der Länge $k$ ($k\geq 3$) ist ein Graph mit $k$ Knoten $v_1, \ldots, v_k$ und den Kanten $v_iv_{i+1}, 1 \leq i \leq k$ und $v_kv_1$.

    Kreise der Länge $k\geq 5$ werden auch als \emph{Hole}\index{Hole}\index{Hole|see{Kreis}} bezeichnet.
\end{mydef}

%\begin{mydef}[Rad\index{Rad}, $W_k$\index{$W_k$}]
%    Gegeben sei ein $C_k=(V_c,E_c)$. Ein Rad-Graph (engl. wheel graph) $W_k=(V_w,E_w)$ der Größe $k$ sei dann wie folgt definiert:
%    \begin{align*}
%        V_w&:=V_c \cup \{v\} \text{ mit }v \notin V_c \\
%        E_w&:=E_c \cup \{uv \ |\ u \in V_c\}
%    \end{align*}
% %\begin{itemize}
% %  \item $V_w:=V_c \cup \{v\}$ mit $v \notin V_c$
% %  \item $E_w:=E_c \cup \{uv \ |\ u \in V_c\}$
% %\end{itemize}
%\end{mydef}

\begin{mydef}[Clique\index{Clique}, $K_i$\index{$K_i$}]
    Ein Graph $G=(V,E)$ ist eine Clique der Größe~$i$ genau dann, wenn $|V|=i$ und $E=\{uv \ |\ u,v \in V \wedge u \neq v\}$.
\end{mydef}

Man bezeichnet Cliquen auch als vollständige Graphen. Das Ermitteln einer größten Clique in einem Graphen (also eines größten Teilgraphen, der eine Clique darstellt) ist NP-vollständig \cite{Karp72}.

\begin{figure}[htb]
    \centering
\hspace*{\fill}
\subfloat[Ein Kreis der Länge 5 ($C_5$)]{\begin{tikzpicture}
  [thick]

  \foreach \x/\y in {18/a,90/b,162/c,234/d,306/e} {
     \node[nN] (\y) at (\x:0.75cm) {};
  }
 
  \foreach \x/\y in {a/b,b/c,c/d,d/e,e/a} {
      \draw (\x) -- (\y);
  }

  \foreach \x in {45,135,-45,-135} {
     \node at (\x:1.1cm) {};
  }
 
 
  \node[hN] at (-1.125,0) {};
  \node[hN] at (1.25,0) {};
  
\end{tikzpicture}}
\hspace*{\fill}
%\subfloat[Ein Rad der Größe 5 ($W_5$)]{\begin{tikzpicture}
%  [thick,node distance=.5cm]
%
%  \foreach \x/\y in {18/a,90/b,162/c,234/d,306/e} {
%     \node[nN] (\y) at (\x:0.75cm) {};
%  }
% 
%  \node[nN] (center) at (0,0) {};
%  
%  \foreach \x/\y in {a/b,b/c,c/d,d/e,e/a} {
%      \draw (\x) -- (\y);
%      \draw (\x) -- (center);
%  }
%  
% \foreach \x in {45,135,-45,-135} {
%     \node at (\x:1.1cm) {};
% }
% 
%  \node[hN] at (-1.125,0) {};
%  \node[hN] at (1.25,0) {};
%
%\end{tikzpicture}}
%\hspace*{\fill}
\subfloat[Eine Clique der Größe 5 ($K_5$)]{\begin{tikzpicture}
  [thick]

  \node[hN] at (-1.125,0) {};
  \node[hN] at (1.25,0) {};
      
  \foreach \x/\y in {18/a,90/b,162/c,234/d,306/e} {
     \node[nN] (\y) at (\x:0.75cm) {};
  }
 
 \foreach \x/\y in {a/b,a/c,a/d,a/e,
                 b/c,b/d,b/e,
                 c/d,c/e,
                 d/e} {
     \draw (\x) -- (\y);
 }

 \foreach \x in {45,135,-45,-135} {
     \node at (\x:1.1cm) {};
 }
  
\end{tikzpicture}}
\hspace*{\fill}
\caption{Die Graphen $C_5$ und $K_5$.}
\label{pic:bsp_WCK}
\end{figure}

\begin{mydef}[Sun\index{Sun}, $S_k$\index{$S_k$}]
    Gegeben seien die Knoten $V_u=\{u_0,\ldots,u_{k-1}\}$ und $V_v=\{v_0,\ldots,v_{k-1}\}$. Eine \emph{complete Sun} $S_k=(V_s,E_s)$ der Größe $k$ ($k\geq 3$) sei dann wie folgt definiert (Index Arithmetik modulo $k$):
    \begin{align*}
        V_s&:=V_u \cup V_v \\
        E_s&:=\{u_iv_i,u_iv_{i+1},v_iv_j \ |\ 0\leq i,j<k ;\ i \neq j \}
    \end{align*}
\end{mydef}

\begin{figure}[htb]
\centering
\begin{tikzpicture}
  [thick]

  \foreach \x/\y in {18/a,90/b,162/c,234/d,306/e} {
     \node[nN] (\y) at (\x:0.75cm) {};
  }
 
  \foreach \x/\y in {54/a,126/b,198/c,270/d,342/e} {
     \node[nN] (\y\y) at (\x:1.2135cm) {};
  }
 
 \foreach \x/\y in {a/b,a/c,a/d,a/e,
                 b/c,b/d,b/e,
                 c/d,c/e,
                 d/e} {
     \draw (\x) -- (\y);
 }

 \foreach \x/\y in {a/b,b/c,c/d,d/e,e/a} {
     \draw (\x) -- (\x\x);
     \draw (\x\x) -- (\y);
 }

\end{tikzpicture}
\caption{Der Graph $S_5$}
\label{pic:bsp_Sun}
\end{figure}

\section{Chordale Graphen}\label{sec:chordalGraphs}
Unter chordalen Graphen versteht man Graphen, die keine Kreise der Größe~$4$ oder größer als induzierte Teilgraphen besitzen. 
Man bezeichnet sie in der (engl.) Literatur auch als \emph{triangulated graphs}, \emph{rigid-circuit graphs}, \emph{monotone transitive graphs} und \emph{perfect elemination graphs} \cite{brandstaedt1999graph}.

\begin{mydef}[chordaler Graph\index{chordal}]
    Ein Graph $G$ ist \emph{chordal} genau dann, wenn er $C_k$-frei ($k\geq4$) ist.
\end{mydef}

Chordale Graphen besitzen einige hilfreiche Eigenschaften. Beispielsweise lassen sich einige NP-vollständige Probleme bei chordalen Graphen in Polynomial- oder gar Linearzeit lösen.

\subsection{Perfekte Eliminationsordnungen}
Bei einer \emph{perfekten Eliminationsordnung} handelt es sich um eine spezielle Knotenreihenfolge. Dazu ist es jedoch nötig, \emph{simpliziale Knoten} zu definieren.

\begin{mydef}[simplizialer Knoten\index{simplizial}]
    Ein Knoten $v$ im Graphen $G$ ist \emph{simplizial} genau dann, wenn $N(v)$ eine Clique in $G$ ist.
\end{mydef}

Jeder nichtleere, chordale Graph $G$ besitzt einen simplizialen Knoten; ist $G$ keine Clique, so gibt es sogar zwei simpliziale Knoten in $G$, die nicht miteinander verbunden sind \cite{Dirac1961}. Es ist somit möglich, die Knoten eines chordalen Graphen schrittweise zu eliminieren, indem man vom verbleibenden Graphen einen simplizialen Knoten entfernt. Diese Knotenreihenfolge wird als \emph{perfekte Eliminationsordnung} bezeichnet.

\begin{mydef}[perfekte Eliminationsordnung\index{Eliminationsordnung!perfekte}]
    Gegeben sei ein Graph $G=(V,E)$ mit $|V|=n$. Eine Folge $(v_1,\ldots,v_n)$ von Knoten ist eine \emph{perfekte Eliminationsordnung} für $G$ genau dann, wenn für alle $i\in\{1,\ldots,n\}$ gilt: $v_i$ ist simplizial in $G[\{v_i,\ldots,v_n\}]$.
\end{mydef}

\begin{Theorem}\cite{Dirac1961}
    Ein Graph $G$ ist chordal\index{chordal} genau dann, wenn $G$ eine perfekte Eliminationsordnung\index{Eliminationsordnung!perfekte} besitzt.
\end{Theorem}

Es ist in Linearzeit möglich, die Chordalität eines Graph zu erkennen und eine perfekte Eliminationsordnung zu bilden, falls er chordal ist \cite{Rose1976}.

\subsection{Strongly chordale Graphen\index{strongly chordal}}
Eine (echte) Teilmenge der chordalen Graphen sind die strongly chordalen Graphen. Es gibt verschiedene (äquivalente) Definitionen für sie. Eine davon ist über das Verbot von Suns als induzierte Teilgraphen.

\begin{mydef}[strongly chordaler Graph\index{chordal!strongly}\index{strongly chordal}]
    Ein Graph $G$ ist \emph{strongly chordal} genau dann, wenn $G$ chordal und $S_k$-frei ($k\geq 3$) ist.
\end{mydef}

Strongly chordale Graphen besitzen als chordale Graphen eine perfekte Eliminationsordnung. Es gibt allerdings auch andere Eliminationsordnungen für strongly chordale Graphen. Eine davon wird als \emph{strong perfect elemination ordering} bezeichnet.

\begin{mydef}[strong perfect elimination ordering\index{Eliminationsordnung!strong (perfect)}, \cite{Faber1983}]\label{def:strongPerfectEO}
    Eine perfekte Eliminationsordnung $(v_1,\ldots,v_n)$ eines Graphen $G=(V,E)$ ist \emph{strong perfect} genau dann, wenn für alle $i<j,k<l$ gilt:
    %\[ v_iv_l \in E \land v_iv_k \in E \land v_kv_j \in E \Rightarrow v_lv_j \in E \]
    \[ \{v_iv_l, v_iv_k, v_jv_k \} \subseteq E \Rightarrow v_lv_j \in E \]
\end{mydef}

\begin{figure}[htb]
\centering
        \begin{tikzpicture}
           [thick,lbl/.style={font=\small},llbl/.style={left,lbl},rlbl/.style={lbl,right}]
           
           \node[nN] (i) at (180:1.5cm) {}; \node[llbl] at (i.west) {$i$};
           \node[nN] (j) at (120:1.5cm) {}; \node[llbl] at (j.west) {$j$};
           \node[nN] (k) at (60:1.5cm)  {}; \node[rlbl] at (k.east) {$k$};
           \node[nN] (l) at (0:1.5cm)   {}; \node[rlbl] at (l.east) {$l$};
           
           \begin{pgfonlayer}{background}
               \draw[dashed]
                   (i) -- (j)
                   (k) -- (l);

               \draw[very thick,clDark25Green]
                   (i.center) -- (l.center)
                   (i.center) -- (k.center)
                   (j.center) -- (k.center);
           
               \draw[very thick,clBlue,decoration={snake,amplitude=1},decorate] (j) -- (l);
            
           \end{pgfonlayer}
        \end{tikzpicture}
\caption[Skizze für Definition \ref{def:strongPerfectEO}]
    {Skizze für Definition \ref{def:strongPerfectEO}.  Wenn die Kanten $il$, $ik$ und $jk$ (grün) vorhanden sind, muss auch die Kante $jl$ (blau gewellt) vorhanden sein. Über die Kanten $ij$ und $kl$ (gestrichelt) wird keine Aussage gemacht.}
\label{pic:SkizzeBeweisStrPerfEO}
\end{figure}

Eine strong perfect elimination ordering lässt sich mit einem Zeitaufwand von $\mO(n^3)$ für einen gegebenen strongly chordalen Graphen finden \cite{Anstee1984}.

Eine weitere Eliminationsordnung ist die \emph{simple Eliminationsordnung}. Dabei müssen die entfernten Knoten sogenannte \emph{simple Knoten} sein.

\begin{mydef}[simpler Knoten\index{simple}, \cite{Faber1983}]\label{def:simpleNode}
    Ein Knoten $v$ ist ein \emph{simpler Knoten} im Graphen $G$ genau dann, wenn für alle seine Nachbarn $x,y\in N[v]$ gilt:
    \[ N[x] \subseteq N[y] \text{ oder } N[y] \subseteq N[x] \]
\end{mydef}

Aus Definition \ref{def:simpleNode} folgt, dass es für die Nachbarn $u_1,\ldots,u_i$ eines simplen Knotens $v$ eine lineare Ordnung auf deren Nachbarschaft gibt: \[ N[u_1] \subseteq \ldots \subseteq N[u_i] \]

\begin{mydef}[simple Eliminationsordnung\index{Eliminationsordnung!simple}, \cite{Faber1983}]\label{def:simpleEO}
    Eine Eliminationsordnung $v_1,\ldots,v_n$ ist eine \emph{simple Eliminationsordnung} genau dann, wenn für alle $i \in \{1, \ldots, n\}$ gilt: $v_i$ ist simpel in $G[\{v_i,\ldots,v_n\}]$.
\end{mydef}

\begin{Lemma}\label{lem:StrCordSimplNode}
    \cite{Faber1983} Ein Graph $G$ ist strongly chordal\index{chordal!strongly}\index{strongly chordal} genau dann, wenn jeder induzierte Teilgraph von $G$ einen simplen\index{simple} Knoten besitzt.
\end{Lemma}

Aus Lemma~\ref{lem:StrCordSimplNode} folgt nun unmittelbar Satz \ref{theo:StrCordSimplEO}:

\begin{Theorem}\label{theo:StrCordSimplEO}
    Ein Graph $G$ ist strongly chordal\index{chordal!strongly}\index{strongly chordal} genau dann, wenn $G$ eine simple Eliminationsordnung\index{Eliminationsordnung!simple} hat.
\end{Theorem}

\subsection{Intervallgraphen}

 Die Idee bei Intervallgraphen ist es, die Überschneidung von verschiedenen Zeitintervallen darzustellen. Dazu wird für jedes Intervall ein Knoten erstellt und diese dann miteinander verbunden, wenn sich die entsprechenden Intervalle überschneiden. Abbildung~\ref{pic:bsp_Interval} zeigt dies an einem Beispiel.

\begin{mydef}[Intervallgraph\index{Intervallgraph}]
    Ein Graph $G=(V,E)$ ist ein Intervallgraph genau dann, wenn eine Menge von Intervallen $\mI =\{I_1,\ldots,I_n\}$ existiert, sodass gilt:
    \begin{align*}
        V &= \{v_1, \ldots, v_n\} \\
        E &= \{v_iv_j \ |\ I_i \cap I_j \neq \emptyset ;\ i \neq j \}
    \end{align*}
\end{mydef}

\begin{figure}[htbp]
    \centering
    \subfloat[\label{pic:bsp_Interval_Int}]{
    \begin{tikzpicture}
        \foreach \n/\f/\t/\h/\c in {a/1/7/2/clOrange,
                                    b/2/4/3/clViolet,
                                    c/3/15/1/clRed,
                                    d/6/14/3/clBlue,
                                    e/9/11/2/clAqua,
                                    f/13/18/2/clGreen,
                                    g/17/19/1/black}
        {
            \draw[very thick,\c] ($0.5*(\f,\h)$) -- ($0.5*(\t,\h)$)
                              ($0.5*(\f,\h)+0.5*(0,0.18)$) -- ($0.5*(\f,\h)-0.5*(0,0.18)$)
                              ($0.5*(\t,\h)+0.5*(0,0.18)$) -- ($0.5*(\t,\h)-0.5*(0,0.18)$);
            \node[llbl] at ($0.5*(\f,\h)$) {\n};
        }
    \end{tikzpicture}
    }
    
    \subfloat[\label{pic:bsp_Interval_Graph}]{
    \begin{tikzpicture}[thick]
    
        \node[nN,fill=clOrange] (a) at (120:2) {};
        \node[nN,fill=clViolet] (b) at (180:2) {};
        \node[nN,fill=clRed] (c) at (0,0) {};
        \node[nN,fill=clBlue] (d) at (60:2) {};
        \node[nN,fill=clAqua] (e) at (30:1.1547) {};
        \node[nN,fill=clGreen] (f) at (0:2) {};
        \node[nN,fill=black] (g) at ($(f.center)+(30:1.1547)$) {};
        
%        \foreach \n/\f/\t/\h/\c in {a/2/7/3/clOrange,
%                                    b/2/4/1/clViolet,
%                                    c/3/10/1/clRed,
%                                    d/4/13/3/clBlue,
%                                    e/8/9/2/clAqua,
%                                    f/8/13/1/clGreen,
%                                    g/12/14/2/black}
%        {
%            \node[nN,fill=\c] (\n) at ($0.25*(\f,0) + 0.25*(\t,0) + (0,\h)$) {};
%        }
        
        \node[llbl] at (a.west) {a};
        \node[blbl] at (b.south) {b};
        \node[blbl] at (c.south) {c};
        \node[rlbl] at (d.east) {d};
        \node[blbl] at (e.south) {e};
        \node[blbl] at (f.south) {f};
        \node[blbl] at (g.south) {g};
            
        \foreach \f/\t in {a/b,a/c,a/d,b/c,c/d,c/e,c/f,d/e,d/f,f/g}
        {
            \draw (\f)--(\t);
        }
    \end{tikzpicture}
    }
    
    \caption[Eine Mengen von Intervallen und der dazugehörige Intervallgraph]
    {Eine Mengen von Intervallen \subref{pic:bsp_Interval_Int} und der dazugehörige Intervallgraph \subref{pic:bsp_Interval_Graph} (Vorlage:~\cite{wikiMedia:IntervalGraph})}
    \label{pic:bsp_Interval}
\end{figure}

Intervallgraphen sind strongly chordal \cite[Abschnitt~4.4]{brandstaedt1999graph}\cite{FarberPhD} und können in Linearzeit erkannt werden \cite{Booth1976}.

\section{Distanzerbliche Graphen}
Bei distanzerblichen Graphen handelt es sich um eine Klasse von Graphen, in denen der Abstand zwischen zwei Knoten in jedem zusammenhängenden induzierten Teilgraph gleich ist. Das heißt, dass (so lange es einen Pfad gibt) sich die Entfernung zwischen zwei Knoten nicht verändert, unabhängig davon, wie viele Knoten entfernt werden.

Ursprünglich wurden distanzerbliche Graphen in \cite{Howorka1977} vorgestellt. In \cite{Bandelt1986} wurde gezeigt, dass sie sich auch über das Verbot von Teilgraphen definieren lassen. Dabei handelt es sich um die Graphen House, Hole ($C_k$, $k\geq 5$), Domino und Gem.

\begin{mydef}[distanzerblicher Graph\index{distanzerblich}, \cite{Bandelt1986}]
    Ein Graph $G$ ist \emph{distanzerblich} (engl.: distance hereditary) genau dann, wenn $G$ (House, Hole, Domino, Gem)-frei ist.
\end{mydef}

\begin{figure}[htbp]
    \centering
    
    \begin{tikzpicture}
        [thick]
        
        \node[nN] (a) at (150:1.5) {};
        \node[nN] (b) at (210:1.5) {};
        \node[nN] (c) at ($(210:1.5)+(-90:1.5)$) {};
        \node[nN] (d) at (90:1.5) {};
        \node[nN] (e) at (0,0) {};
        \node[nN] (f) at (-90:1.5) {};
        \node[nN] (g) at (30:1.5) {};
        \node[nN] (h) at (-30:1.5) {};
        \node[nN] (i) at ($2*(30:1.5)$) {};
        \node[nN] (j) at ($(30:1.5)+(-30:1.5)$) {};
        
        \draw (a) -- (d)
              (a) -- (e)
              (a) -- (f);
        
        \draw (b) -- (c)
              (b) -- (d)
              (b) -- (e)
              (b) -- (f);
        
        \draw (d) -- (g)
              (d) -- (h);
        
        \draw (e) -- (g)
              (e) -- (h);

        \draw (f) -- (g)
              (f) -- (h);
        
        \draw (g) -- (h)
              (g) -- (i)
              (g) -- (j);
        
        \draw (i) -- (j);

    \end{tikzpicture}
    
    \caption[Beispiel für einen distanzerblichen Graphen]{Beispiel für einen distanzerblichen Graphen (Vorlage: \cite{wikiMedia:DistanceHereditaryGraph})}
    \label{pic:bsp_DistanceHereditaryGraph}
\end{figure}


Zwar sind Holes (Kreise der Länge~5 oder größer) verboten, aber nicht Kreise der Länge~4. Somit sind distanzerbliche Graphen nicht notwendigerweise chordal. Umgekehrt beinhalten chordale Graphen bereits per Definition kein House, Hole oder Domino als induzierten Teilgraphen. Es gilt somit Lemma~\ref{lem:GemFreeChordalDistanzerblich}.

\begin{Lemma}\label{lem:GemFreeChordalDistanzerblich}
    Ein chordaler\index{chordal} Graph $G$ ist Gem-frei $\Rightarrow$ $G$ ist distanzerblich\index{distanzerblich}.
\end{Lemma}

Da jede Sun ein Gem enthält, ist ein chordaler distanzerblicher Graph auch immer strongly chordal.

\section{Dually chordale Graphen}\label{sec:DuallyChordalGraphs}

Bei dually chordalen Graphen handelt es sich um eine Graphenklasse, die sich über eine Nachbarschaftsordnung ihrer Knoten definiert. Basis dafür ist ein sogenannter maximaler Nachbar.

\begin{mydef}[maximaler Nachbar\index{Nachbar!maximal}]
    Ein Knoten $u\in N[v]$ ist ein \emph{maximaler Nachbar} von $v$ genau dann, wenn gilt:
    \[ \forall\,w \in N[v]: N[w] \subseteq N[u] \]
    \end{mydef}

Kann nun ein Graph $G$ schrittweise reduziert werden, wobei jeder entfernte Knoten $v$ einen maximalen Nachbarn $u$ im Restgraphen hat, spricht man von einer maximalen Nachbarschaftsordnung. Dies schließt auch die Möglichkeit ein, dass $v$ sein eigener maximaler Nachbar ist ($v=u$).

\begin{mydef}[maximale Nachbarschaftsordnung\index{Nachbarschaftsordnung!maximal}]
    Eine Ordnung von Knoten $(v_1,\ldots,v_n)$ ist eine \emph{maximale Nachbarschaftsordnung} genau dann, wenn für alle $i\in\{1,\ldots,n\}$ jeder Knoten $v_i$ einen maximalen Nachbar in $G[\{v_i,\ldots,v_n\}]$ hat.
\end{mydef}

Dually chordale Graphen definieren sich nun durch eine maximale Nachbarschaftsordnung.

\begin{mydef}[dually chordaler Graph\index{dually chordal}\index{chordal!dually}\label{def:duallyChordal}]
    Ein Graph $G$ ist \emph{dually chordal} genau dann, wenn $G$ eine maximale Nachbarschaftsordnung hat.
\end{mydef}

Im Laufe der Arbeit werden noch weitere Charakterisierungen für dually chordale Graphen genannt. Satz~\ref{theo:duallyChordalChar} (S.~\pageref{theo:duallyChordalChar}) fasst diese zusammen.

Jeder Graph $G$ lässt sich in einen dually chordalen Graphen $G'$ umwandeln, indem man einen Knoten $v$ einfügt, wobei $v$ mit allen Knoten in $G$ benachbart ist. Auf diese Weise ist $v$ ein maximaler Nachbar für alle anderen Knoten. Als Folge daraus kann ein dually chordaler Graph jeden Graphen als Teilgraphen haben. Somit sind dually chordale Graphen nicht notwendigerweise chordal.

Man kann die Definition der dually chordalen Graphen erweitern, indem auch jeder induzierte Teilgraph dually chordal sein muss. Die so entstehende Klasse nennt man \emph{hereditary dually chordal}.

\begin{Theorem}\cite{duallyChordal}\label{theo:hereDuCh_strCh}
    Ein Graph $G$ ist strongly chordal\index{chordal!strongly}\index{strongly chordal} genau dann, wenn jeder induzierte Teilgraph von $G$ dually chordal\index{chordal!dually}\index{dually chordal} ist.
\end{Theorem}

Zwar sind die strongly chordalen Graphen eine Teilmenge der dually chordalen Graphen, aber nicht jeder Graph, der chordal und dually chordal ist, ist auch strongly chordal. Abbildung~\ref{pic:bsp_chDuCh_notStrCh} stellt einen solchen Graphen dar.

\begin{figure}[htbp]
    \centering
    \begin{tikzpicture}
        [thick]
        
        \node[nN] (m) at (0,0) {};
        
        \node [nN] (b) at ($(m.center)+(150:0.866025)$) {};
        \node [nN] (c) at ($(b.center)+(0:1.5)$) {};
        \node [nN] (e) at ($(b.center)+(-60:1.5)$) {};
        
        
        \node [nN] (a) at ($(b.center)+(60:1.5)$) {};
        \node [nN] (d) at ($(b.center)-(60:1.5)$) {};
        \node [nN] (f) at ($(c.center)+(-60:1.5)$) {};

        \draw (a)--(b)--(d)--(e)--(b)--(c)--(e)--(f)--(c)--(a);
        
        \foreach \n in {a,b,c,d,e,f}
        {
            \draw[] (m) -- (\n);
        }
        
    \end{tikzpicture}
    \caption{Ein Graph der chordal, dually chordal jedoch nicht strongly chordal ist.}
    \label{pic:bsp_chDuCh_notStrCh}
\end{figure}


\section{Bäume}
Bäume sind eine der bekanntesten Graphenklassen. Bei ihnen handelt es sich um zusammenhängende Graphen ohne Zyklen ($C_k$, $k\geq 3$). Nicht zusammenhängende Graphen, die frei von Zyklen sind, werden als Wald bezeichnet.

\begin{mydef}[Baum\index{Baum}]
    Ein zusammenhängender Graph ist ein Baum genau dann, wenn er keinen Zyklus besitzt.
\end{mydef}

Es gibt noch weitere Charakterisierungen für Bäume: Beispielsweise ist ein Graph $(V,E)$ ein Baum genau dann, wenn er zusammenhängend ist und $|V| = |E|+1$ ist.

%Es ist möglich, jeden Baum wie folgt zu konstruieren: Man startet mit einem einzelnen Knoten. Ist bereits ein Baum $T=(V,E)$ vorhanden, so fügt man einen neuen Knoten $v$ hinzu sowie genau eine Kante $uv$ ($u\in V$).

\subsection{Spannbäume}
Eine Anwendung von Bäumen ist das Konzept der Spannbäume. Hierbei handelt es sich um einen Baum~$T$, der ein (in der Regel nicht induzierter) Teilgraph eines Graphen~$G$ ist. Allerdings ist jeder Knoten von $G$ auch in $T$ vorhanden.

\begin{mydef}[Spannbaum\index{Spannbaum}]
    Gegeben sei ein zusammenhängender Graph $G=(V_g,E_g)$ und ein Baum $T=(V_t,E_t)$. $T$ ist ein \emph{Spannbaum} von $G$ genau dann, wenn die beiden nachfolgenden Bedingungen erfüllt sind:
    \begin{align*}
       V_t &= V_g \\
       E_t &\subseteq E_g
    \end{align*}
\end{mydef}

Es ist leicht zu sehen, dass jeder zusammenhängende Graph auch einen Spannbaum besitzt. Dieser muss jedoch nicht eindeutig sein. Abbildung~\ref{pic:bsp_SpanningTree} stellt einen Graphen mit einem Spannbaum dar.

\begin{figure}[htbp]
    \centering
    \hspace*{\fill}
    \subfloat[\label{pic:bsp_SpanningTree_Graph}]{
    \begin{tikzpicture}
        [thick]
        
        \def\len{1.25}
        \node[nN] (a) at (150:\len) {};
        \node[nN] (b) at (210:\len) {};
        \node[nN] (c) at ($(210:\len)+(-90:\len)$) {};
        \node[nN] (d) at (90:\len) {};
        \node[nN] (e) at (0,0) {};
        \node[nN] (f) at (-90:\len) {};
        \node[nN] (g) at (30:\len) {};
        \node[nN] (h) at (-30:\len) {};
        \node[nN] (i) at ($2*(30:\len)$) {};
        \node[nN] (j) at ($(30:\len)+(-30:\len)$) {};
        
        \draw (a) -- (d)
              (a) -- (e)
              (a) -- (f);
        
        \draw (b) -- (c)
              (b) -- (d)
              (b) -- (e)
              (b) -- (f);
        
        \draw (d) -- (g)
              (d) -- (h);
        
        \draw (e) -- (g)
              (e) -- (h);

        \draw (f) -- (g)
              (f) -- (h);
        
        \draw (g) -- (h)
              (g) -- (i)
              (g) -- (j);
        
        \draw (i) -- (j);

    \end{tikzpicture}
    }
    \hspace*{\fill}
    \subfloat[\label{pic:bsp_SpanningTree_Tree}]{
    \begin{tikzpicture}
        [thick]
        
        \def\len{1.25}

        \node[nN] (a) at (150:\len) {};
        \node[nN] (b) at (210:\len) {};
        \node[nN] (c) at ($(210:\len)+(-90:\len)$) {};
        \node[nN] (d) at (90:\len) {};
        \node[nN] (e) at (0,0) {};
        \node[nN] (f) at (-90:\len) {};
        \node[nN] (g) at (30:\len) {};
        \node[nN] (h) at (-30:\len) {};
        \node[nN] (i) at ($2*(30:\len)$) {};
        \node[nN] (j) at ($(30:\len)+(-30:\len)$) {};
        
        \draw[lightgray]
            (a) -- (d)
            (a) -- (f)
            (b) -- (d)
            (d) -- (h)
            (f) -- (g)
            (f) -- (h)
            (g) -- (h)
            (i) -- (j);

        \draw
            (a) -- (e)
            (b) -- (c)
            (b) -- (e)
            (b) -- (f)
            (d) -- (g)
            (e) -- (g)
            (e) -- (h)
            (g) -- (i)
            (g) -- (j);

    \end{tikzpicture}
    }
    \hspace*{\fill}

    \caption[Beispiel für einen Graphen und einen Spannbaum]{Beispiel für einen Graphen~\subref{pic:bsp_SpanningTree_Graph} und einen Spannbaum~\subref{pic:bsp_SpanningTree_Tree} \\ (Vorlage für \subref{pic:bsp_SpanningTree_Graph}: \cite{wikiMedia:DistanceHereditaryGraph})}
    \label{pic:bsp_SpanningTree}
\end{figure}

Spannbäume bieten eine weitere Möglichkeit, dually chordale Graphen zu charakterisieren.

\begin{Theorem}\label{theo:DuallyChordalSpanningTree} \cite{duallyChordal}
    Ein Graph $G$ ist dually chordal\index{chordal!dually}\index{dually chordal} genau dann, wenn er einen Spannbaum~$T$ besitzt, so dass jede maximale Clique in $G$ einen Teilbaum von $T$ induziert.
\end{Theorem}


\subsection{Blätter}
Blätter sind eine spezielle Sorte von Knoten, die man üblicherweise mit Bäumen in Verbindung bringt. Dabei handelt es sich um Knoten, die nur einen Nachbarn haben.

\begin{mydef}[Blatt\index{Blatt}]
    In einem Graphen ist ein Knoten $v$ ein \emph{Blatt} genau dann, wenn $v$ genau einen Nachbarn hat ($|N(v)|=1$).
\end{mydef}

Es ist leicht zu sehen, dass jeder Baum (mit zwei oder mehr Knoten) mindestens zwei Blätter besitzt. Trotzdem können auch in anderen Graphen Blätter vorkommen.




\section{Cliquenweite}

Bei Cliquenweite handelt es sich um ein Komplexitätsmaß für Graphen. Als Basis dienen vier Operationen, mit denen sich jeder Graph erzeugen lässt: 

%\begin{itemize}
%    \item [$\odot_i$]  Es wird ein neuer Knoten erstellt und mit dem Label~$i$ versehen.
%
%\item[$G_1 \oplus G_2$]  Wurden bereits zwei Graphen $G_1$ und $G_2$ erzeugt, werden sie nun als ein Graph $G$ betrachtet. Es werden jedoch keine neuen Kanten erstellt. Somit ist $G$ nicht zusammenhängend.
%
%\item [$\eta_{i,j}(G)$]  Bei dieser Operation werden zwei Knoten~$u$ und $v$ miteinander verbunden, wenn $u$ das Label~$i$ und $v$ das Label~$j$ hat. Diese Operation betrifft alle Knoten im Graphen mit den entsprechenden Labels. Außerdem gilt, dass $i \neq j$ sein muss. Es ist die einzige Möglichkeit, um Kanten zu erzeugen.
%
%\item[$\rho_{i \rightarrow j}(G)$]  Vorhandene Labels können umbenannt werden. Dabei erhalten alle Knoten mit dem Label~$i$ das Label~$j$. Es ist nicht möglich nur einen einzelnen Knoten umzubenennen.
%\end{itemize}

\LTXtable{\textwidth}{tex-Dateien/CliquenweiteOperationen}

Die Cliquenweite eines Graphen definiert sich nun über diese vier Operationen.

\begin{mydef}[Cliquenweite\index{Cliquenweite}]
    Die Cliquenweite eines Graphen~$G$ ist die minimal notwendige Anzahl an verschiedenen Labels, die nötig ist, um $G$ mit den Operationen $\odot_i$, $\oplus$, $\eta_{i,j}$ und $\rho_{i \rightarrow j}$ zu erzeugen.
\end{mydef}

%\subsection{Beispiel}
Anhand eines Gems sei nachfolgend demonstriert, wie sich mit den oben genannten Operationen ein Graph erzeugen lässt. Abbildung~\ref{pic:bsp_CliqueWidthGem} stellt dies graphisch dar.

\begin{figure}[htbp]
    \centering
    %\hspace*{\fill}
    \subfloat[\label{pic:bsp_CliqueWidthGem_a}$\eta_{1,3}(\odot_1 \oplus \odot_3)$]{
    \begin{tikzpicture}
        [thick,grlbl/.style={lbl,circle,draw=clDark25Green,fill=clLight60Green,inner sep=2pt},
         bllbl/.style={lbl,circle,draw=clDark25Blue,fill=clLight80Blue,inner sep=2pt},
         orlbl/.style={lbl,circle,draw=clDark25Orange,fill=clLight60Orange,inner sep=2pt}]
        \def\len{1.25}
        
        \node[hN] (a) at (180:\len) {};
        \node[hN] (b) at (120:\len) {};
        \node[orlbl] (c) at (60:\len) {1};
        \node[bllbl] (d) at (0:\len) {3};
        \node[hN] (e) at (0,0) {};
        
        \draw (d)--(c);
    \end{tikzpicture}
    }
    \hspace*{\fill}
    \subfloat[\label{pic:bsp_CliqueWidthGem_b}$\text{\protect\subref{pic:bsp_CliqueWidthGem_a}} \oplus \odot_2 \oplus \odot_1$]{
    \begin{tikzpicture}
        [thick,grlbl/.style={lbl,circle,draw=clDark25Green,fill=clLight60Green,inner sep=2pt},
         bllbl/.style={lbl,circle,draw=clDark25Blue,fill=clLight80Blue,inner sep=2pt},
         orlbl/.style={lbl,circle,draw=clDark25Orange,fill=clLight60Orange,inner sep=2pt}]
        \def\len{1.25}
        
        \node[orlbl] (a) at (180:\len) {1};
        \node[hN] (b) at (120:\len) {};
        \node[orlbl] (c) at (60:\len) {1};
        \node[bllbl] (d) at (0:\len) {3};
        \node[grlbl] (e) at (0,0) {2};
        
        \draw (d)--(c);
    \end{tikzpicture}
    }
    \hspace*{\fill}
    \subfloat[\label{pic:bsp_CliqueWidthGem_c}$\eta_{1,2}(\eta_{2,3}(\text{\protect\subref{pic:bsp_CliqueWidthGem_b}}))$]{
    \begin{tikzpicture}
        [thick,grlbl/.style={lbl,circle,draw=clDark25Green,fill=clLight60Green,inner sep=2pt},
         bllbl/.style={lbl,circle,draw=clDark25Blue,fill=clLight80Blue,inner sep=2pt},
         orlbl/.style={lbl,circle,draw=clDark25Orange,fill=clLight60Orange,inner sep=2pt}]
        \def\len{1.25}
        
        \node[orlbl] (a) at (180:\len) {1};
        \node[hN] (b) at (120:\len) {};
        \node[orlbl] (c) at (60:\len) {1};
        \node[bllbl] (d) at (0:\len) {3};
        \node[grlbl] (e) at (0,0) {2};
        
        \draw (a)--(e)--(c)--(d)--(e);
    \end{tikzpicture}
    }
    \par\vspace*{20pt}
    \hspace*{\fill}
    \subfloat[\label{pic:bsp_CliqueWidthGem_d}$\rho_{1 \rightarrow 2}(\text{\protect\subref{pic:bsp_CliqueWidthGem_c}}) \oplus \odot_1$]{
    \begin{tikzpicture}
        [thick,grlbl/.style={lbl,circle,draw=clDark25Green,fill=clLight60Green,inner sep=2pt},
         bllbl/.style={lbl,circle,draw=clDark25Blue,fill=clLight80Blue,inner sep=2pt},
         orlbl/.style={lbl,circle,draw=clDark25Orange,fill=clLight60Orange,inner sep=2pt}]
        \def\len{1.25}
        
        \node[grlbl] (a) at (180:\len) {2};
        \node[orlbl] (b) at (120:\len) {1};
        \node[grlbl] (c) at (60:\len) {2};
        \node[bllbl] (d) at (0:\len) {3};
        \node[grlbl] (e) at (0,0) {2};
        
        \draw (a)--(e)--(c)--(d)--(e);
    \end{tikzpicture}
    }
    \hspace*{\fill}
    \subfloat[\label{pic:bsp_CliqueWidthGem_e}$\eta_{1,2}(\text{\protect\subref{pic:bsp_CliqueWidthGem_d}})$]{
    \begin{tikzpicture}
        [thick,grlbl/.style={lbl,circle,draw=clDark25Green,fill=clLight60Green,inner sep=2pt},
         bllbl/.style={lbl,circle,draw=clDark25Blue,fill=clLight80Blue,inner sep=2pt},
         orlbl/.style={lbl,circle,draw=clDark25Orange,fill=clLight60Orange,inner sep=2pt}]
        \def\len{1.25}
        
        \node[grlbl] (a) at (180:\len) {2};
        \node[orlbl] (b) at (120:\len) {1};
        \node[grlbl] (c) at (60:\len) {2};
        \node[bllbl] (d) at (0:\len) {3};
        \node[grlbl] (e) at (0,0) {2};
        
        \draw (b)--(a)--(e)--(b)--(c)--(e)--(d)--(c);
    \end{tikzpicture}
    }
    \hspace*{\fill}
    \caption{Erstellen eines Gems}
    \label{pic:bsp_CliqueWidthGem}
\end{figure}


\begin{enumerate}
    \item[\subref{pic:bsp_CliqueWidthGem_a}] Begonnen wird mit dem Erstellen von zwei Knoten mit den Labels~$1$ und $3$. Diese werden zu einem Graphen zusammengefügt und mit einer Kante verbunden.
    \[ G:= \eta_{1,3}(\odot_1 \oplus \odot_3) \]
    
    \item[\subref{pic:bsp_CliqueWidthGem_b}] Als nächstes wird der bisherige Graph um zwei weitere Knoten mit den Labels~$1$ und $2$ erweitert.
    \[ G:= G \oplus \odot_2 \oplus \odot_1 \]
    
    \item[\subref{pic:bsp_CliqueWidthGem_c}] Im nächsten Schritt verbindet man den Knoten mit Label~$2$ mit den Knoten, welche die Labels~$1$ und $3$ haben.
    \[ G:= \eta_{1,2}(\eta_{2,3}(G)) \]
    
    \item[\subref{pic:bsp_CliqueWidthGem_d}] Nun erhalten die Knoten mit Label~$1$ das Label~$2$. Außerdem wird ein neuer Knoten mit Label~$1$ eingefügt.
    \[ G:= \rho_{1 \rightarrow 2}(G) \oplus \odot_1 \]
    
    \item[\subref{pic:bsp_CliqueWidthGem_e}] Abschließend werden erneut die Knoten mit Label~$1$ mit den Knoten mit Label~$2$ verbunden.
    \[ G:= \eta_{1,2}(G) \]
    

\end{enumerate}

Der so erzeugte Graph lässt sich auch in einer einzigen Formel ausdrücken:
\[ G:=\eta_{1,2}(\rho_{1 \rightarrow 2}(\eta_{1,2}(\eta_{2,3}(\eta_{1,3}(\odot_1 \oplus \odot_3) \oplus \odot_2 \oplus \odot_1))) \oplus \odot_1)  \]
Eine solche Formel wird auch als Cliquenweite-Ausdruck bezeichnet.

%\subsection{Cliquenweite verschiedener Graphenklassen}
%Das ermitteln der Cliquenweite eines Graphen ist im Allgemeinen ein schwiriges Problem. \todo{Quelle und genaue Komplexität} Für einige Graphenklassen ist es aber möglich eine maximale Cliquenweite anzugeben. Das heißt, die Klasse enthält keinen Graphen, der eine höhere Cliquenweite hat. Tabelle~\ref{tbl:Cliquewidth_classes} gibt eine Übersicht über die maximale Cliquenweite der in diesem Kapitel vorgestellten Graphenklassen.
%
%\begin{table}[htbp]\linespread{1.15}
%    \setlength\tabcolsep{.75em}
%    \arrayrulecolor{clChapTxt}
%    \setlength\arrayrulewidth{1pt}
%    \centering
%    \begin{tabular}{lc}
%        \hline
%            \bfseries\rmfamily\textcolor{clChapTxt}{Klasse} &     \bfseries\rmfamily\textcolor{clChapTxt}{Cliquenweite}\\
%        \hline
%            chordal &     unbegrenzt\\
%            strongly chordal &     unbegrenzt\\
%            Intervall- &     unbegrenzt\\
%            distanzerblich &     3\\
%            dually chordal &     unbegrenzt\\
%            Baum &     3\\
%        \hline
%    \end{tabular}
%    \caption[Cliquenweite verschiedener Graphenklassen]{Cliquenweite verschiedener Graphenklassen. \cite{cliqueWidthPerfectGraphs}}
%    \label{tbl:Cliquewidth_classes}
%\end{table}
%

\chapter{Hypergraphen}

Dieses Kapitel befasst sich mit Hypergraphen. Der Fokus liegt dabei auf den azyklischen Hypergraphen sowie der Charakterisierung ihrer Linegraphen.

\section{Einleitung}
Hypergraphen sind eine Verallgemeinerung von Graphen. Kanten sind nun nicht mehr auf zwei Knoten beschränkt, sondern können beliebig viele (jedoch nicht null) Knoten beinhalten.

\begin{mydef}[Hypergraph\index{Hypergraph}]
    Ein \emph{Hypergraph} $H$ ist ein 2-Tupel $H=(V,\mE)$.  Dabei ist $V$ eine endliche Menge von Knoten und $\mE \subseteq \{e \ |\ e \subseteq V ;\ e \neq \emptyset\}$ die Menge der Hyperkanten.
\end{mydef}

Da Hypergraphen nicht auf genau zwei Knoten pro Kante beschränkt sind, reicht die bisherige Definition für Graphisomorphie (Definition~\ref{def:Graphisomorphie}, S.~\pageref{def:Graphisomorphie}) nicht mehr aus. Die ursprüngliche Definition wird deshalb so erweitert, dass Kanten nicht auf zwei Knoten beschränkt sind.
\begin{mydef}[Isomorphie von Hypergraphen\index{Isomorphie!von Hypergraphen}]
    Zwei Hypergraphen $H_1 = (V_1,\mE_1)$ und $H_2 = (V_2,\mE_2)$ heißen \emph{isomorph} genau dann, wenn eine bijektive Abbildung $\varphi: V_1 \rightarrow V_2$ existiert, so dass für alle Hyperkanten $\{v_1,\ldots,v_i\} \in \mE_1$ gilt: \[ \{v_1,\ldots,v_i\} \in \mE_1 \Leftrightarrow \{\varphi(v_1), \ldots, \varphi(v_i)\} \in \mE_2 \]
    
    Die Isomorphie zweier Hypergraphen wird wie folgt dargestellt: $H_1 \sim H_2$
\end{mydef}



%\subsection{Isomorphie und Teilgraphen-Konzepte}
%Da Hypergraphen nicht auf genau zwei Knoten pro Kante beschränkt sind, reichen die bisherigen Definionen für Teilgraphen\footnote{Definition~\ref{def:Subgraph}, S.~\pageref{def:Subgraph}} und Graphisomorphie\footnote{Definition~\ref{def:Graphisomorphie}, S.~\pageref{def:Graphisomorphie}} nicht mehr aus.
%
%\todo{eventuell ohne Fußnoten}
%
%\subsubsection{Isomorphie}
%Für Isomorphie ist es ausreichend, die ursprüngliche Definition so zu erweitern, dass Kanten nicht auf zwei Knoten beschränkt sind.
%\begin{mydef}[Isomorphie von Hypergraphen\index{Isomorphie!von Hypergraphen}]
%    Zwei Hypergraphen $H_1 = (V_1,\mE_1)$ und $H_2 = (V_2,\mE_2)$ heißen \emph{isomorph} genau dann, wenn eine bijektive Abbildung $\varphi: V_1 \rightarrow V_2$ existiert, so dass für alle Hyperkanten $\{v_1,\ldots,v_i\} \in \mE_1$ gilt: \[ \{v_1,\ldots,v_i\} \in \mE_1 \Leftrightarrow \{\varphi(v_1), \ldots, \varphi(v_i)\} \in \mE_2 \]
%    
%    Die Isomorphie zweier Hypergraphen wird wie folgt dargestellt: $H_1 \sim H_2$
%\end{mydef}
%
%\subsubsection{Teilgraphen}
%
%\todo{Text und verschiedene Konzepte}
%
%\todo{eventuell bessere Überschrift}


\section{Umwandlungsfunktionen}
Es gibt verschiedene Möglichkeiten, Graphen und Hypergraphen in einander umzuformen. In diesem Abschnitt werden drei Varianten vorgestellt. 

\subsection{2-Section Graphen}
Die 2-Section Graphen sind eine Möglichkeit, einen Hypergraphen~$H$ als Graphen darzustellen. Dazu werden die Knoten von $H$ übernommen und miteinander verbunden, wenn sie in der gleichen Hyperkante liegen.

\begin{mydef}[2-Section Graph\index{2-Section Graph}\index{$2Sec(\cdot)$}]
    Es sei $H=(V,\mE)$ ein Hypergraph. Der \emph{2-Section Graph} $2Sec(H)=(V_2,E_2)$ von $H$ ist dann wie folgt definiert:
    \begin{align*}
        V_2 &= V \\
        E_2 &= \{uv\ |\ \exists\,e\in\mE\ u,v\in e\}
    \end{align*}
\end{mydef}

Bei dieser Umwandlung gehen jedoch Informationen über den Hypergraphen verloren. Dadurch lässt sich ein 2-Section Graph nicht eindeutig einem Hypergraphen zuordnen. Abbildung~\ref{pic:bsp_2Sec} stellt zwei Hypergraphen und deren 2-Section Graphen dar.

\begin{figure}[htb]
    \centering
    \hspace*{\fill}
    \subfloat[\label{pic:bsp_2Sec_H1}]{
        \begin{tikzpicture}
            
            \node[nN] (a) at (90:.75) {};
            \node[nN] (b) at (210:.75) {};
            \node[nN] (c) at (330:.75) {};
            
            \begin{pgfonlayer}{background}
                \draw[very thick,clLight60Blue,fill opacity=.5,radius=1.2cm] (0,0) circle;
                \node [fill=clRed,ellipse,fill opacity=.0,rotate fit= 60] (ab)
                      [fit=(a) (b)] {};

                \node [fill=clLight60Blue,ellipse,fill opacity=.0,rotate fit= 0] (bc)
                      [fit=(c) (b)] {};

                \node [fill=clGreen,ellipse,fill opacity=.0,rotate fit= -60] (ac)
                      [fit=(a) (c)] {};

            \end{pgfonlayer}
            
        \end{tikzpicture}
    }
    \hspace*{\fill}
    \subfloat[\label{pic:bsp_2Sec_H2}]{
        \begin{tikzpicture}
            
            \node[nN] (a) at (90:.75) {};
            \node[nN] (b) at (210:.75) {};
            \node[nN] (c) at (330:.75) {};
            
            \begin{pgfonlayer}{background}
                \node [very thick,draw=clRed,ellipse,fill opacity=.5,rotate fit= 60] (ab)
                      [fit=(a) (b)] {};

                \node [very thick,draw=clLight60Blue,ellipse,fill opacity=.5,rotate fit= 0] (bc)
                      [fit=(c) (b)] {};

                \node [very thick,draw=clGreen,ellipse,fill opacity=.5,rotate fit= -60] (ac)
                      [fit=(a) (c)] {};

                \fill[clLight60Blue,fill opacity=.0,radius=1.2cm] (0,0) circle;
            \end{pgfonlayer}
            
        \end{tikzpicture}
    }
    \hspace*{\fill}
    \subfloat[\label{pic:bsp_2Sec_S}]{
        \begin{tikzpicture}[thick]
            
            \node[nN] (a) at (90:.75) {};
            \node[nN] (b) at (210:.75) {};
            \node[nN] (c) at (330:.75) {};
            
            \draw (a)--(b)--(c)--(a);
            
            \begin{pgfonlayer}{background}
                \node [fill=clRed,ellipse,fill opacity=.0,rotate fit= 60] (ab)
                      [fit=(a) (b)] {};

                \node [fill=clLight60Blue,ellipse,fill opacity=.0,rotate fit= 0] (bc)
                      [fit=(c) (b)] {};

                \node [fill=clGreen,ellipse,fill opacity=.0,rotate fit= -60] (ac)
                      [fit=(a) (c)] {};

                \fill[clLight60Blue,fill opacity=.0,radius=1.2cm] (0,0) circle;
            \end{pgfonlayer}

        \end{tikzpicture}
    }
    \hspace*{\fill}

    \caption[Zwei Hypergraphen und ihr 2-Section Graph]
    {Die Hypergraphen \subref{pic:bsp_2Sec_H1} und \subref{pic:bsp_2Sec_H2} sowie ihr 2-Section Graph \subref{pic:bsp_2Sec_S}}
    
    \label{pic:bsp_2Sec}
\end{figure}

\subsection{Linegraphen}

Die Idee der Linegraphen ist es, die Nachbarschaft von Hyperkanten in einem Hypergraphen zu betrachten und als Graphen darzustellen. Dazu werden die Hyperkanten eines gegebenen Hypergraphen~$H$ als Knoten betrachtet. Zwei Knoten sind dann durch eine Kante verbunden, wenn die entsprechenden Hyperkanten einen gemeinsamen Knoten in $H$ haben. Abbildung~\ref{pic:bsp_Linegraph} stellt ein Beispiel für einen Hypergraphen und seinen Linegraphen dar.

\begin{mydef}[Linegraph\index{Linegraph}\index{$L(\cdot)$}]\label{def:Linegraph}
    Gegeben sei ein Hypergraph $H=(V,\mE)$. Der Graph $L(H)=(\mE,E)$ mit $$E=\{ef\ |\ e,f\in\mE ; e\neq f ; e\cap f\neq\emptyset\}$$ ist dann der \emph{Linegraph} von $H$.
\end{mydef}

\begin{figure}[htbp]
    \centering
    \hspace*{\fill}
    \subfloat[]{\label{pic:bsp_Linegraph_HG}
        \begin{tikzpicture}
            
            \coordinate (c) at (0,0);
            \coordinate (lt) at (150:1);
            \coordinate (lb) at (210:1);
            \coordinate (lbb) at ($(210:1)+(-90:.5)$);
            \coordinate (ll) at ($(210:1)+(150:1)$);
            \coordinate (rt) at (30:1);
            \coordinate (rr) at ($(0:1.4)$);
            
            %\draw[gray] (lt)--(c)--(lb)--(ll)--(lt)--(lb)--(lbb) (c)--(30:1)--(rr);
            
            \node[ellipse,thick,draw=clGreen,fit=(rr),inner sep=8pt] (e3) {};
                \node[lbl,black,inner sep=2pt,fill=white,circle] at (e3.west) {3};
                
            \node[ellipse,thick,draw=clBlue,fit=(e3)(rt)(c),inner sep=1pt] (e1) {};
                \node[lbl,black,inner sep=2pt,fill=white,circle] at (e1.north) {1};
                
            \node[ellipse,thick,draw=clOrange,fit=(c)(lb)(lt)(ll),inner sep=0pt] (e4) {};
                \node[lbl,black,inner sep=2pt,fill=white,circle] at (e4.north) {4};
                
            \node[ellipse,thick,draw=clRed,fit=(lb)(lbb),inner sep=7pt] (e5) {};
                \node[lbl,black,inner sep=2pt,fill=white,circle] at (e5.north) {5};
                
            \node[ellipse,thick,draw=clViolet,fit=(e4)(e5),inner sep=-3pt] (e2) {};
                \node[lbl,black,inner sep=2pt,fill=white,circle] at (e2.west) {2};
                
            
        \end{tikzpicture}
    }
    \hspace*{\fill}
%    \subfloat[]{\includegraphics[width=0.33\textwidth]{bilder/bsp_baumstruktur.png}}
%    \hspace*{\fill}
    \subfloat[]{\label{pic:bsp_Linegraph_L}
        \begin{tikzpicture}
            
            \node[nN,fill=clBlue] (1) at ( 0, 0) {}; \node[tlbl] (lbl1) at (1.north) {$1$};
            \node[nN,fill=clViolet] (2) at (150:1.5) {}; \node[tlbl] (lbl2) at (2.north) {$2$};
            \node[nN,fill=clGreen] (3) at (0:1.5) {}; \node[rlbl] (lbl3) at (3.east) {$3$};
            \node[nN,fill=clOrange] (4) at (210:1.5) {}; \node[blbl] (lbl4) at (4.south) {$4$};
            \node[nN,fill=clRed] (5) at ($(210:1.5)+(150:1.5)$) {}; \node[llbl] (lbl5) at (5.west) {$5$};
            
            \draw[thick] (2) -- (4) -- (5) -- (2) -- (1) -- (4)  (1) -- (3);
        \end{tikzpicture}
    }
    \hspace*{\fill}
    \caption[Beispiel für einen Hypergraphen und seinen Linegraphen]{Beispiel für einen Hypergraphen~\subref{pic:bsp_Linegraph_HG} und seinen Linegraphen~\subref{pic:bsp_Linegraph_L}}
    \label{pic:bsp_Linegraph}
\end{figure}


In Abschnitt~\ref{sec:LinegraphOfHypergraph} werden die Linegraphen von verschiedenen Hypergraphen charakterisiert.

\subsection{Cliquen(hyper)graph}
Eine Variante, einen Hypergraphen aus einem Graphen $G$ zu erzeugen, ist das Bilden des Cliquenhypergraphen. Dabei wird für jede maximale Clique in $G$ eine Hyperkante erzeugt.

\begin{mydef}[Cliquenhypergraph\index{Cliquenhypergraph}\index{$\mC(\cdot)$}]
    Es sei $G=(V,E)$ ein Graph. Dessen \emph{Cliquenhypergraph} $\mC(G)=(V,\mE)$ sei dann wie folgt definiert:
    \[ \mE:= \{ c \ |\ \text{$c$ ist maximale Clique in $G$} \} \]
\end{mydef}

Erzeugt man für jede maximale Clique statt einer Hyperkante einen Knoten und verbindet diese, falls die entsprechenden Cliquen einen gemeinsamen Knoten in $G$ besitzen, so wird der erzeugt Graph als Cliquengraph bezeichnet.

\begin{mydef}[Cliquengraph\index{Cliquengraph}\index{$K(\cdot)$}]
    Für einen gegebenen Graphen $G=(V_g,E_g)$ sei sein \emph{Cliquengraph} $K(G)=(V_k,E_k)$ wie folgt definiert:
    \begin{align*}
        V_k & := \{ c \ |\ \text{$c$ ist maximale Clique in $G$} \} \\
        E_k & := \{ cd \ |\ c \cap d \neq \emptyset \}
    \end{align*}
\end{mydef}

Wendet man die Definition von Linegraphen auf die von Cliquenhypergraphen an, stellt man fest, dass der Cliquengraph genau der Linegraph eines Cliquenhypergraphen ist.

\begin{Lemma}\label{lem:Cliquegraph}
    Für jeden Graphen $G$ gilt: $K(G) = L(\mC(G))$\index{Cliquenhypergraph}\index{Linegraph}\index{Cliquengraph}.
\end{Lemma}

Der Cliquengraph bietet auch eine weitere Möglichkeit, um dually chordale Graphen (siehe Abschnitt~\ref{sec:DuallyChordalGraphs}) zu definieren.

\begin{Theorem}\label{theo:CliqueChorDuallyChor}
    \index{dually chordal}\index{Cliquengraph}\index{chordal}
    \cite{duallyChordal}
    Ein Graph $G$ ist dually chordal genau dann, wenn $G$ der Cliquengraph eines chordalen Graphen ist.
\end{Theorem}

\section{Eigenschaften}
Hypergraphen können Eigenschaften besitzen, die üblicherweise bei Graphen nicht betrachtet werden. So stellen Bäume die einzige Graphenklasse dar, welche die beiden nachfolgend vorgestellten Eigenschaften hat.

\subsection{Helly-Eigenschaft}
Die Helly-Eigenschaft trifft eine Aussage über gemeinsame Knoten von Kanten. Hat ein Hypergraph die Helly-Eigenschaft, dann existiert für jede Menge von Hyperkanten, die sich paarweise schneiden, ein gemeinsamer Knoten, der in jeder der Hyperkanten vorhanden ist.

\begin{mydef}[Helly-Eigenschaft\index{Helly-Eigenschaft}]\label{def:Helly}
    Ein Hypergraph $H=(V,\mE)$ hat die Helly-Eigenschaft, wenn für alle $\mE^*\subseteq\mE$ gilt:
    \[ \Big(\forall\,e_1,e_2 \in \mE^*: e_1 \cap e_2 \neq \emptyset\Big) \Rightarrow \bigcap_{e \in \mE^*}e \neq \emptyset \]
\end{mydef}

Für den Linegraphen $\mL$ eines Hypergraphen $H$ bedeutet die Helly-Eigenschaft, dass es für jede maximale Clique in $\mL$ einen gemeinsamen Knoten der entsprechenden Hyperkanten in $H$ gibt. 

\subsection{Conformalität}
Conformalität stellt eine zusätzliche Einschränkung für den 2-Section Graphen eines Hypergraphen dar. Ist ein Hypergraph conformal, so gibt es für jede (maximale) Clique in dessen 2-Section Graph auch eine Hyperkante, die alle Knoten der Clique enthält.

\begin{mydef}[conformal\index{Conformalität}\index{Hypergraph!conformal|see{Conformalität}}]\label{def:conformal}
    Ein Hypergraph $H=(V,\mE)$ ist \emph{conformal}, wenn für jede Clique $K$ mit den Knoten $V_k$ in $2Sec(H)$ gilt:
    \[ \exists\,e\in\mE\text{ mit } V_k\subseteq e\]
\end{mydef}

Ähnlich wie die Helly-Eigenschaft lässt sich Conformalität auch über die Knoten benachbarter Kanten beschreiben. Diese Gesetzmäßigkeit ist als \emph{Gilmore Theorem} bekannt.

\begin{Theorem}[Gilmore Theorem\index{Gilmore Theorem}]\label{theo:GilmoreTheorem}\cite{berge1989hypergraphs}
    \index{Conformalität}
    Ein Hypergraph $H=(V,\mE)$ ist \emph{conformal} genau dann, wenn für alle 3-elementigen Knotenmengen $\{ e_1, e_2, e_3 \} \subseteq \mE$ ein $e \in \mE$ existiert mit $(e_1 \cap e_2) \cup (e_1 \cap e_3) \cup (e_2 \cap e_3) \subseteq e$.
\end{Theorem}

Zwar treffen sowohl Conformalität als auch die Helly-Eigenschaft Aussagen über paarweise benachbarte Hyperkanten, jedoch sind sie nicht äquivalent und bedingen auch einander nicht. Abbildung~\ref{pic:bsp_conformalHelly} stellt zwei Hypergraphen dar, von denen einer conformal ist und einer die Helly-Eigenschaft hat.

\begin{figure}[htb]
    \centering
    \hspace*{\fill}
    \subfloat[\label{pic:bsp_Conf_keineHE}]{
        \begin{tikzpicture}[thick]
            
            \node[nN] (a) at (90:.75) {};
            \node[nN] (b) at (210:.75) {};
            \node[nN] (c) at (330:.75) {};
            
            %\path[red] node [draw,ellipse,fit=(c) (b)] {};
            
            \begin{pgfonlayer}{background}
                \node [very thick,draw=clRed,ellipse,fill opacity=.5,rotate fit= 60] (ab)
                      [fit=(a) (b)] {};

                \node [very thick,draw=clLight60Blue,ellipse,fill opacity=.5,rotate fit= 0] (bc)
                      [fit=(c) (b)] {};

                \node [very thick,draw=clGreen,ellipse,fill opacity=.5,rotate fit= -60] (ac)
                      [fit=(a) (c)] {};

            \end{pgfonlayer}
            \begin{pgfonlayer}{lowerBackground}
                \draw[very thick,clOrange,fill opacity=.5,radius=1.4cm] (0,0) circle;                   
                %\path[] node [fill=white,ellipse,rotate fit=-30,fit=(a) (b)] {};
                %\path[] node [fill=white,ellipse,fit=(c) (b)] {};
                %\path[] node [fill=white,ellipse,rotate fit=30,fit=(a) (c)] {};
            \end{pgfonlayer}

        \end{tikzpicture}
    }
    \hspace*{\fill}
    \subfloat[\label{pic:bsp_nichtConf_HE}]{
        \begin{tikzpicture}[thick]
            
            \node[nN] (a) at (90:.75) {};
            \node[nN] (b) at (210:.75) {};
            \node[nN] (c) at (330:.75) {};
            \node[nN] (m) at (0,0) {};
            
            %\path[red] node [draw,ellipse,fit=(c) (b)] {};
            
            \begin{pgfonlayer}{background}

                %\path[rounded corners=.3cm,fill=clLight60Blue,fill opacity=.5]
                %    ($(b)+3.8637*(195:.3)$) -- ($(c)+3.8637*(-15:.3)$) -- ($(m)+(90:.3)$) --cycle;
                    
                \path[very thick,draw=clRed,fill opacity=.5]
                    ($(m)+(-60:.3)$)  arc [start angle=-60,delta angle=60,radius=.3cm] -- ($(a)+(0:.3)$) 
                    arc [start angle=0,delta angle=150,radius=.4cm] -- ($(b)+(150:.4866)$) 
                    arc [start angle=150,delta angle=150,radius=.4cm] -- cycle;
                    
                \path[very thick,draw=clGreen,fill opacity=.5]
                    ($(m)+(180:.3)$)  arc [start angle=180,delta angle=60,radius=.3cm] -- ($(c)+(240:.3)$) 
                    arc [start angle=240,delta angle=150,radius=.4cm] -- ($(a)+(30:.4866)$) 
                    arc [start angle=30,delta angle=150,radius=.4cm] -- cycle;
                    
                \path[very thick,draw=clLight60Blue,fill opacity=.5]
                    ($(m)+(60:.3)$)  arc [start angle=60,delta angle=60,radius=.3cm] -- ($(b)+(120:.3)$) 
                    arc [start angle=120,delta angle=150,radius=.4cm] -- ($(c)+(-90:.4866)$) 
                    arc [start angle=-90,delta angle=150,radius=.4cm] -- cycle;
                %\node [fill=clLight60Blue,ellipse,fill opacity=.5,rotate fit= 0] (bc) [fit=(c) (b) (m)] {};


            \end{pgfonlayer}
            \begin{pgfonlayer}{lowerBackground}
                \fill[white,fill opacity=0,radius=1.4cm] (0,0) circle;                   
            \end{pgfonlayer}

        \end{tikzpicture}
    }
    \hspace*{\fill}
    
    \caption[Unterschied von Conformalität und Helly-Eigenschaft]
    {Unterschied von Conformalität\index{Conformalität} und Helly-Eigenschaft\index{Helly-Eigenschaft}: Der Hypergraph~\subref{pic:bsp_Conf_keineHE} ist conformal, aber erfüllt nicht die Helly-Eigenschaft. Der Hypergraph~\subref{pic:bsp_nichtConf_HE} hingegen erfüllt die Helly-Eigenschaft, aber ist nicht conformal.}
    
    \label{pic:bsp_conformalHelly}
\end{figure}

%\todo{Eventuell zusammenhang 2Sec - Conformal (nötig für Satz~\ref{theo:chordalWkFrei})\\
%H conformal $\Leftrightarrow$ Jede Clique in 2Sec(H): es gibt Kante in H die ganze Clique enthält}
%\begin{mydef}[conformal\index{conformal}]
%Ein Hypergraph $H=(V,\mE)$ ist \emph{conformal}, wenn für jede Clique $K$ mit den Knoten $V_k$ in $2Sec(H)$ gilt:
%\[ \exists\,e\in\mE\text{ mit } V_k\subseteq e\]
%\end{mydef}

\section{Azyklische Hypergraphen}

Dieser Abschnitt definiert drei Klassen von azyklischen Hypergraphen. Zusätzlich wird die Graham-Reduktion vorgestellt.

\subsection{$\alpha$-azyklische Hypergraphen}
Für $\alpha$-azyklische Hypergraphen, die auch als \emph{dual hypertrees} bezeichnet werden, gibt es verschiedene Definitionen. Eine davon bezieht sich auf Conformalität und den 2-Section Graphen.
\begin{mydef}[$\alpha$-azyklischer Hypergraph\index{$\alpha$-azyklisch}\index{Hypergraph!$\alpha$-azyklisch}\label{def:alphaAzyklisch}]
    Ein Hypergraph $H$ ist \emph{$\alpha$-azyklisch} genau dann, wenn $H$ conformal und $2Sec(H)$ chordal ist.
\end{mydef}

Aus der Definition für $\alpha$-azyklische Hypergraphen und für Conformalität folgt nun unmittelbar, dass der Cliquenhypergraph eines chordalen Graphen $\alpha$-azyklisch ist.

\begin{Lemma}\label{lem:ChliqueChordalAlphaAzyk}\cite[Corollary 1.3.2]{brandstaedt1999graph}
    \index{chordal}\index{Cliquenhypergraph}
    Ein Graph $G$ ist chordal genau dann, wenn sein Cliquenhypergraph $\mC(G)$ $\alpha$-azyklisch ist.
\end{Lemma}

Ein nützlicher Aspekt von $\alpha$-azyklischen Hypergraphen ist die \emph{Graham-Reduktion}. Dabei handelt es sich um zwei einfache Eliminationsregeln für einen gegeben Hypergraphen.

\begin{mydef}[Graham-Reduktion\index{Graham Reduktion}\label{def:GrahamReduktion}]
    Unter der \emph{Graham-Reduktion} versteht man das wiederholte Anwenden der beiden nachfolgenden Regeln auf einen Hypergraphen $H = (V,\mE)$.
    \begin{enumerate}
        \item Ist ein Knoten $v$ in genau einer Hyperkante enthalten, dann kann $v$ entfernt werden.
        \item Ist eine Hyperkante $e$ vollständig in einer anderen Hyperkante $f$ enthalten ($e \subseteq f$), kann $e$ entfernt werden.
    \end{enumerate}
    
    Führt die Reduktion dazu, dass $H$ nur noch eine leere Kante besitzt ($\mE=\{\emptyset\}$), so sagt man, die Reduktion war \emph{erfolgreich}.
\end{mydef}

\begin{Theorem}\label{theo:AlphaAzykGraham} \cite{Beeri1983}
    \index{Graham Reduktion}
    Ein Hypergraph~$H$ ist $\alpha$-azyklisch genau dann, wenn die Graham-Reduktion erfolgreich ist für $H$.
\end{Theorem}

Aufgrund von Satz~\ref{theo:AlphaAzykGraham} ist die Graham-Reduktion nicht nur eine Eliminationsregel für $\alpha$-azyklische Hypergraphen, sondern auch eine Konstruktionsregel.

\subsection{$\beta$-azyklische Hypergraphen}
Erweitert man die Bedingungen für einen $\alpha$-azyklischen Hypergraphen so, dass auch jede Teilmenge der Hyperkanten sie erfüllt, so erhält man die Klasse der $\beta$-azyklischen Hypergraphen.

\begin{mydef}[$\beta$-azyklischer Hypergraph\index{$\beta$-azyklisch}\index{Hypergraph!$\beta$-azyklisch}, \cite{Fagin1983}]
    Ein Hypergraph $H=(V,\mE)$ ist \emph{$\beta$-azyklisch} genau dann, wenn für alle $\mE' \subseteq \mE$ gilt: $\mE'$ ist $\alpha$-azyklisch.
\end{mydef}

$\beta$-azyklische Hypergraphen werden auch als \emph{totally balanced} bezeichnet. Sie erfüllen die Helly-Eigenschaft \cite{berge1989hypergraphs}.
      
\subsection{$\gamma$-azyklische Hypergraphen}
Eine Teilmenge der $\beta$-azyklischen Hypergraphen sind die $\gamma$-azyklischen. Sie definieren sich über einen nicht erlaubten Teilgraphen, einen sogenannten $\gamma$-cycle.

\begin{mydef}[$\gamma$-cycle\index{$\gamma$-cycle}, \cite{Fagin1983}]\label{def:GammyCycle}
    Ein \emph{$\gamma$-cycle} in einem Hypergraphen $H$ ist eine Folge $(v_1,e_1,\ldots,v_k,e_k)$ mit $k \geq 3$, welche die folgenden Bedingungen erfüllt:
    \begin{enumerate}
        \item $v_1,\ldots, v_k$ sind paarweise verschiedene Knoten in $H$.
        \item $e_1,\ldots, e_k$ sind paarweise verschiedene Hyperkanten in $H$ und $e_{k+1}=e_1$.
        \item $\forall\, i\ (1 \leq i \leq k): v_i \in e_i \cap e_{i+1}$
        \item\label{BedingungDefGammaCycle} $\forall\, i\ (1 \leq i < k): \forall\, j\ (j \neq i,i+1): v_i \notin e_j$
    \end{enumerate}
\end{mydef}

Bedingung~\ref{BedingungDefGammaCycle} in Definition~\ref{def:GammyCycle} bedeutet, dass alle Knoten~$v_i$ nur in den Hyperkanten~$e_i$ und $e_{i+1}$ sind, jedoch nicht der Knoten~$v_k$. Dieser darf auch in den anderen Hyperkanten enthalten sein. Abbildung~\ref{pic:bsp_GammaCycle} gibt ein Beispiel für einen $\gamma$-cycle.

\begin{figure}[htbp]
    \centering
    \begin{tikzpicture}
        
        \node[nN] (v1) at (150:1.5) {}; \node[blbl] (lbl1) at (v1.south) {$v_1$};
        \node[nN] (v2) at (30:1.5) {};  \node[blbl] (lbl2) at (v2.south) {$v_2$};
        \node[nN] (v3) at (0,0) {};     \node[blbl] (lbl3) at (v3.south) {$v_3$};
        
        %\draw[very thick,clLight60Blue,opacity=.2,radius=2.5cm] (90:1.5) circle;
        %\draw[very thick,clLight60Blue,opacity=.2,radius=2cm] (-30:1.5) circle;
        %\draw[very thick,clLight60Blue,opacity=.2,radius=2cm] (210:1.5) circle;
        
        %\draw [black] (210:1.5) -- ($(210:1.5)+(125:2)$) -- ($(210:1.5)+(-5:2)$) -- cycle;
        %\draw [clRed] (90:1.5) -- ($(90:1.5)+(193:2.5)$) -- ($(90:1.5)+(347:2.5)$) -- cycle;
        %\draw [clGreen] (-30:1.5) -- ($(-30:1.5)+(55:2)$) -- ($(-30:1.5)+(185:2)$) -- cycle;
        
%        \draw[thick,clBlue]
%            ($(90:1.5)+(193:2.5)+(13:.5)+(90:.5) $)
%            arc[start angle=90,end angle=193,radius=.5]
%            arc[start angle=193,end angle=347,radius=2.5]
%            arc[start angle=-13,end angle=90,radius=.5] -- cycle;

%        \draw[thick,clBlue]
%            ($(90:1.75)+(200:2.5)+(20:.5)+(90:.5) $)
%            arc[start angle=90,end angle=200,radius=.5]
%            arc[start angle=200,end angle=340,radius=2.5]
%            arc[start angle=-20,end angle=90,radius=.5] -- cycle;

%        \draw[thick,clBlue]
%            ($(90:1.5)+(193:2.5)$) arc[start angle=193,end angle=347,radius=2.5];
%
%        \draw[thick,clBlue]
%            ($(90:1.5)+(193:2.5)$) .. 
%            controls ($(90:1.5)+(193:2.5)+(103:1)$) and ($(90:1.5)+(347:2.5)+(77:1)$) .. 
%            node[tlbl,black] {$e_2$}
%            ($(90:1.5)+(347:2.5)$);
            
%        \draw[thick,clGreen]
%            ($(-30:1.5)+(55:2)+(-125:.5)+(-60:.5) $)
%            arc[start angle=-60,end angle=55,radius=.5]
%            arc[start angle=55,end angle=185,radius=2]
%            arc[start angle=185,end angle=300,radius=.5] -- cycle;

%        \draw[thick, clGreen]
%            ($(-30:1.5)+(55:2)$) arc[start angle=55,end angle=185,radius=2];
%             
%        \draw[thick, clGreen]
%            ($(-30:1.5)+(55:2)$) .. 
%            controls ($(-30:1.5)+(55:2)+(-35:1.5)$) and ($(-30:1.5)+(185:2)+(275:1.5)$) .. 
%            node[lbl,black,anchor=north west] {$e_3$}
%            ($(-30:1.5)+(185:2)$);
            
%        \draw[thick,clOrange]
%            ($(210:1.5)+(-5:2)+(175:.5)+(-120:.5) $)
%            arc[start angle=-120,end angle=-5,radius=.5]
%            arc[start angle=-5,end angle=125,radius=2]
%            arc[start angle=125,end angle=240,radius=.5] -- cycle;

%        \draw[thick,clOrange]
%            ($(210:1.5)+(-5:2)$) arc[start angle=-5,end angle=125,radius=2];
%            
%        \draw[thick,clOrange]
%            ($(210:1.5)+(-5:2)$) .. 
%            controls ($(210:1.5)+(-5:2)+(-95:1.5)$) and ($(210:1.5)+(125:2)+(215:1.5)$) .. 
%            node[lbl,black,anchor=north east] {$e_1$}
%            ($(210:1.5)+(125:2)$);
            
%        \fill [gray]
%            ($(90:1.5)+(193:2.5)+(103:1)$) circle (2pt)
%            ($(90:1.5)+(347:2.5)+(77:1)$) circle (2pt)
%            ($(-30:1.5)+(55:2)+(-35:1.5)$) circle (2pt)
%            ($(-30:1.5)+(185:2)+(275:1.5)$) circle (2pt)
%            ($(210:1.5)+(-5:2)+(-95:1.5)$) circle (2pt)
%            ($(210:1.5)+(125:2)+(215:1.5)$) circle (2pt);
         
         
         \node[ellipse,thick,draw=clOrange,fit=(v3)(v1),rotate fit=-30,inner sep=5pt] (e1) {};
         \node[lbl,black,inner sep=.75pt,fill=white,circle] at (e1.south) {$e_1$};

         \node[ellipse,thick,draw=clGreen,fit=(v3)(v2),rotate fit=30,inner sep=5pt] (e3) {};
         \node[lbl,black,inner sep=.75pt,fill=white,circle] at (e3.south) {$e_3$};
         
         \node[ellipse,thick,draw=clBlue,fit=(e3)(e1),inner sep=5pt] (e2) {};
         \node[lbl,black,inner sep=.75pt,fill=white,circle] at (e2.south) {$e_2$};
         %\draw (current bounding box.south west) -- (current bounding box.south east) -- (current bounding box.north east) -- (current bounding box.north west) -- cycle;
    \end{tikzpicture}
    \caption{Beispiel für einen $\gamma$-cycle}
    \label{pic:bsp_GammaCycle}
\end{figure}

\begin{mydef}[$\gamma$-azyklischer Hypergraph\index{$\gamma$-azyklisch}\index{Hypergraph!$\gamma$-azyklisch}, \cite{Fagin1983}]
    Ein Hypergraph ist \emph{$\gamma$-azyklisch} genau dann, wenn er keinen $\gamma$-cycle enthält.
\end{mydef}

Für azyklische Hypergraphen gilt die folgende Hierarchie:

\begin{Theorem}\label{theo:AzykHiera}\cite{Fagin1983}
    \index{Hypergraph!$\gamma$-azyklisch}\index{Hypergraph!$\beta$-azyklisch}\index{Hypergraph!$\alpha$-azyklisch}
    $\gamma$-azyklisch $\subset$ $\beta$-azyklisch $\subset$ $\alpha$-azyklisch
\end{Theorem}


\section{Linegraphen von azyklischen Hypergraphen}\label{sec:LinegraphOfHypergraph}

Dieser Abschnitt beschäftigt sich mit der Charakterisierung der Linegraphen von $\alpha$-azyklischen Hypergraphen und deren Unterklassen ($\beta$- und $\gamma$-azyklisch).


Zuerst wird gezeigt, dass die Klasse der Linegraphen von $\alpha$-azyklischen Hypergraphen genau die Klasse der dually chordalen Graphen ist. Dazu seien die folgenden Aussagen wiederholt:
\begin{itemize}

    \item $K(G) = L(\mC(G))$ (Lemma~\ref{lem:Cliquegraph})
    
    \item $G$ ist chordal $\Leftrightarrow$ $\mC(G)$ ist $\alpha$-azyklisch (Lemma~\ref{lem:ChliqueChordalAlphaAzyk})
    
    \item $G$ ist dually chordal $\Leftrightarrow$ $G$ ist der Cliquengraph eines chordalen Graphen. (Satz~\ref{theo:CliqueChorDuallyChor})
    
    \item $H$ ist $\alpha$-azyklisch $\Leftrightarrow$ Die Graham-Reduktion ist erfolgreich für $H$. (Satz~\ref{theo:AlphaAzykGraham})
    
\end{itemize}

\begin{Lemma}\label{lem:AlphaAzyklDuallyChordal}
    \index{Hypergraph!$\gamma$-azyklisch}\index{chordal!dually}\index{dually chordal}
    $H$ ist $\alpha$-azyklisch $\Rightarrow$ $L(H)$ ist dually chordal.
\end{Lemma}

\begin{Proof}
    $H=(V,\mE)$ ist $\alpha$-azyklisch.    Man füge nun in jede Hyperkante $e\in\mE$ einen Knoten $v_e$ so ein, dass $v_e$ nur in $e$ enthalten ist. Der so entstehende Hypergraph $H'$ ist weiterhin $\alpha$-azyklisch (Graham-Reduktion).
    
    Es gilt, dass $L(H)\sim L(H')$, da das Einfügen der Knoten nichts an der Nachbarschaft der Hyperkanten geändert hat.
    
    Aufgrund der eingefügten Knoten $v_e$ gibt es für jede Hyperkante $e$ eine maximale Clique in $2Sec(H')$. Angenommen es gäbe eine weitere maximale Clique $K$, dann wären ihre Knoten aufgrund der Confomalität von $H'$ auch in einer Hyperkante $e$ vorhanden. $K$ ist dann jedoch entweder nicht maximal, oder keine weitere maximale Clique. Somit gilt: $\mC(2Sec(H'))\sim H'$.
    
    Es sei nun $G := 2Sec(H')$. $G$ ist chordal und $\mC(G) \sim H'$. Daraus folgt, dass $L(H')\sim L(\mC(G))=K(G)$. Da $G$ chordal ist, ist somit $K(G)$ dually chordal. Also gilt $L(H)\sim L(H')$ ist dually chordal.
    \qed
\end{Proof}

\begin{Lemma}\label{lem:DuallyChordalAlphaAzykl}
    Für alle dually chordalen Graphen $G$ existiert ein $\alpha$-azyklischer Hypergraph $H$ mit $L(H)\sim G$.
\end{Lemma}

\begin{Proof}
    $G$ ist dually chordal. Das heißt, es existiert ein chordaler Graph $G'$, so dass $K(G')\sim G$. Es gilt $K(G') = L(\mC(G'))$. Es sei nun $H:=\mC(G')$. $H$ ist $\alpha$-azyklisch. Da $L(H) = L(\mC(G')) = K(G')$ ist und $K(G') \sim G$, gilt auch $L(H) \sim G$.
    \qed
\end{Proof}

Aus den Lemmata~\ref{lem:AlphaAzyklDuallyChordal} und \ref{lem:DuallyChordalAlphaAzykl} folgt nun unmittelbar Satz~\ref{theo:AlphaAzyklDuallyChordal}:

\begin{Theorem}\label{theo:AlphaAzyklDuallyChordal}
    Ein Graph ist dually chordal genau dann, wenn er der Linegraph eines $\alpha$-azyklischen Hypergraphen ist.
\end{Theorem}

In Verbindung mit Satz~\ref{theo:hereDuCh_strCh} (S.~\pageref{theo:hereDuCh_strCh}) ergibt sich nun für $\beta$-azyklische Hypergraphen, dass diese strongly chordal sind.

\begin{Theorem}\label{theo:BetaLineStronglyChordal}
    Die Linegraphen von $\beta$-azyklischen Hypergraphen sind strongly chordal.
\end{Theorem}

\begin{Proof}
    Es seien $H=(V,\mE)$ ein $\beta$-azyklischer Hypergraph und $\mL=L(H)=(\mE,E)$ sein Linegraph.
    
    Per Definition gilt, dass alle Teilmengen $\mE'$ der Hyperkanten von $H$ ($\mE' \subseteq \mE$) einen $\alpha$-azyklischen Hypergraphen bilden. Übertragen auf den Linegraphen bedeutet dies, dass jeder induzierte Teilgraph $\mL[\mE']$ dually chordal ist.
    
    Es gilt, dass ein Graph strongly chordal ist, wenn jeder induzierte Teilgraph von $G$ dually chordal ist (Satz~\ref{theo:hereDuCh_strCh}). Somit ist $\mL$ ebenfalls strongly chordal.
    \qed
\end{Proof}

Als nächstes wird gezeigt, dass die Linegraphen von $\gamma$-azyklischen Hypergraphen distanzerblich chordal sind. Dazu werden folgende Aussagen verwendet:

\begin{itemize}
    \item $H$ ist $\gamma$-azyklisch $\Rightarrow$ $H$ ist $\beta$-azyklisch. (Satz~\ref{theo:AzykHiera})
    \item $\beta$-azyklische Hypergraphen erfüllen die Helly-Eigenschaft. \cite{berge1989hypergraphs}
    \item $G$ ist chordal und Gem-frei $\Rightarrow$ $G$ ist distanzerblich. (Lemma~\ref{lem:GemFreeChordalDistanzerblich}, S.~\pageref{lem:GemFreeChordalDistanzerblich})
\end{itemize}

\begin{Theorem}\label{theo:GammaLineGrpah}
    Die Linegraphen von $\gamma$-azyklischen Hypergraphen sind distanzerblich chordal.
\end{Theorem}

\begin{Proof}
    Es seien $H$ ein $\gamma$-azyklischer Hypergraph und $\mL=L(H)$ sein Linegraph.
    
    Da jeder $\gamma$-azyklische Hypergraph auch $\beta$-azyklisch ist, ist $\mL$ strongly chordal. $\mL$ ist somit distanzerblich, wenn $\mL$ Gem-frei ist.
    
    Angenommen, $\mL$ enthalte einen Gem $G$ mit den Knoten $e_1$, $e_2$ und $e_3$. Außerdem seien $K_1$, $K_2$ und $K_3$ die maximalen Cliquen in $G$. Abbildung~\ref{pic:GemGinL} stellt dies dar.

    \begin{figure}[htbp]
        \centering
        \begin{tikzpicture}
        
            \node[nN,fill=clBlue] (e2) at (0,0) {}; \node[blbl] at (e2.south) {$e_2$};
            \node[nN] (e0) at (180:1.5) {};
            \node[nN,fill=clOrange] (e1) at (120:1.5) {}; \node[tlbl] at (e1.north) {$e_1$};
            \node[nN,fill=clGreen] (e3) at (60:1.5) {}; \node[tlbl] at (e3.north) {$e_3$};
            \node[nN] (e4) at (0:1.5) {};
            
            \draw[thick] (e1) -- (e2) -- (e3) -- (e4) -- (e2) -- (e0) -- (e1) -- (e3);
            
            \node [lbl] at (150:0.866)              {$K_1$};
            \node [lbl] at (30:0.866)              {$K_2$};
            \node [lbl] at (90:0.866)              {$K_3$};

        \end{tikzpicture}
        \caption{Der Gem $G$}
        \label{pic:GemGinL}
    \end{figure}

    
    Aufgrund der Helly-Eigenschaft gibt es für jede Clique~$K_i$ in $G$ einen Knoten~$k_i$ in $H$, der in den Hyperkanten enthalten ist, welche die entsprechende Clique in $\mL$ bilden. Somit gilt: $k_1 \in e_1 \cap e_2$, $k_2 \in e_2 \cap e_3$ und $k_3 \in e_1 \cap e_2 \cap e_3$. Abbildung~\ref{pic:GemGinH} stellt dies dar.
    
    \begin{figure}[htbp]
        \centering
        \begin{tikzpicture}
        
        \node[nN] (v1) at (-150:1.5) {}; \node[blbl] (lbl1) at (v1.south) {$k_1$};
        \node[nN] (v2) at (-30:1.5) {};  \node[blbl] (lbl2) at (v2.south) {$k_2$};
        \node[nN] (v3) at (0,0) {};     \node[blbl] (lbl3) at (v3.south) {$k_3$};         
         
         \node[ellipse,thick,draw=clOrange,fit=(v3)(v1),rotate fit=30,inner sep=5pt] (e1) {};
         \node[lbl,black,inner sep=.75pt,fill=white,circle] at (e1.north) {$e_1$};

         \node[ellipse,thick,draw=clGreen,fit=(v3)(v2),rotate fit=-30,inner sep=5pt] (e3) {};
         \node[lbl,black,inner sep=.75pt,fill=white,circle] at (e3.north) {$e_3$};
         
         \node[ellipse,thick,draw=clBlue,fit=(e3)(e1),inner sep=3pt] (e2) {};
         \node[lbl,black,inner sep=.75pt,fill=white,circle] at (e2.south) {$e_2$};

        \end{tikzpicture}
        \caption[Der Gem $G$ in $H$]{Der Gem $G$ in $H$ -- Es wurden nur zu drei der Knoten aus $G$ die entsprechenden Hyperkanten dargestellt.}
        \label{pic:GemGinH}
    \end{figure}

    $(k_1,e_1,k_2,e_2,k_3,e_3)$ bildet nun einen $\gamma$-cycle. Somit kann $\mL$ keinen Gem enthalten.
    \qed    
    
\end{Proof}

\subsection{Charakterisierung mittels Graham-Reduktion}
Eine weitere Möglichkeit, die Linegraphen von $\alpha$-azyklischen Hypergraphen zu charakterisieren, ergibt sich aus der Graham-Reduktion. Als Konstruktionsregel angewendet besteht die Graham-Reduktion aus zwei Regeln:
\begin{enumerate}
	\item \label{case:GrahamConstrVertex} Einfügen eines Knotens in genau eine Hyperkante
	\item \label{case:GrahamConstrEdge} Einfügen einer Hyperkante in eine bereits bestehende
\end{enumerate}

Regel~\ref{case:GrahamConstrVertex} ist für den Linegraphen nicht relevant. Der hinzugefügte Knoten ist nur in genau einer Hyperkante enthalten. Somit haben sich die Nachbarschaften der Kanten durch das Einfügen nicht verändert.

Aus Regel~\ref{case:GrahamConstrEdge} hingegen ergibt sich für die Hyperkanten eine (gerichtete) Baumstruktur, welche die Reihenfolge darstellt, mit der die Hyperkanten eingefügt wurden. Die Wurzel ist dabei die erste Hyperkante und Blätter die zuletzt eingeführten Hyperkanten. Abbildung~\ref{pic:bsp_BaumstrukturGraham} stellt dies an einem Beispiel dar.

\begin{figure}[htbp]
    \centering
    \hspace*{\fill}
    \subfloat[]{\label{pic:bsp_BaumstrukturGraham_HG}
        \begin{tikzpicture}
            
            \coordinate (c) at (0,0);
            \coordinate (lt) at (150:1);
            \coordinate (lb) at (210:1);
            \coordinate (lbb) at ($(210:1)+(-90:.5)$);
            \coordinate (ll) at ($(210:1)+(150:1)$);
            \coordinate (rt) at (30:1);
            \coordinate (rr) at ($(0:1.4)$);
            
            %\draw[gray] (lt)--(c)--(lb)--(ll)--(lt)--(lb)--(lbb) (c)--(30:1)--(rr);
            
            \node[ellipse,thick,draw=clGreen,fit=(rr),inner sep=8pt] (e3) {};
                \node[lbl,black,inner sep=2pt,fill=white,circle] at (e3.west) {3};
                
            \node[ellipse,thick,draw=clBlue,fit=(e3)(rt)(c),inner sep=1pt] (e1) {};
                \node[lbl,black,inner sep=2pt,fill=white,circle] at (e1.north) {1};
                
            \node[ellipse,thick,draw=clOrange,fit=(c)(lb)(lt)(ll),inner sep=0pt] (e4) {};
                \node[lbl,black,inner sep=2pt,fill=white,circle] at (e4.north) {4};
                
            \node[ellipse,thick,draw=clRed,fit=(lb)(lbb),inner sep=7pt] (e5) {};
                \node[lbl,black,inner sep=2pt,fill=white,circle] at (e5.north) {5};
                
            \node[ellipse,thick,draw=clViolet,fit=(e4)(e5),inner sep=-3pt] (e2) {};
                \node[lbl,black,inner sep=2pt,fill=white,circle] at (e2.west) {2};
                
            
        \end{tikzpicture}
    }
    \hspace*{\fill}
    \subfloat[]{\label{pic:bsp_BaumstrukturGraham_T}
        \begin{tikzpicture}
        [lbl/.style={font=\small},
         llbl/.style={left,lbl},rlbl/.style={lbl,right},tlbl/.style={lbl,above}]
            
            \node[nN] (1) at ( 0, 0) {}; \node[tlbl] (lbl1) at (1.north) {$1$};
            \node[nN] (2) at (-1,-1) {}; \node[llbl] (lbl2) at (2.west) {$2$};
            \node[nN] (3) at ( 1,-1) {}; \node[rlbl] (lbl3) at (3.east) {$3$};
            \node[nN] (4) at (-2,-2) {}; \node[llbl] (lbl4) at (4.west) {$4$};
            \node[nN] (5) at ( 0,-2) {}; \node[rlbl] (lbl5) at (5.east) {$5$};
  
            \begin{pgfonlayer}{background}
                \foreach \c/\p in {4/2,5/2,2/1,3/1}
                {
                    \draw[->,very thick,clBlue,decoration={snake,amplitude=1},decorate] (\c.center) -- (\p);
                }
            \end{pgfonlayer}
        \end{tikzpicture}
    }
    \hspace*{\fill}
    \caption[Beispiel für die Baumstruktur der Graham-Reduktion]
    {Beispiel für die Baumstruktur der Graham-Reduktion: Der Hypergraph~\subref{pic:bsp_BaumstrukturGraham_HG} kann erzeugt werden, indem die Hyperkanten in der Reihenfolge ihrer Nummerierung ineinander eingefügt werden ($2$ in $1$, $3$ in $1$, $4$ in $2$, $5$ in $2$). Der Baum~\subref{pic:bsp_BaumstrukturGraham_T} gibt diese Ordnung wieder.}
    \label{pic:bsp_BaumstrukturGraham}
\end{figure}

Ein solcher Baum~$T$ ist nun ein erster Ansatz für den gesuchten Linegraphen. Zwar ist $T$ ein Spannbaum des gesuchten Linegraphen, allerdings werden nicht alle möglichen Nachbarschaften der Hyperkanten wiedergegeben.

Zwei Knoten~$a$ und $b$ sind benachbart in einem Linegraphen, wenn ihre entsprechenden Hyperkanten einen gemeinsamen Knoten~$v$ besitzen. Die Graham-Reduktion erlaubt das Einfügen von Knoten jedoch nur in genau eine Hyperkante. Alle anderen Knoten werden geerbt. Eine Hyperkante kann dabei Knoten nur von der Hyperkante erben, in die sie eingefügt wurde. Daraus ergeben sich nun die zwei folgenden Möglichkeiten, wenn $v$ sowohl in $a$ als auch in $b$ ist:

\begin{enumerate}
    \item \label{case:a_in_b} $a$ wurde in $b$ eingefügt oder umgekehrt.
    \item \label{case:ab_in_e} Es gibt einen gemeinsamen Elternknoten $e$ in $T$, wobei $v$ in die entsprechende Hyperkante eingefügt wurde.
\end{enumerate}

Für den Fall~\ref{case:ab_in_e} bedeutet dies, dass auch alle Hyperkanten, deren Knoten auf dem Pfad (in $T$) von $e$ zu $a$ und von $e$ zu $b$ liegen, den Knoten $v$ besitzen. Andernfalls ließe sich $v$ nicht von $e$ auf $a$ und $b$ vererben. Somit sind $a$ und $b$ auch mit allen Knoten auf diesem Pfad benachbart. Abbildung~\ref{pic:bsp_NachbarschaftPfad} stellt dies dar.

\begin{figure}[htbp]
    \centering
    \begin{tikzpicture}
       
       \foreach \name/\ang in {a/210,a_t/170,e_l/130,e/90,e_r/50,b_t/10,b/-30}
       {
           \node[nN] (\name) at (\ang:2cm) {};
       }
    
       \node[llbl] (lbla) at (a.west) {$a$};
       \node[rlbl] (lblb) at (b.east) {$b$};
       \node[tlbl] (lble) at (e.north) {$e$};
       
       \begin{pgfonlayer}{background}
           \foreach \name in {e_l,e,e_r,b_t,b}
           {
               \draw (a.center) -- (\name.center);
           }

           \foreach \name in {a,a_t,e_l,e,e_r}
           {
               \draw (b.center) -- (\name.center);
           }

           \foreach \f/\t in {a/a_t,b/b_t,e_l/e,e_r/e}
           {
               \draw[->,very thick,clBlue,decoration={snake,amplitude=1},decorate]
                   (\f.center) -- (\t);
           }
            
           \foreach \f/\t in {a_t/e_l,b_t/e_r}
           {
               \draw[->,dotted,very thick,clBlue,decoration={snake,amplitude=1},decorate]
                   (\f.center) -- (\t);
           }
           
           \draw[->,thick,clDark25Green] (100:2.35cm) arc[start angle=100,end angle=195,radius=2.35cm];
           \draw[->,thick,clDark25Green] (80:2.35cm) arc[start angle=80,delta angle=-95,radius=2.35cm];
           \node[lbl,above left] (lblvl) at (147.5:2.35cm) {$v$};
           \node[lbl,above right] (lblvr) at (32.5:2.35cm) {$v$};
            
       \end{pgfonlayer}

    
    \end{tikzpicture}
    \caption[Die Nachbarschaft zweier Hyperkanten entlang des Spannbaums]
    {Die Nachbarschaft der Hyperkanten $a$ und $b$ entlang des Spannbaums (blau gewellt): Der Knoten $v$ wird entlang des Spannbaums von $e$ auf $a$ und $b$ vererbt (grün). Somit sind $a$ und $b$ auch mit allen Hyperkanten entlang dieser Pfade benachbart.}
    \label{pic:bsp_NachbarschaftPfad}
\end{figure}

Entsprechend der obigen Argumentation ergibt sich nun Definition~\ref{def:aLGraph} für die Linegraphen von $\alpha$-azyklischen Hypergraphen.

\begin{mydef}[Linegraph eines $\alpha$-azyklischen Hypergraphen]\label{def:aLGraph}    
    Es sei $P_T(u,v)$ die Menge der Knoten auf dem Pfad von $u$ nach $v$ in $T$ ($u,v \notin P_T(u,v)$).
    
    Ein Graph $G=(V,E)$ ist der Linegraph eines $\alpha$-azyklischen Hypergraphen genau dann, wenn $G$ einen Spannbaum $T$ besitzt, so dass für alle Kanten $uv \in E$ gilt:
    \[ \forall\, w \in P_T(u,v):uw, vw \in E \]    
\end{mydef}

Es ist nun zu beweisen, dass die Graphen, welche die Definition~\ref{def:aLGraph} erfüllen, genau die Linegraphen der $\alpha$-azyklischen Hypergraphen sind. Jedoch ist bereits bekannt, dass es sich dabei um die dually chordalen Graphen handelt (Satz~\ref{theo:AlphaAzyklDuallyChordal}). Für dually chordale Graphen ist außerdem eine Definition bekannt, die auf einem Spannbaum beruht (Satz~\ref{theo:DuallyChordalSpanningTree}, S.~\pageref{theo:DuallyChordalSpanningTree}). Deswegen wird an dieser Stelle lediglich gezeigt, dass beide Definitionen äquivalent sind.

\begin{Theorem}\label{theo:SpanningTree}
    Es seien $G=(V,E)$ ein Graph sowie $T$ ein Spannbaum von $G$. Außerdem sei $P_{uv}$ die Menge der Knoten auf dem Pfad von $u$ nach $v$ in $T$ ($u,v \notin P_{uv}$).
    
    Die folgenden Aussagen sind äquivalent:
    \begin{enumerate}
        \item \label{case:CliqueInduceT} Jede maximale Clique in $G$ induziert einen Teilbaum von $T$.
        \item \label{case:PuvT} Für alle Kanten $uv \in E$ gilt: $\forall\, w \in P_{uv}:uw, vw \in E$
    \end{enumerate}
\end{Theorem}

\begin{Proof}
    Es sei $\mK \subseteq V$ eine maximale Clique in $G$ mit den Knoten~$u$ und $v$.
    
    \textbf{\boldmath\ref{case:CliqueInduceT} $\Rightarrow$ \ref{case:PuvT}:} Die Clique~$\mK$ induziert einen Teilbaum von $T$. Somit gilt, dass $P_{uv} \subseteq \mK$. Folglich sind auch alle Knoten $w \in P_{uv}$ mit $u$ und $v$ verbunden. Andernfalls wäre $\mK$ keine Clique.
    
    \textbf{\boldmath\ref{case:CliqueInduceT} $\Leftarrow$ \ref{case:PuvT}:} Angenommen, $\mK$ induziert keinen Teilbaum von $T$. Dann existiert ein Knoten~$w$, der in $P_{uv}$ liegt, jedoch nicht in $\mK$. Da $\mK$ maximal ist, gibt es einen Knoten $k \in \mK$, der nicht mit $w$ verbunden ist.
    
    Die Knoten $u$ und $v$ sind mit $k$ verbunden (alle drei sind in $\mK$). Für die entsprechenden Pfade $P_{uk}$ und $P_{vk}$ gilt nun, dass $w$ nicht auf diesen Pfaden liegt ($w \notin P_{uk} \cup P_{vk}$). Andernfalls wäre $k$ mit $w$ verbunden. Abbildung~\ref{pic:ProofSpanningTree} stellt dies dar.
    
    \begin{figure}[htbp]
        \centering
        \begin{tikzpicture}
            \def\len{1.5}
            \node[nN] (k) at ($0.5*(-90:\len)$) {}; \node[tlbl] (lbl_k) at (k.north) {$k$};
            \node[nN] (v) at (0:\len) {}; \node[rlbl] (lbl_v) at (v.east) {$v$};
            \node[hN] (vl) at (0:0.7*\len) {};
            \node[hN] (ur) at (180:0.7*\len) {};
            \node[nN] (u) at (180:\len) {}; \node[llbl] (lbl_u) at (u.west) {$u$};
            
            \node[nN] (w) at ($0.75*(90:\len)$) {};  \node[tlbl] (lbl_w) at (w.north) {$w$};
            
            
            %\node[lbl,inner sep=2pt,fill=white,circle] (Puk) at (210:\len) {$P_{uk}$};
            %\node[lbl,inner sep=2pt,fill=white,circle] (Pvk) at (-30:\len) {$P_{vk}$};
            
            %\draw (u)--(v)--(k)--(u);

            \begin{pgfonlayer}{background}
            \node[ellipse,very thick,draw=clGreen,fit=(lbl_u)(lbl_v)(k),inner sep=0pt] (clique) {};
                \draw[Tedge] (u.center)--(ur.center);
                \draw[Tedge] (ur.center)--(w.center);
                \draw[Tedge] (w.center)--(vl.center);

                \draw[Tedge] (k.center)--(ur.center);

                \draw[Tedge] (vl.center)--(k.center);
                \draw[Tedge] (v.center)--(vl.center);
            \end{pgfonlayer}
            \node[lbl,inner sep=2pt,fill=white,circle] at (clique.east) {$\mK$};
        \end{tikzpicture}
        \caption[Skizze für den Beweis von Satz~\ref{theo:SpanningTree}]{Skizze für den Beweis von Satz~\ref{theo:SpanningTree}. Die Knoten~$k$, $u$ und $v$ liegen in der Clique~$\mK$ (grün). Die blau gewellten Kanten sind Pfade im Spannbaum~$T$.}
        \label{pic:ProofSpanningTree}
    \end{figure}

    Es gibt nun in $T$ zwei mögliche Pfade von $u$ nach $v$: Zum einen $P_{uv}$ über $w$ und zum anderen $P_{uk}$ und $P_{vk}$. Somit ist $T$ kein Baum. Dies steht im Widerspruch zur Voraussetzung.
    \qed
\end{Proof}

\subsection{Zusammenfassung der Definitionen}

Es folgt nun eine Zusammenfassung der Charakterisierungen für die Linegraphen von $\alpha$-azyklischen Hypergraphen.

\begin{Theorem}\label{theo:duallyChordalChar}
    Es seien $G=(V,E)$ ein Graph und $P_T(u,v)$ die Menge der Knoten auf dem Pfad von $u$ nach $v$ in $T$ ($u,v \notin P_T(u,v)$).

    Die folgenden Aussagen sind äquivalent:
    \begin{enumerate}
        \item \label{case:G_dually} $G$ ist dually chordal.
        \item \label{case:G_maxNeigh} $G$ hat eine maximale Nachbarschaftsordnung.
        \item \label{case:G_chordalClique} $G$ ist der Cliquengraph eines chordalen Graphen.
        \item \label{case:G_alphaLine} $G$ ist der Linegraph eines $\alpha$-azyklischen Hypergraphen.
        \item \label{case:G_SpanningClique} $G$ besitzt einen Spannbaum $T$, so dass jede maximale Clique in $G$ einen Teilbaum von $T$ induziert.
        \item \label{case:G_SpanningPath} $G$ besitzt einen Spannbaum $T$, so dass für alle Kanten $uv \in E$ gilt: $\forall\, w \in P_T(u,v):uw, vw \in E$.
    \end{enumerate}
    
\end{Theorem}

\begin{Proof}

    %\todo{Einführungssatz}
    
    \ref{case:G_dually} $\Leftrightarrow$ \ref{case:G_maxNeigh}:  Definition~\ref{def:duallyChordal} (S.~\pageref{def:duallyChordal})
    
    \ref{case:G_dually} $\Leftrightarrow$ \ref{case:G_chordalClique}:  Satz~\ref{theo:CliqueChorDuallyChor} (S.~\pageref{theo:CliqueChorDuallyChor})
    
    \ref{case:G_dually} $\Leftrightarrow$ \ref{case:G_alphaLine}:  Satz~\ref{theo:AlphaAzyklDuallyChordal} (S.~\pageref{theo:AlphaAzyklDuallyChordal})

    \ref{case:G_dually} $\Leftrightarrow$ \ref{case:G_SpanningClique}:  Satz~\ref{theo:DuallyChordalSpanningTree} (S.~\pageref{theo:DuallyChordalSpanningTree})

    \ref{case:G_SpanningClique} $\Leftrightarrow$ \ref{case:G_SpanningPath}:  Satz~\ref{theo:SpanningTree} (S.~\pageref{theo:SpanningTree})
\qed
\end{Proof}

\chapter{Dominierende Mengen}

%% Kommandos für Sechsecke

% Defieniert eine Koordinate (hexBase). Beispiel: \HexagonBase{\x}{\y}{\len}
%\newcommand{\HexagonBase}[3]
%{
%    \coordinate (hexBase) at ($2*#3*cos(30)*(#1,0)+#2*cos(30)*(#3,0)+1.5*#3*(0,#2)$);
%}
%
%\newcommand{\HexagonTopLeft}[1]{($(hexBase)+(30:#1)$)}
%\newcommand{\HexagonTop}[1]{($(hexBase)+(90:#1)$)}
%\newcommand{\HexagonTopRight}[1]{($(hexBase)+(150:#1)$)}
%\newcommand{\HexagonBottomRight}[1]{($(hexBase)+(210:#1)$)}
%\newcommand{\HexagonBottom}[1]{($(hexBase)+(270:#1)$)}
%\newcommand{\HexagonBottomLeft}[1]{($(hexBase)+(330:#1)$)}
%
% Zeichnet ein Sechseck an den übergeben Koordinaten
% Beispiel \HexagonComplete{\x}{\y}{\len}
%\newcommand{\HexagonComplete}[4]
%{
%    \HexagonBase{#1}{#2}{#3}
%    \draw[#4] \HexagonTopLeft{#3}--\HexagonTop{#3}--\HexagonTopRight{#3}--\HexagonBottomRight{#3}--\HexagonBottom{#3}--\HexagonBottomLeft{#3}--\HexagonTopLeft{#3};
%}

Dieses Kapitel befasst sich mit dominierenden Knoten- und Kantenmengen in Graphen. Es werden verschiedene Varianten vorgestellt und Verbindungen zwischen ihnen aufgezeigt. 
%Kapitel befasst sich mit:\\
%- Dominierende Knoten- / Kanten-Mengen\\
%- verschiedene Varianten davon\\
%- Zusammenhänge dieser Varianten
%
\section{Einleitung}

Angenommen, man hat eine Fläche gegeben, die in einzelne Teilflächen (Parzellen) eingeteilt ist. Ein entsprechendes Beispiel ist in Abbildung~\ref{pic:bsp_DominierendeMenge} gegeben. Diese in Parzellen unterteilte Fläche soll nun komplett abgedeckt werden. Denkbare Anwendungen hierfür sind beispielsweise Video-Überwachung, Bewässerung von Feldern oder das Anbieten von Funknetzen.

 Um eine solche Abdeckung zu erreichen, werden nun einige Parzellen ausgewählt. Eine gewählte Parzelle deckt dabei sich selbst und ihre benachbarten Parzellen ab. Für das aktuelle Beispiel seien zwei Parzellen benachbart, wenn sie zumindest einen gemeinsamen Punkt haben. In Abbildung~\ref{pic:bsp_DominierendeMenge} wurde eine Parzelle gewählt und ihr Abdeckungsbereich grün eingefärbt.

\newcommand{\taCoord}[2]{($#1*(\len,0)+#2*(60:\len)$)}
\begin{figure}[htbp]
    \centering
    \begin{tikzpicture}
        \def\len{0.66}
        \def\size{3}

        \fill[clLight60Green] \taCoord{2}{-1}--\taCoord{2}{1}--\taCoord{3}{1}--\taCoord{5}{-1}--\taCoord{5}{-2}--\taCoord{3}{-2}--cycle;
        
        
        \foreach \y [evaluate=\y as \xMinVal using int(\y<0?0-\y:0),evaluate=\y as \xMaxVal using int(\y>0?\size+\size-\y:\size+\size)] in {-\size,...,\size}
        {
        
            %Parzellen zeichnen
            \draw[gray] \taCoord{\xMinVal}{\y} -- \taCoord{\xMaxVal}{\y};
            \draw[gray] \taCoord{\y+\size}{\xMinVal-\size} -- \taCoord{\y+\size}{\xMaxVal-\size};
            
            \draw[gray] \taCoord{\xMinVal}{int(\y>0?\y-1:\size)} -- 
                        \taCoord{int(\y>0?\xMinVal+3*\size-\xMaxVal+\xMinVal-1:2*\size)}{int(\y>0?\y-3*\size+\xMaxVal-\xMinVal:\size-\xMaxVal+\xMinVal)};
            
            % Koordinaten            
            \foreach \x in {\xMinVal,...,\xMaxVal}
            {               
                %\node[red] at \taCoord{\x}{\y} {\tiny\x,\y};
            }
        }
        
        \draw[very thick,clDark50Green] \taCoord{2}{-1}--\taCoord{2}{1}--\taCoord{3}{1}--\taCoord{5}{-1}--\taCoord{5}{-2}--\taCoord{3}{-2}--cycle;
        
        \fill[very thick,clDark50Green] \taCoord{3}{-1}--\taCoord{3}{0}--\taCoord{4}{-1}--cycle;
        
        
        % Rand zeichnen
        \draw[thick] \taCoord{0}{0} -- \taCoord{0}{\size} -- \taCoord{\size}{\size} -- 
                     \taCoord{2*\size}{0} -- \taCoord{2*\size}{-\size} -- \taCoord{\size}{-\size} -- cycle;
                     
    \end{tikzpicture}
    \caption[Beispiel für eine in Parzellen unterteilte Fläche]{Beispiel für eine in Parzellen unterteilte Fläche. Die dunkelgrün eingefärbte Parzelle deckt sich selbst und ihre Nachbarn (hellgrün) ab.}
    \label{pic:bsp_DominierendeMenge}
\end{figure}

%\begin{figure}[htbp]
%    \centering
%    \begin{tikzpicture}
%        \def\len{0.3}
%        %\foreach \y [evaluate=\y as \yval using ifthenelse(\y>=0, 8-int(\y), 9+\y)] in {-5,...,4}{
%        %\foreach \y [evaluate=\y as \yval using 8-abs(\y)+0.5*div(\y,abs(\y))+0.5)] in {-5,...,4}{
%        \foreach \y [evaluate=\y as \xMinVal using int(\y<0?0-\y:0),evaluate=\y as \xMaxVal using int(\y>0?8-\y:8)] in {4,...,-4}{
%        \foreach \x in {\xMinVal,...,\xMaxVal}
%        {               
%            %\coordinate (seb) at ($2*\len*cos(30)*(\x,0)+\y*cos(30)*(\len,0)+1.5*\len*(0,\y)$);
%            %\Sechseck
%            \HexagonComplete{\x}{\y}{\len}{}
%            \node[gray] at (hexBase) {\tiny\x,\y};
%        }
%        }
%    \end{tikzpicture}
%    \caption{Beispiel dominierende Menge -- TODO}
%    \label{pic:bsp_DominierendeMenge}
%\end{figure}

Gesucht ist nun eine (möglichst kleine) Menge von Parzellen, damit die gesamte Fläche abgedeckt wird. Eine solche Menge wird als \emph{dominierende Menge} bezeichnet. Überträgt man das Konzept auf einen Graphen, so ergibt sich folgende Definition:


%In der Regel bezeichnet eine dominierenden Menge (engl. dominating set) eine Teilmenge der Knoten eines Grahen (siehe \cite[Definition~1.1.18]{brandstaedt1999graph} oder \cite{wikiE:DomSet}). Da sich diese Arbeit aber mit dominierenden Knoten- und Kantenmengen befasst, sei an dieser Stelle eine allgemeine Definition gegeben.

%Dieses Konzept sei nun allgemein definiert für eine belibige Menge.
%\begin{mydef}[dominierende Menge\index{dominierende Menge}]
%    Es sei $M$ eine nichtleere, endliche Menge, $\wp(M)$ deren Potenzmenge und $\mN : M \rightarrow \wp(M)$ eine Nachbarschaftsfunktion.
%    
%    Eine Menge $D\subseteq M$ ist \emph{dominierend} genau dann, wenn gilt:
%    \[    \bigcup_{d \in D}\mN(d) \cup D= M \]
%\end{mydef}
%
%Das dazugehörige Problem sucht nach der kleinsten Teilmenge, die dominierend ist. Im gewichteten Fall wird dabei die Summe der Gewichte minimiert.
%
%\begin{mydef}[Dominierungs-Problem]
%\textbf{Eingabe:} Eine Menge $M$, eine Nachbarschaftsfunktion $\mN : M \rightarrow \wp(M)$, eine Gewichtsfunktion $\omega:M \rightarrow \mathbb{R}$ und ein Maximalgewicht $k$.
%
%\\ \vspace*{5pt}
%\textbf{Frage:} Existiert eine dominierende Menge $D \subseteq M$ mit $\sum_{d \in D}\omega(d) \leq k$?
%\end{mydef}
%
%Der ungewichtete Fall lässt sich dabei auch als gewichteter betrachten, wobei alle Knoten das gleiche Gewicht haben.

%Ein solche allgemeine Definition ist natürlich auch für Graphen gültig. Interessanter ist jedoch, dass sich diese allgemeine Form  immer so umwandeln lässt, dass nach einer dominierenden Knotenmenge in einem Graphen gesucht wird. Dazu übernimmt man $M$ als Knotenmenge und fügt eine (gerichtete) Kante $uv$ genau dann ein, wenn $u$ in der Nachbarschaft von $v$ liegt ($u\in \mN(v)$). Somit ist die nachfolgende Definition genau so allgemeingültig, wie vorherige.

\begin{mydef}[dominierende Menge\index{dominierende Menge}]
    Es sei $G=(V,E)$ ein Graph. Eine Knotenmenge $D \subseteq V$ ist \emph{dominierend} für $G$ genau dann, wenn gilt:
    \[ \bigcup_{d \in D}N[d] = V \]
\end{mydef}

%\todo{Graph für Felderbeispiel vom Anfang}

Das dazugehörige Problem sucht nach einer kleinsten Teilmenge, die dominierend ist. Im gewichteten Fall wird dabei die Summe der Gewichte minimiert. Bereits ungewichtet ist das Problem NP-vollständig \cite{garey1979computers}.

Zur Abgrenzung der in Abschnitt~\ref{sec:DomVar} vorgestellten Varianten wird eine dominierende Menge, die keinen weiteren Bedingungen unterliegt, in dieser Arbeit als \emph{simple} bezeichnet.

%\begin{mydef}[domination problem]
%    \textbf{Eingabe:} Ein Graph $G=(V,E)$, eine Gewichtsfunktion $\omega:V \rightarrow \mathbb{R}$ und ein Maximalgewicht $k$.
%
%    \textbf{Frage:} Existiert eine dominierende Menge $D \subseteq M$ mit $\sum_{d \in D}\omega(d) \leq k$?
%\end{mydef}

\section{Unabhängige Mengen}
Bevor im nächsten Abschnitt verschiedene Varianten von dominierenden Mengen vorgestellt werden, definiert dieser Abschnitt kurz das Konzept von unabhängigen Mengen.

Eine unabhängige Menge (engl.: independent set) ist eine Knotenmenge, bei denen kein Knoten mit einem anderen benachbart ist. Das Ermitteln eines größten independent sets in einem Graphen ist NP-vollständig.

\begin{mydef}[unabhängige Menge]
    In einem Graphen $G=(V,E)$ ist eine Teilmenge der Knoten $\mI \subseteq V$ eine \emph{unabhängige Knotenmenge} genau dann, wenn gilt: \[ \forall\, u,v \in \mI: uv \notin E \]
\end{mydef}

%Ähnlich wie bei Funktionen (siehe~\cite{wikiD:Extremwert}) kann man bei unabhängigen Mengen zwischen einem lokalen und einem globalen Maximum unterscheiden. In \cite{brandstaedt1999graph} wird zwischen einem \emph{maximum} bzw. \emph{maximal} independent set unterschieden.
%
%Im Fall eines lokal maximalen independent sets $\mI$ gibt es in einem Graphen~$G$ keinen weiteren Knoten~$v$, so dass $\mI \cup \{v\}$ ebenfalls unabhängig ist. Ist ein independent set $\mI$ ein globales Maximum, dann existiert kein weiteres independent set $\mI^*$ mit $|\mI|<|\mI^*|$.
%
%Jedes (lokal) maximale independent set~$\mI$ stellt auch eine dominierende Menge dar. Gäbe es einen Knoten~$v$, der nicht von $\mI$ dominiert wird, wäre $\mI \cup \{v\}$ ebenfalls ein independent set und somit $\mI$ nicht maximal.

\section{Varianten dominierender Mengen}\label{sec:DomVar}
Die Anforderungen an eine dominierende Menge lassen sich erweitern. Dieser Abschnitt stellt drei Varianten für dominierende Mengen vor.

\subsection{Independent Dominating Set}
Ein mögliches Kriterium ist, dass die Knoten der dominierenden Menge ein independent set bilden. Man spricht dann von einem \emph{independent dominating set}.

\begin{mydef}[independent dominating set]
    In einem Graphen $G=(V,E)$ ist eine dominierende Menge $D \subseteq V$ \emph{unabhängig} genau dann, wenn gilt:
    \[\forall \, d,e \in D: de \notin E\]
\end{mydef}

\subsection{Perfect Dominating Set}
Eine weitere Variante wird als \emph{perfect dominating set} bezeichnet. Dabei wird jeder Knoten, der nicht zur dominierenden Menge gehört, von genau einem Knoten dominiert.

\begin{mydef}[perfect domination]
    Es sei $D \subseteq V$ eine dominierende Menge im Graphen $G=(V,E)$. $D$ ist \emph{perfekt} genau dann, wenn für alle $v \in V \setminus D$ gilt:
    \[ \exists !\, d \in D: v \in N[d] \]
\end{mydef}

Bei einem perfect dominating set ist zu beachten, dass es sich nur auf die Knoten bezieht, die nicht zur dominierenden Menge gehören. Es muss sich somit nicht notwendigerweise auch um eine unabhängige dominierende Menge handeln.

\subsection{Efficient Dominating Set}

%\todo{$D$ ist ED in $G$ $\iff$ $D$ ist dom set in $G$ und ind set in $G^2$}

Ist eine dominierende Menge sowohl unabhängig als auch perfekt, dann spricht man von einem \emph{efficient dominating set}. Dabei ist jeder Knoten in der abgeschlossenen Nachbarschaft von genau einem Knoten der dominierenden Menge.

\begin{mydef}[efficient dominating set]
    In einem Graphen $G=(V,E)$ ist eine Menge $D \subseteq V$ ein \emph{efficient dominating set} genau dann, wenn gilt:
    \[ \forall\, v \in V: \exists!\, d \in D: v \in N[d] \]
\end{mydef}

Während sowohl eine unabhängige als auch eine perfekte dominierende Menge für jeden Graphen möglich sind, besitzt hingegen nicht jeder Graph ein efficient dominating set. Beispiele hierfür sind unter anderem die Graphen $C_4$ und $C_5$.

\subsection{Bekannte Fälle}
Für die in Kapitel~\ref{chp:Graphen} vorgestellten Graphenklassen ist für einige Varianten ein effizienter Algorithmus bekannt oder es konnte nachgewiesen werden, dass die Variante NP-vollständig ist. Die Tabelle~\ref{tbl:domination} gibt dazu eine Übersicht.
\begin{table}[htbp]
    \centering
    \setlength\tabcolsep{.75em}
    \setlength\arrayrulewidth{1pt}
    %\renewcommand{\arraystretch}{1.4} 
    \arrayrulecolor{clChapTxt}
    \begin{tabularx}{\linewidth}{@{}lXl@{}}
        \midrule[1pt]
        \bfseries\rmfamily\textcolor{clSecTxt}{Klasse} & \bfseries\rmfamily\textcolor{clSecTxt}{Dominierungs-Variante} & \bfseries\rmfamily\textcolor{clSecTxt}{Komplexität}  \\
        \midrule[1pt]
        chordal & simple & NP-vollständig \cite{booth1982} \\
         & independent & Linear \cite{Faber1982} \\
         & weighted independent & NP-vollständig \cite{GerardJ2004} \\
         & efficient & NP-vollständig \cite{ChainChin1996} \\
        \midrule[.4pt]
        dually chordal & simple & Linear \cite{Brandstaedt199843} \\
        \midrule[.4pt]
        strongly chordal\footnotemark & simple & Linear \cite{Faber1984} \\
         & weighted independent & Linear \cite{Faber1984} \\
         \midrule[.4pt]
         Intervall- & efficient & Linear \cite{wipdIntervall} \\
        \midrule[1pt]
    \end{tabularx}
    \caption{Übersicht über die Komplexität verschiedener Dominierungsprobleme}
    \label{tbl:domination}
    %\footnote{mit gegebener strong perfect elimination ordering} 
\end{table}
\footnotetext{mit gegebener strong perfect elimination ordering}


\section{Dominierende Kantenmengen}

Bei Kantenmengen sucht man häufig nach sogenannten Matchings. Dabei handelt es sich um eine Teilmenge der Kanten eines Graphen, wobei zwei Kanten keinen gemeinsamen Knoten besitzen.

\begin{mydef}[Matching\index{Matching}]\label{def:Matching}
    Gegeben sei ein Hypergraph $H=(V,\mE)$. Eine Menge $M \subseteq \mE$ heißt \emph{Matching} genau dann, wenn für alle $e,e' \in M$ ($e \neq e'$) gilt: $e\cap e' = \emptyset$.
\end{mydef}

Ein solches Matching kann nun auch dominierend sein. Dabei sollen die anderen Kanten dominiert werden. Zusätzlich sei noch die Bedingung gegeben, dass zwischen zwei Kanten des Matchings mindestens zwei weitere Kanten liegen. Man spricht dann von einem \emph{induced} Matching.

\begin{mydef}[dominating induced Matching\index{dominating induced Matching}]
Gegeben sei ein Hypergraph $H=(V,\mE)$ und ein Matching $M$. $M$ ist ein \emph{dominating induced Matching} genau dann, wenn gilt:
%%% Original-Definition
% \forall\ e,e' \in M&: dist_G(e,e') \geq 2 & \text{\emph{induced}} \\
% \forall\ e \in \mE\, \backslash\, M &: \exists\ \varepsilon \in M \text{ mit } \varepsilon\cap e \neq \emptyset & \text{\emph{dominating}}
\[ \forall\,e \in \mE : \exists!\, m \in M \text{ mit } m\cap e \neq \emptyset \]
\end{mydef}

Vergleicht man diese Definition nun mit der für ein efficient dominating set, so stellt man folgenden Zusammenhang fest:

\begin{Theorem}\label{theo:DimIffED}
    In einem Hypergraphen $H=(V,\mE)$ ist eine Kantenmenge $M \subseteq \mE$ ein dominating induced Matching genau dann, wenn $M$ ein efficient dominating set im Linegraph~$L(H)$ ist.
\end{Theorem}

\begin{Proof}
    Es sei $H=(V,\mE)$ ein Hypergraph, $L(H)=(\mE,E)=(V',E)$ dessen Linegraph und $M$ ein dominating induced Matching in $H$.

    Es gilt nun per Definition:
    \[ \forall\, e \in \mE: \exists! \,m \in M: m\cap e \neq \emptyset \]
    
    Bildet man nun den Linegraphen~$L(H)$, kann man die Aussage wie folgt formulieren:
    \[ \forall\, e \in V': \exists!\,m \in M: (me \in E \lor m=e) \]
    
    In einem Graphen ist ein Knoten~$m$ mit einem Knoten~$e$ verbunden oder identisch ($me \in E \lor m=e$) genau dann, wenn $m$ in der abgeschlossenen Nachbarschaft von $e$ liegt ($m \in N[e])$. Somit lässt sich die obige Aussage erneut umformen.
    \[ \forall\, e \in V': \exists!\,m \in M: e \in N[m] \]
    
    Dies entspricht nun der Definition eines efficient dominating sets. Da alle durchgeführten Umformungen Äquivalenzumformungen waren, ist somit auch Satz~\ref{theo:DimIffED} erfüllt.
    \qed
\end{Proof}

%\begin{align*}
%    M \subseteq \mE \text{ ist DIM in H} & \Leftrightarrow \forall\, e \in \mE: \exists!\,m \in M: m\cap e \neq \emptyset \\
%     & & \text{(Linegraph bilden)}\\
%    M \subseteq V' \text{ ist DIM in H} & \Leftrightarrow \forall\, e \in V': \exists!\,m \in M: (me \in E \lor m=e) \\ 
%                                        & \Leftrightarrow \forall\, e \in V': \exists!\,m \in M: e \in N[m] \\ 
%                                        & \Leftrightarrow M \subseteq V' \text{ ist PID in $L(H)$}\\ 
%\end{align*}
%
Für einen Hypergraphen~$H$ lässt sich somit ein dominating induced Matching ermitteln, indem man ein efficient dominating set in dessen Linegraph $L(H)$ sucht.

\section{Reduktion von Efficient Domination}\label{sec:Reduktion}
%Wie oben gezeigt, lässt sich ein dominating induced Matching für einen Hypergraphen $H$ dadurch ermitteln, dass man ein efficient dominating set für dessen Linegraph $L(H)$ ermittelt. 
In diesem Abschnitt wird nun gezeigt, dass sich die Suche nach einem efficient dominating set in einem Graphen reduziert werden kann auf die Suche nach einer dominierenden Menge mit minimalem Gewicht und auf die Suche nach einer unabhängigen Menge mit maximalem Gewicht im Quadrat des Graphen.

%\subsubsection{Definitionen}
%\paragraph{dominating induced Matching (DIM):} Gegeben: Hypergraph $H=(V,\mE)$
%\[ M \subseteq \mE \text{ ist DIM}:\Leftrightarrow \forall\, e \in \mE: \exists!\,m \in M: m\cap e \neq \emptyset \]
%
%\paragraph{independent perfect dominating set (PID):} Gegeben: Graph $G=(V,E)$
%\[ D \subseteq V \text{ ist PID} :\Leftrightarrow \forall\, v \in V: \exists!\,d \in D: v \in N[d] \]
%
%Folgerung:
%\[ D \text{ ist PID} \Rightarrow \forall\, u,v \in D (u \neq v): N[u] \cap N[v] = \emptyset \]
%
%\subsubsection{Reduktion von DIM auf PID}
%Gegeben: Hypergraph $H=(V,\mE)$ und Linegraph $(V',E)=L(H)=(\mE,E)$
%\begin{align*}
%    M \subseteq \mE \text{ ist DIM in H} & \Leftrightarrow \forall\, e \in \mE: \exists!\,m \in M: m\cap e \neq \emptyset \\
%     & & \text{(Linegraph bilden)}\\
%    M \subseteq V' \text{ ist DIM in H} & \Leftrightarrow \forall\, e \in V': \exists!\,m \in M: (me \in E \lor m=e) \\ 
%                                        & \Leftrightarrow \forall\, e \in V': \exists!\,m \in M: e \in N[m] \\ 
%                                        & \Leftrightarrow M \subseteq V' \text{ ist PID in $L(H)$}\\ 
%\end{align*}
%

Der Kern der Reduktionen ist die Gewichtung der Knoten. Dafür sei eine Gewichtsfunktion $\omega$ wie folgt definiert: Jeder Knoten~$v$ erhält als Gewicht die Anzahl der Knoten in seiner abgeschlossenen Nachbarschaft.
\[ \omega(v) := |N[v]| \]

 Um eine kürzere Schreibweise zu ermöglichen, seien zusätzlich für eine Knotenmenge $D \subseteq V$ deren Gewicht~$\omega(D)$ und deren Nachbarschaft~$N[D]$ definiert.
\begin{align*}
    \omega(D) &:= \sum_{v\in D}\omega(v) = \sum_{v\in D}|N[v]| \\
    N[D] &:= \bigcup_{v\in D}N[v]
\end{align*}

\subsection{Reduktion auf Weighted Domination}
Es seien $G=(V,E)$ ein Graph und $D \subseteq V$ eine Knotenmenge in $G$.

\begin{Lemma}\label{lem:OmegaDomGeqV}
    Wenn $D$ dominierend ist, dann gilt: $\omega(D) \geq |V|$.
\end{Lemma}

\begin{Proof}
    Angenommen $\omega(D) < |V|$. Das bedeutet, dass die Anzahl der Knoten in der Nachbarschaft $N[D]$ von $D$ kleiner ist, als die Anzahl der Knoten im Graphen. Somit muss es einen Knoten geben, der nicht in $N[D]$ liegt. Also kann $D$ keine dominierende Menge sein.
    \qed
\end{Proof}

\begin{Lemma}\label{lem:OmegaDomEqualV}
    Ist $D$ ein efficient dominating set, dann gilt: $\omega(D) = |V|$.
\end{Lemma}

\begin{Proof}
    Aufgrund von Lemma~\ref{lem:OmegaDomGeqV} ist $\omega(D) \geq |V|$. Es sei nun angenommen, dass $\omega(D) > |V|$ ist. Somit wird ein Knoten~$u$ für die Gewichtsfunktion mehrfach gezählt. Das bedeutet, dass $u$ in der Nachbarschaft von mehreren Knoten aus $D$ liegt. Also kann $D$ auch kein efficient dominating set sein.
    \qed
\end{Proof}

%\paragraph{Lemma (*):} $D$ ist dominierend $\Rightarrow$ $\omega(D) \geq |V|$
%
%\emph{Beweis:}\nopagebreak\\
%Angenommen, $\omega(D) < |V|$
%\begin{align*}
%    &\Rightarrow \sum_{v \in D}|N[v]| < |V| \\
%    &\Rightarrow \bigcup_{v \in D}N[v] \subset V \\
%    &\Rightarrow \text{$D$ ist nicht dominierend}
%\end{align*}
%\qed

%\paragraph{Lemma (**):} $D$ ist PID $\Rightarrow$ $\omega(D) = |V|$
%
%\emph{Beweis:}\nopagebreak\\
%Angenommen, $\omega(D) < |V|$ -- Widerspruch zu (*).
%
%Angenommen, $\omega(D) > |V|$
%\begin{align*}
%    &\Rightarrow \sum_{v \in D}|N[v]| > |V| \\
%    &\Rightarrow \exists\, u,v \in D (u \neq v):N[u] \cap N[v] \neq \emptyset \\
%    &\Rightarrow \text{$D$ ist nicht independent perfect}
%\end{align*}
%\qed

%\paragraph{Reduktion:} $D$ ist PID $\Leftrightarrow$ $D$ ist mwD und $\omega(D) = |V|$

\begin{Theorem}\label{theo:DEdIffOmegaEqV}
    Eine Knotenmenge $D$ ist ein efficient dominating set genau dann, wenn $D$ ein minimum weight dominating set mit $\omega(D)=|V|$ ist.
\end{Theorem}

\begin{Proof}
    \prR $D$ ist ein efficient dominating set. Aufgrund von Lemma~\ref{lem:OmegaDomEqualV} ist $\omega(D) = |V|$. Es sei nun angenommen, dass $D$ nicht minimal ist. Dann existiert eine dominierende Menge $D'$ mit $\omega(D') < |V|$. Dies steht im Widerspruch zu Lemma~\ref{lem:OmegaDomGeqV}.
    
    \prL $D$ ist eine dominierende Menge mit minimalem Gewicht, wobei $\omega(D) = |V|$ ist. Es sei nun angenommen, dass $D$ kein efficient dominating set ist. Dann gibt es zwei Knoten $u$ und $v$ in $D$ ($u,v \in D$, $u \neq v$), die einen gemeinsamen Nachbarn $w$ haben ($w \in N[u] \cap N[v]$). Somit wird $w$ für das Gesamtgewicht von $D$ mehrfach gezählt. Also ist $\omega(w) > |V|$. Dies steht im Widerspruch zur Voraussetzung.
    \qed
\end{Proof}

%\emph{Beweis:}\nopagebreak\\
%\prR $D$ ist PID.
%
%$\omega(D) = |V|$ -- Lemma (**)
%
%Angenommen, $D$ ist nicht minimal. Dann existiert ein $D'$ mit $\omega(D') < |V|$. -- Widerspruch zu Lemma (*)
%
%\prL $D$ ist mwD und $\omega(D) = |V|$.
%
%Angenommen, $D$ ist nicht dominierend -- Wiederspruch zur Vorraussetzung
%
%Angenommen, $D$ ist nicht independent perfect
%\begin{align*}
%    &\Rightarrow \exists\, u,v \in D (u \neq v):N[u] \cap N[v] \neq \emptyset \\
%    &\Rightarrow |N[u]|+|N[v]| > |N[u] \cup N[v]| \\
%    &\Rightarrow \sum_{v \in D}|N[v]| = \omega(D) > |V| \\
%    &\text{Wiederspruch zur Vorraussetzung}
%\end{align*}
%\qed

\subsection{Reduktion auf Weighted Independent Set}
Es seien $G=(V,E)$ ein Graph, $N[v]$ die abgeschlossene Nachbarschaft von $v$ in $G$ und $\mI$ ein maximum weight independent set in $G^2$. 

\begin{Lemma}\label{lem:NachbarschaftMaxIndSetDisjunkt}
    Für alle Knoten $i,j\in\mI$ ($i\neq j$) gilt: $N[i] \cap N[j] = \emptyset$.
\end{Lemma}

\begin{Proof}
    Angenommen, es existiert ein Knoten $v\in V\setminus \mI$ und zwei Knoten $i,j \in \mI$ ($i\neq j$) mit $v \in N[i]\cap N[j]$. Dann ist der Abstand von $i$ zu $j$ höchstens $2$. Somit ist $ij$ eine Kante in $G^2$. Also ist $\mI$ kein independent set in $G^2$. Dies steht im Widerspruch zur Voraussetzung. \qed
\end{Proof}

Aufgrund von Lemma~\ref{lem:NachbarschaftMaxIndSetDisjunkt} gilt nun: \[ \omega(\mI) \leq |V| \]

\begin{Theorem}
    $\mI$ ist ein efficient dominating set in $G$ genau dann, wenn $\omega(\mI)=|V|$ ist.
\end{Theorem}

\begin{Proof}
    \prR $\mI$ ist ein efficient dominating set. Aufgrund von Lemma~\ref{lem:OmegaDomEqualV} ist $\omega(\mI) = |V|$.
    
    \prL $\omega(\mI) = |V|$

    Angenommen, $\mI$ ist nicht dominierend in $G$. Dann existiert ein Knoten $v \in V$, für den es keinen Knoten $i \in \mI$ gibt mit $v \in N[i]$. Aufgrund von Lemma~\ref{lem:NachbarschaftMaxIndSetDisjunkt} ist somit $\omega(\mI) < |V|$.

    Daraus folgt, dass $\mI$ dominierend in $G$ ist. Aufgrund von Satz~\ref{theo:DEdIffOmegaEqV} ist $\mI$ somit auch ein efficient dominating set.
    \qed
\end{Proof}

\subsection{Zusammenfassung}
Das oben Bewiesene lässt sich nun wie folgt zusammenfassen:
\begin{Theorem}\label{theo:EEDiffEDiffMWDiffMWIS}
    Gegeben seien ein Hypergraph $H$, dessen Linegraph $\mL=L(H)=(V,E)$ sowie eine Gewichtsfunktion $\omega(v\in V):=|N[v]|$, wobei $N[v]$ die abgeschlossene Nachbarschaft von $v$ in $\mL$ ist.
    
    Für eine Menge $M \subseteq V$ sind folgenden Aussagen äquivalent:
    \begin{enumerate}
        \item $M$ ist dominating induced Matching in $H$
        \item $M$ ist efficient dominating set in $\mL$
        \item $M$ ist minimum weight dominating set in $\mL$ mit $\omega(M) = |V|$
        \item $M$ ist maximum weight independent set in $\mL^2$ mit $\omega(M) = |V|$
    \end{enumerate}
\end{Theorem}

\chapter{Algorithmen}

%\todo{Algorithmendesign}
%\todo{Kapitelanfang neu formulieren}

Dieses Kapitel stellt einige Algorithmen vor, mit denen sich das Dominating Induced Matching Problem für azyklische Hypergraphen lösen lässt. Dabei wird dieses jedoch nicht direkt gelöst, sondern wie in Abschnitt~\ref{sec:Reduktion} gezeigt, auf die Suche nach einer dominierenden Menge bzw. einem independent set reduziert.

%Bei der Vorstellung der Algorithmen steht dabei die Algorithmusidde im Vordergrund. Für Implementierungsdetails oder Korrektheitsbeweise sei jeweils auf die Originalquelle verwiesen. Auch die Komplexität wird nicht genauer betrachtet. Zwar lassen sich einige der vorgestellten Algorithmen in Linearzeit implementieren, allerdings ist davon auszugehen, dass die Größe des Linegraphen bzw. dessen Quadrat jeweils quadratisch zunimmt. Somit muss für einen Linearzeitalgorithmus vermutlich ein anderer Ansatz verfolgt werden.

Das Kapitel beginnt mit der Überprüfung, ob ein Hypergraph $\alpha$-azyklisch ist. Die nachfolgenden Abschnitte befassen sich dann lediglich mit den entsprechenden Linegraphen. Es wird der Einfachheit halber davon ausgegangen, dass der Hypergraph $\beta$- bzw. $\gamma$-azyklisch war, wenn der Linegraph strongly chordal bzw. distanzerblich ist.

\section{Überprüfung des Hypergraphen}
Um zu überprüfen, ob ein Hypergraph $\alpha$-azyklisch ist, bieten sich verschiedene Möglichkeiten an. Nachfolgend werden drei Varianten vorgestellt.

\subsection{Mittels Definition}
Eine Variante ist es, schlicht zu überprüfen, ob ein gegebener Hypergraph die in Definition~\ref{def:alphaAzyklisch} (S.~\pageref{def:alphaAzyklisch}) gestellten Bedingungen erfüllt. In diesem Fall müsste der 2-Section-""Graph chordal sein und jede (maximale) Clique des 2-Section-""Graphen auch eine Hyperkante darstellen.

Wie bereits in Abschnitt~\ref{sec:chordalGraphs} (S.~\pageref{sec:chordalGraphs}) erwähnt, lässt sich in Linearzeit überprüfen, ob ein Graph chordal ist. In \cite{Rose1976} wird dazu ein Verfahren vorgestellt, bei dem auch eine perfekte Eliminationsordnung ermittelt wird (siehe auch Abschnitt~\ref{sec:SquareDC}). Mit dieser lassen sich dann auch die maximalen Cliquen ermitteln und mit den Hyperkanten vergleichen.

\subsection{Mittels Graham-Reduktion}
Eine weitere Möglichkeit bietet die Graham-Reduktion (siehe Definition~\ref{def:GrahamReduktion}, S.~\pageref{def:GrahamReduktion}). Aus ihrer Definition ergibt sich direkt ein Algorithmus, um zu testen, ob ein Hypergraph $\alpha$-azyklisch ist. Dazu entfernt man so lange entsprechende Knoten und Hyperkanten, bis nur noch eine leere Kante übrig bleibt, oder weder Knoten noch Kanten entfernt werden können.

\subsection{Mittels des Algorithmus von Tarjan und Yannakakis}
Ein Möglichkeit, die sich auch in Linearzeit implementieren lässt, wird in \cite{LinearTimeAcylicityTest} vorgestellt.

Der erste Schritt des Verfahren ist eine von~$i=1$ bis~$i=k$ aufsteigende Nummerierung der Knoten und Hyperkanten, wobei sich der Wert von $k$ erst während der Nummerierung ergibt. Die dabei vergebenen Nummern seien der $\beta$-Wert der Hyperkante bzw. des Knotens. Allerdings muss nicht jede Hyperkante~$e$ eine Nummer erhalten. In diesem Fall ist $\beta(e)$ nicht definiert. 

Für die Nummerierung wird zunächst eine Hyperkante~$e$ gewählt. Nun werden sowohl $e$ als auch alle unnummerierten Knoten~$v \in e$ mit $i$ nummeriert ($\beta(e) = \beta(v) = i$). Anschließend wird $i$ um eins erhöht ($i:=i+1$) und die nächste Hyperkante gewählt. Als nächste Hyperkante wird dabei eine Hyperkante gewählt, die noch unnummerierte Knoten enthält und deren Anzahl an nummerierten Knoten maximal ist. Dies wird nun so lange wiederholt, bis alle Knoten nummeriert wurden.

Neben dem $\beta$-Wert erhält eine Hyperkante~$e$ auch einen $\gamma$-Wert ($\gamma(e)$). Dieser ist wie folgt definiert:
\[ \gamma(e) = \begin{cases}
\max \{ \beta(v) \mid v \in e \} & \text{$\beta(e)$ ist nicht definiert} \\ 
\text{nicht definiert} & \forall\, v \in e: \beta(v) = \beta(e) \\ 
\max \{ \beta(v) \mid v \in e, \beta(v) < \beta(e) \} & \text{sonst} 
\end{cases} \]

%\[ \gamma(e) = \begin{cases}
%\text{nicht definiert} & \forall\, v \in e: \beta(v) = \beta(e) \\ 
%\max \{ \beta(v) \mid v \in e \} & \text{sonst} 
%\end{cases} \]


%\todo{eventuell Übersetzung für maximum cardinality search}
%Basis des Verfahrens ist eine sogenannte \emph{maximum cardinality search}.

%Bei der maximum cardinality search handelt es sich eine von $n=|V|$ bis~$1$ absteigende Nummerierung der Knoten eines Hypergraphen. Als nächster Knoten wird dabei ein unnummerierter Knoten gewählt, der in der Hyperkante mit den meisten nummerierten Knoten liegt. Ist es nicht eindeutig, welcher Knoten oder welche Hyperkante als nächstes zu wählen ist, kann belibig gewählt werden.
%
%Eine Hyperkante sei \emph{erschöpft} (engl.: exhausted), wenn sie keinen unnummerierten Knoten besitzt. Es gehöre nun eine Hyperkante~$e$ zu den nicht erschöpften Hyperkanten mit den meisten nummerierten Knoten. Außerdem wird als nächstes der Knoten~$v \in e$ nummeriert. Ist nun $e$ weiterhin unerschöpft, gehört $e$ somit auch weiterhin zu den Hyperkanten mit den meisten nummerierten Knoten. Somit lässt sich die maximum cardinality search wie folgt verändern: Nach der Wahl einer Hyperkante~$e$ mit den meisten unnummerierten Knoten, werden alle Knoten in ihr nummeriert, bis $e$ erschöpft ist.

In Abbildung~\ref{pic:Bsp_HypergraphGamaBeta} wurden für zwei Beispiel-Hypergraphen die $\beta$- und $\gamma$-Werte mit dem oben beschriebenen Verfahren vergeben.

\begin{figure}[htbp]
    \hspace*{\fill}
    \subfloat[]{
    \begin{tikzpicture}
        \def\len{1.25}
        \def\dis{0.025}
        
        \node[sN,lbl,inner sep=3pt] (n3) at (90:\len) { 1};
        \node[sN,inner sep=3pt] (n4) at (-30:\len) {\small 1};
        \node[sN,inner sep=3pt] (n2) at (210:\len) {\small 2};
        
        \node[sN,inner sep=3pt] (n1) at ($2*(150:\len)$) {\small 3};
        
%        \node[below=2pt,lbl] at ($-2*(0,\len)+2*cos(30)*(0,\len)$) {$2: 1$};
        \node[below=2pt,lbl] at ($-1*(0,\len)$) {$2:1$};
        \node[above right=2pt,lbl] at (30:\len) {$1:-$};
        \node[above=2pt,lbl] at ($(150:\len)+2*(0,\len)-2*cos(30)*(0,\len)$) {$3 :2$};
%        \path[thick,draw] ($(n3)+(-30:\dis)$) arc[start angle=-30,end angle=105,radius=\dis] -- ($(n1)+(105:\dis)$)
%            arc[start angle=105,end angle=195,radius=\dis] -- ($(n2)+(195:\dis)$)
%            arc[start angle=195,end angle=330,radius=\dis] -- cycle;
        \begin{pgfonlayer}{background}
        \path[very thick,draw,fill,clBlue] ($(n3)+(90:\dis)$) 
            arc[start angle=90,delta angle=-120,radius=\dis+\len]
            arc[start angle=-30,delta angle=-180,radius=\dis]
            arc[start angle=-30,delta angle=120,radius=-\dis+\len]
            arc[start angle=-90,delta angle=-180,radius=\dis];
        
        %\def\dis{0.6}
  
        \path[very thick,draw,fill,clOrange] ($(n2)+(30:\dis)$)
            arc[start angle=210,delta angle=120,radius=-\dis+\len]
            arc[start angle=150,delta angle=-180,radius=\dis]
            arc[start angle=-30,delta angle=-120,radius=\dis+\len]
            arc[start angle=210,delta angle=-180,radius=\dis];
            
  
        %\def\dis{0.7}
  
        \path[very thick,draw,fill,clGreen] ($(n3)+(-60:\dis)$)
            arc[start angle=-60,delta angle=180,radius=\dis]
            arc[start angle=-60,delta angle=-60,radius=-\dis+2*cos(30)*\len,label={[red]center:blub}] 
            arc[start angle=60,delta angle=180,radius=\dis]
            arc[start angle=60,delta angle=-60,radius=-\dis+2*cos(30)*\len]
            arc[start angle=180,delta angle=180,radius=\dis]
            arc[start angle=180,delta angle=-60,radius=-\dis+2*cos(30)*\len]
            ;

            %arc[start angle=-90,end angle=105,radius=\dis] -- ($(n1)+(105:\dis)$)
            %arc[start angle=105,end angle=195,radius=\dis] -- ($(n2)+(195:\dis)$)
            %arc[start angle=195,delta angle=195,radius=\dis]
            %arc[start angle=210,end angle=90,radius=-\dis+\len];
        
        %\def\dis{0.8}
  
%        \path[very thick,fill,clRed] ($(n3)+(0:\dis)$)
%            arc[start angle=0,end angle=180,radius=\dis]
%            arc[start angle=0,end angle=-60,radius=-\dis+2*cos(30)*\len]
%            arc[start angle=120,delta angle=180,radius=\dis]
%            arc[start angle=120,delta angle=-60,radius=-\dis+2*cos(30)*\len]
%            arc[start angle=240,delta angle=180,radius=\dis]
%            arc[start angle=240,delta angle=-60,radius=-\dis+2*cos(30)*\len];
        \end{pgfonlayer}
    \end{tikzpicture}
    }
    \hspace*{\fill}
    \subfloat[]{
    \begin{tikzpicture}
        \def\len{1.25}
        \def\dis{0.025}
        
        \node[sN,lbl,inner sep=3pt] (n3) at (90:\len) { 1};
        \node[sN,inner sep=3pt] (n4) at (-30:\len) {\small 1};
        \node[sN,inner sep=3pt] (n2) at (210:\len) {\small 2};
        
        \node[sN,inner sep=3pt] (n1) at ($2*(150:\len)$) {\small 3};
        
        \node[below=2pt,lbl] at ($-2*(0,\len)+2*cos(30)*(0,\len)$) {$2: 1$};
        \node[below=2pt,lbl] at ($-1*(0,\len)$) {$-: 2$};
        \node[above right=2pt,lbl] at (30:\len) {$1:-$};
        \node[above=2pt,lbl] at ($(150:\len)+2*(0,\len)-2*cos(30)*(0,\len)$) {$3 :2$};
%        \path[thick,draw] ($(n3)+(-30:\dis)$) arc[start angle=-30,end angle=105,radius=\dis] -- ($(n1)+(105:\dis)$)
%            arc[start angle=105,end angle=195,radius=\dis] -- ($(n2)+(195:\dis)$)
%            arc[start angle=195,end angle=330,radius=\dis] -- cycle;
        \begin{pgfonlayer}{background}
        \path[very thick,draw,fill,clBlue] ($(n3)+(90:\dis)$) 
            arc[start angle=90,delta angle=-120,radius=\dis+\len]
            arc[start angle=-30,delta angle=-180,radius=\dis]
            arc[start angle=-30,delta angle=120,radius=-\dis+\len]
            arc[start angle=-90,delta angle=-180,radius=\dis];
        
        %\def\dis{0.6}
  
        \path[very thick,draw,fill,clOrange] ($(n2)+(30:\dis)$)
            arc[start angle=210,delta angle=120,radius=-\dis+\len]
            arc[start angle=150,delta angle=-180,radius=\dis]
            arc[start angle=-30,delta angle=-120,radius=\dis+\len]
            arc[start angle=210,delta angle=-180,radius=\dis];
            
  
        %\def\dis{0.7}
  
        \path[very thick,draw,fill,clGreen] ($(n3)+(-60:\dis)$)
            arc[start angle=-60,delta angle=180,radius=\dis]
            arc[start angle=-60,delta angle=-60,radius=-\dis+2*cos(30)*\len,label={[red]center:blub}] 
            arc[start angle=60,delta angle=180,radius=\dis]
            arc[start angle=60,delta angle=-60,radius=-\dis+2*cos(30)*\len]
            arc[start angle=180,delta angle=180,radius=\dis]
            arc[start angle=180,delta angle=-60,radius=-\dis+2*cos(30)*\len]
            ;

            %arc[start angle=-90,end angle=105,radius=\dis] -- ($(n1)+(105:\dis)$)
            %arc[start angle=105,end angle=195,radius=\dis] -- ($(n2)+(195:\dis)$)
            %arc[start angle=195,delta angle=195,radius=\dis]
            %arc[start angle=210,end angle=90,radius=-\dis+\len];
        
        %\def\dis{0.8}
  
        \path[very thick,fill,clRed] ($(n3)+(0:\dis)$)
            arc[start angle=0,end angle=180,radius=\dis]
            arc[start angle=0,end angle=-60,radius=-\dis+2*cos(30)*\len]
            arc[start angle=120,delta angle=180,radius=\dis]
            arc[start angle=120,delta angle=-60,radius=-\dis+2*cos(30)*\len]
            arc[start angle=240,delta angle=180,radius=\dis]
            arc[start angle=240,delta angle=-60,radius=-\dis+2*cos(30)*\len];
        \end{pgfonlayer}
    \end{tikzpicture}
    }
%    \hspace*{\fill}
%    \subfloat[]{
%    \begin{tikzpicture}[scale=0.5]
%        \def\len{1.5}
%        \def\dis{0.5}
%        
%        \node[sN,inner sep=2pt] (n3) at (90:\len) {3};
%        \node[sN,inner sep=2pt] (n4) at (-30:\len) {4};
%        \node[sN,inner sep=2pt] (n2) at (210:\len) {2};
%        
%        \node[sN,inner sep=2pt] (n1) at ($(n3)+sin(60)/sin(45)*(195:\len)$) {1};
%        
%        \path[thick,draw] ($(n3)+(-30:\dis)$) arc[start angle=-30,end angle=105,radius=\dis] -- ($(n1)+(105:\dis)$)
%            arc[start angle=105,end angle=195,radius=\dis] -- ($(n2)+(195:\dis)$)
%            arc[start angle=195,end angle=330,radius=\dis] -- cycle;
%        
%        \path[thick,draw] ($(n3)+(90:\dis)$) 
%            arc[start angle=90,delta angle=-120,radius=\dis+\len]
%            arc[start angle=-30,delta angle=-180,radius=\dis]
%            arc[start angle=-30,delta angle=120,radius=-\dis+\len]
%            arc[start angle=-90,delta angle=-180,radius=\dis];
%        
%        \def\dis{0.6}
%  
%        \path[thick,draw] ($(n2)+(30:\dis)$)
%            arc[start angle=210,delta angle=120,radius=-\dis+\len]
%            arc[start angle=150,delta angle=-180,radius=\dis]
%            arc[start angle=-30,delta angle=-120,radius=\dis+\len]
%            arc[start angle=210,delta angle=-180,radius=\dis];
%            
%  
%        \def\dis{0.7}
%  
%        \path[thick,draw] ($(n3)+(-90:\dis)$)
%            arc[start angle=-90,end angle=105,radius=\dis] -- ($(n1)+(105:\dis)$)
%            arc[start angle=105,end angle=195,radius=\dis] -- ($(n2)+(195:\dis)$)
%            arc[start angle=195,delta angle=195,radius=\dis]
%            arc[start angle=210,end angle=90,radius=-\dis+\len];
%        
%        \def\dis{0.8}
%  
%        \path[thick,draw,clRed] ($(n3)+(0:\dis)$)
%            arc[start angle=0,end angle=180,radius=\dis]
%            arc[start angle=0,end angle=-60,radius=-\dis+2*cos(30)*\len]
%            arc[start angle=120,delta angle=180,radius=\dis]
%            arc[start angle=120,delta angle=-60,radius=-\dis+2*cos(30)*\len]
%            arc[start angle=240,delta angle=180,radius=\dis]
%            arc[start angle=240,delta angle=-60,radius=-\dis+2*cos(30)*\len];
%
%    \end{tikzpicture}
%    }
%    \hspace*{\fill}
%    \subfloat[]{
%    \begin{tikzpicture}[scale=0.5]
%        \def\len{1.5}
%        \def\dis{0.5}
%        
%        \node[sN] (n3) at (90:\len) {};
%        \node[sN] (n4) at (-30:\len) {};
%        \node[sN] (n2) at (210:\len) {};
%        
%        \node[sN] (n1) at ($(n3)+sin(60)/sin(45)*(195:\len)$) {};
% 
%        \path[thick,draw] ($(n3)+(30:\dis)$) arc[start angle=30,end angle=210,radius=\dis] -- ($(n4)+(-150:\dis)$) arc[start angle=-150,end angle=30,radius=\dis] -- cycle;
%        
%        \def\dis{0.6}
%  
%        \path[thick,draw] ($(n2)+(90:\dis)$) arc[start angle=90,end angle=270,radius=\dis] -- ($(n4)+(-90:\dis)$)
%            arc[start angle=-90,end angle=90,radius=\dis] -- cycle;
%            
%  
%        \def\dis{0.7}
%  
%        \path[thick,draw] ($(n3)+(-30:\dis)$) arc[start angle=-30,end angle=105,radius=\dis] -- ($(n1)+(105:\dis)$)
%            arc[start angle=105,end angle=195,radius=\dis] -- ($(n2)+(195:\dis)$)
%            arc[start angle=195,end angle=330,radius=\dis] -- cycle;
%        
%        \def\dis{0.8}
%  
%        \path[thick,draw] ($(n3)+(30:\dis)$) arc[start angle=30,end angle=150,radius=\dis] -- ($(n2)+(150:\dis)$)
%            arc[start angle=150,end angle=270,radius=\dis] -- ($(n4)+(270:\dis)$)
%            arc[start angle=-90,end angle=30,radius=\dis] -- cycle;
%        
%    \end{tikzpicture}
%    }
    \hspace*{\fill}
    \caption[Beispiel für die $\beta$- und $\gamma$-Werte in Hypergraphen]{Beispiel für die $\beta$- und $\gamma$-Werte in Hypergraphen: Die Zahlen in den Knoten sind deren $\beta$-Werte. Die erste Zahl einer Hyperkante ist ihr $\beta$- und die zweite ihr $\gamma$-Wert. (Beispiel entnommen aus \cite{LinearTimeAcylicityTest})}
    \label{pic:Bsp_HypergraphGamaBeta}
\end{figure}


Mit Hilfe der $\beta$- und $\gamma$-Werte lässt sich nun überprüfen, ob ein Hypergraph $\alpha$-azyklisch ist.

\begin{Theorem} \label{theo:AcyclicityTest} \cite{LinearTimeAcylicityTest}
     Es seien $H=(V,\mE)$ ein Hypergraph und $e_i \in \mE$ die Hyperkante mit dem $\beta$-Wert $i$ ($\beta(e_i) = i$).
    
    $H$ ist $\alpha$-azyklisch genau dann, wenn gilt:
    \[ \forall\,i \in \{1, \ldots, k\}: \forall\, e \in \mE\, (\gamma(e)=i):e\cap \{v \mid \beta(v) < i \} \subseteq e_i \]
\end{Theorem}

Aus Satz~\ref{theo:AcyclicityTest} leitet sich nun direkt ein Algorithmus ab. Dieser lässt sich so modifizieren, dass er in Linearzeit läuft.

\section{$\gamma$-azyklische Hypergraphen}
Die Linegraphen $\gamma$-azyklischer Hypergraphen sind distanzerblich chordal (Satz~\ref{theo:GammaLineGrpah}, S.~\pageref{theo:GammaLineGrpah}). Für diesen Abschnitt ist dabei die Distanzerblichkeit die interessante Eigenschaft. In \cite{Damiand200199} wird ein Algorithmus angegeben, mit dem sich in Linearzeit überprüfen lässt, ob ein gegebener Graph distanzerblich ist.

Interessant ist die Distanzerblichkeit eines Graphen deswegen, weil sie die Cliquenweite beschränkt.

\begin{Theorem} \cite{cliqueWidthPerfectGraphs}
    Die Cliquenweite von distanzerblichen Graphen ist höchstens 3. Der dazugehörige Cliquenweite-Ausdruck lässt sich in Linearzeit ermitteln.
\end{Theorem}

Ist die Cliquenweite einer Graphenklasse beschränkt und ist zusätzlich für einen Graphen der entsprechende Cliquenweite-Ausdruck gegeben, so lässt sich ein Problem für diesen Graphen in Linearzeit lösen, wenn es in monadischer Prädikatenlogik zweiter Stufe (engl.: monadic second-order logic)  definiert werden kann \cite{ClWidthSOL}. Prädikatenlogik zweiter Stufe erweitert die Prädikatenlogik erster Stufe um die Möglichkeit, dass auch Prädikate an Quantoren gebunden werden können. Beschränkt man sich zusätzlich auf einstellige Prädikate, dann spricht man von monadischer Prädikatenlogik zweiter Stufe. Eine ausführliche Betrachtung des Themas wird es voraussichtlich in \cite{GrStrucAndMSOL} geben.

 Gelingt es also, die Suche nach einem efficient dominating set in monadischer Prädikatenlogik zweiter Stufe zu formulieren, dann lässt sich diese Suche auch in Linearzeit für distanzerbliche Graphen abschließen.
 


\section{$\beta$-azyklische Hypergraphen}
Ist der gegebene Hypergraph $\beta$-azyklisch, so ist sein Linegraph strongly chordal. Dies schließt auch die Möglichkeit ein, dass der Linegraph ein Intervallgraph ist.

\subsection{Intervallgraphen}
Das Überprüfen, ob ein Graph ein Intervallgraph ist, ist in Linearzeit möglich. In \cite{Booth1976}, \cite{Habib200059} und \cite{korte68} sind dazu Algorithmen angegeben. Außerdem wird in \cite{wipdIntervall} ein Linearzeit-Algorithmus vorgestellt, mit dem sich ein effcient dominating set für einen Intervallgraphen ermitteln lässt.

%- Erkennen \\
%\cite{Booth1976,korte68,Habib200059} \\
%{\footnotesize \url{http://epubs.siam.org/sicomp/resource/1/smjcat/v18/i1/p68_s1?isAuthorized=no} \\
%\url{http://www.springerlink.com/content/p16527606774t37m/} \\
%\url{http://www.sciencedirect.com/science/article/pii/S0304397597002417}}
%
%- ind per dom \cite{wipdIntervall}

\subsection{Strongly chordale Graphen}
Im Fall der strongly chordalen Graphen ist bisher kein Algorithmus bekannt, der eine Erkennung in Linearzeit ermöglicht. Zwei der schnellsten Algorithmen laufen in $\mO(|E| \log|V|)$ \cite{PaigeTarjan1987}  bzw. $\mO(|V|^2)$  \cite{Spinrad1993229}. Der nächste Schritt ist das Ermitteln einer strong perfect elimination ordering. Der zeitliche Aufwand hierfür liegt bei $\mO(|V|^3)$  \cite{Anstee1984}. Mit Hilfe einer strong perfect elimination ordering lässt sich nun ein minimum weight independent dominating set in Linearzeit berechnen \cite{Faber1984}. Aufgrund von Satz~\ref{theo:EEDiffEDiffMWDiffMWIS} lässt sich somit auch überprüfen, ob der gegebene Graph ein efficient dominating set besitzt.


\section{$\alpha$-azyklische Hypergraphen}
Ist für einen Hypergraphen nur bekannt, dass er $\alpha$-azyklisch ist, so lässt sich der Linegraph auch lediglich auf die Klasse der dually chordalen Graphen eingrenzen. 


\subsection{Erkennen von dually chordalen Graphen}
Prinzipiell können auch Hypergraphen, die nicht $\alpha$-azyklisch sind, einen dually chordalen Graphen als Linegraphen besitzen. Somit wäre es sinnvoll die Klasse des Linegraphen statt des Hypergraphen zu ermitteln.

In \cite{duallyChordal} wird gezeigt, dass in Linearzeit getestet werden kann, ob ein Graph dually chordal ist. Allerdings wird dabei der Graph in einen Hypergraphen umgewandelt und getestet, ob der Hypergraph $\alpha$-azyklisch ist. Bei dieser Umwandlung wird der sogenannte \emph{Nachbarschaftshypergraph} gebildet. 

\begin{mydef}[Nachbarschaftshypergraph\index{Nachbarschaftshypergraph}\index{$\mN(\cdot)$}]
    Gegeben sei ein Graph $G=(V,E)$. Dessen \emph{Nachbarschaftshypergraph} $\mN(G)=(V,\mE)$ ist wie folgt definiert:
    \[ \mE = \{N[v] \mid v \in V\} \]
\end{mydef}

\begin{Theorem}
    \cite{duallyChordal} Ein Graph $G$ ist dually chordal genau dann, wenn sein Nachbarschaftshypergraph $\mN(G)$ $\alpha$-azyklisch ist.
\end{Theorem}

Da die Größe des Nachbarschaftshypergraphen linear mit der Größe des ursprünglichen Graphen wächst, lässt sich somit in Linearzeit testen, ob ein Graph dually chordal ist.

\subsection{Minimum Weight Dominating Set}
Um nun ein efficient dominating set in einem dually chordalen Graphen~$G$ zu finden, bietet Satz~\ref{theo:EEDiffEDiffMWDiffMWIS} prinzipiell zwei Möglichkeiten. Eine davon ist das Finden einer dominierenden Menge mit minimalem Gewicht in~$G$. Diese Variante erweist sich allerdings als ungeeignet.

\begin{Theorem}
    Das Ermitteln eines minimum weight dominating set ist NP-vollständig für dually chordalen Graphen.
\end{Theorem}

\begin{Proof}
    Gegeben sei ein beliebiger Graph $G=(V,E)$. Es wird nun ein Graph $G^*=(V^*,E^*)$ wie folgt erstellt:
    \begin{align*}
        V^* &= \{ v^* \} \cup V \\
        E^* &= \{v^*v \mid v \in V \} \cup E
    \end{align*}
    
    Offensichtlich ist $G^*$ dually chordal ($v^*$ ist maximaler Nachbar für alle Knoten). Eine Gewichtsfunktion $\omega$ sei nun wie folgt definiert:
    \[ \omega(v)=
            \begin{cases}
                |V| + 1 & \text{wenn } v = v^* \\ 
                1 & \text{sonst} 
            \end{cases}
    \]
    
    Es ist nun jede dominierende Menge $D \subseteq V$ für $G$ auch eine dominierende Menge in $G^*$. Außerdem ist das Gesamtgewicht von $D$ immer kleiner als das des Knotens~$v*$. Somit ist $D$ die kleinste dominierende Menge in $G$ genau dann, wenn $D$ die dominierende Menge mit dem kleinsten Gewicht in $G^*$ ist.
    \qed
\end{Proof}

\subsection{Quadrat dually chordaler Graphen}\label{sec:SquareDC}
Neben der Suche nach einer dominierenden Menge mit minimalem Gewicht bietet Satz~\ref{theo:EEDiffEDiffMWDiffMWIS} noch die Option, ein independent set mit maximalem Gewicht im Quadrat eines Graphen zu suchen.

\begin{Theorem}\label{theo:dcSquareChordal}
    \cite{Brandstaedt199843} Das Quadrat eines dually chordalen Graphen~$G$ ist chordal. Außerdem ist eine maximale Nachbarschaftsordnung für $G$ eine perfekte Eliminationsordnung für $G^2$.
\end{Theorem}

Da das Quadrat des vorliegenden Linegraphen chordal ist, ist es nun möglich, ein independent set zu ermitteln. In \cite{Frank1976} wird dafür ein Linearzeit"=Algorithmus vorgestellt. Voraussetzung für diesen ist eine perfekte Eliminationsordnung.

\subsubsection{Ermitteln einer perfekten Eliminationsordnung}
Es gibt verschiedene Möglichkeiten, eine perfekte Eliminationsordnung für chordale Graphen zu ermitteln. Einige Algorithmen mit linearer Laufzeit werden in \cite{Panda1996111}, \cite{Rose1976} und \cite{LinearTimeAcylicityTest} vorgestellt. Ein weiteres Verfahren, mit dem sich  jede perfekte Eliminationsordnung finden lässt, ist in \cite{Shier1984325} beschrieben.

Auch Satz~\ref{theo:dcSquareChordal} bietet einen Ansatz zum Finden einer perfekten Eliminationsordnung. Da es sich bei dem gegebenen chordalen Graphen um das Quadrat eines dually chordalen Graphen~$G$ handelt, kann auch eine maximale Nachbarschaftsordnung für $G$ ermittelt werden. In \cite{Brandstaedt199843} wird dafür der folgende Algorithmus vorgestellt. 

\begin{Algorithm}[MNO, \cite{Brandstaedt199843}]\label{algo:mno}
    \textbf{Eingabe:} Ein dually chordaler Graph $G=(V,E)$.\\
    \textbf{Ausgabe:} Eine maximale Nachbarschaftsordnung $(v_1,\ldots,v_n)$ von $G$.\\

    \begin{codeLine}
        \item Initialisiere alle Knoten $v \in V$ als unnummeriert und unmarkiert.
        \item Wähle einen beliebigen Knoten $v \in V$. Nummeriere $v$ mit $n$ (also $v_n = v$) und setze $mn(v) := v$.
        \item \textbf{Repeat}
              \begin{innerCodeLine}
                  \item Wähle aus allen unmarkierten Knoten einen nummerierten Knoten~$u$, so dass $N[u]$ eine maximale Anzahl an nummerierten Knoten enthält.
                  \item Nummeriere alle unnummerierten Knoten $x$ aus $N[u]$ fortlaufend mit der höchstmöglichen Nummer zwischen $1$ und $n-1$, die noch frei ist. Setze außerdem $mn(x) := u$.
                  \item Markiere $u$;
              \end{innerCodeLine}
              \textbf{Until} Alle Knoten sind nummeriert.
    \end{codeLine}
\end{Algorithm}

Die so ermittelte maximale Nachbarschaftsordnung für $G$ ist nun auch eine perfekte Eliminationsordnung für $G^2$. Außerdem weist der Algorithmus jedem Knoten einen maximalen Nachbarn zu. Beides wird sich später noch als hilfreich erweisen (siehe Abschnitt~\ref{sec:VermeidenDesQuadrats}). Abbildung~\ref{pic:bsp_Algo_MNO} zeigt die Anwendung des Algorithmus an einen Beispielgraphen.

\begin{figure}[htbp]
    \centering
    \begin{tikzpicture}
        
        \def\len{1.8}
        
        \node[nN,inner sep=2pt,fill=clLight40Green] (ll) at (0,0) {6};
        \node[nN,inner sep=2pt,fill=clLight40Green] (lt) at (30:\len) {5};
        \node[nN,inner sep=2pt,fill=clLight40Green] (lb) at (-30:\len) {4};
        \node[nN,inner sep=2pt] (rt) at ($(lt.center)+(\len,0)$) {2};
        \node[nN,inner sep=2pt,fill=clLight40Green] (rb) at ($(lb.center)+(\len,0)$) {3};
        \node[nN,inner sep=2pt] (rr) at ($(rb.center)+(30:\len)$) {1};
            
        \draw (lb) -- (lt) -- (rb) -- (rt) -- (rr);
        
        \draw[<-,very thick,clBlue] (ll)--(lt);
        \draw[<-,very thick,clBlue] (ll)--(lb);
        \draw[<-,very thick,clBlue] (lt)--(rt);
        \draw[<-,very thick,clBlue] (lb)--(rb);
        \draw[<-,very thick,clBlue] (rb)--(rr);
    \end{tikzpicture}
    \caption[Beispiel für Algorithmus~\ref{algo:mno}]{Beispiel für Algorithmus~\ref{algo:mno}. Die Nummerierung der Knoten ist die vom Algorithmus vergebene Nummerierung. Markierte Knoten sind grün hinterlegt. Die blauen Pfeile zeigen jeweils auf den maximalen Nachbarn eines Knotens (ausgenommen Knoten~$6$). Der Algorithmus beginnt mit Knoten~$6$ und wählt die Knoten danach in folgender Reihenfolge aus: $(6,4,5,3)$. Es sind allerdings auch andere Reihenfolgen möglich. (Graph entnommen aus~\cite{Frank1976})}
    \label{pic:bsp_Algo_MNO}
\end{figure}

\subsubsection{Ermitteln eines Maximum Weight Independent Sets}
Da nun eine perfekte Eliminationsordnung gegeben ist, lässt sich nun auch ein maximum weight independent set mittels des in \cite{Frank1976} vorgestellten Algorithmus berechnen. Ein Beispiel für die Anwendung des Algorithmus ist in Abbildung~\ref{pic:bsp_Algo_mwIS} gegeben.

\begin{Algorithm}[mwIS, \cite{Frank1976}]\label{algo:mwis}
    \textbf{Eingabe:} Ein chordaler Graph $G=(V,E)$ mit perfekter Eliminationsordnung $(v_1,\ldots,v_n)$ und einer Gewichtsfunktion~$\omega$.\\
    \textbf{Ausgabe:} Ein maximum weight independent set~$\mI$.\\

    \begin{codeLine}
        \item  $\mI := \emptyset$
        \item \textbf{For} $i:= 1$ \textbf{To} $n$
              \begin{innerCodeLine}
                  \item[] Wenn $\omega(v_i)>0$ ist, dann markiere $v_i$ und setze $\omega(u) := \max(\omega(u) - \omega(v_i),0)$ für alle Knoten $u \in N(v_i)$.
              \end{innerCodeLine}
              
        \item \textbf{For} $i:= n$ \textbf{DownTo} $1$
              \begin{innerCodeLine}
                  \item[] Wenn $v_i$ markiert ist, dann setze $\mI := \mI \cup \{v_i\}$ und entferne von allen Knoten $u \in N(v_i)$ die Markierung.
              \end{innerCodeLine}
    \end{codeLine}
\end{Algorithm}

\begin{figure}[htbp]
    %\hspace*{\fill}
    \subfloat[Der Graph~$G$ mit Gewichten, Nachbarschaftsordnung (blaue Pfeile) und den Kanten in $G^2$ (gestrichelt)\label{pic:bsp_Algo_mwIS_a}]
    {\begin{tikzpicture}
        
        \def\len{1.8}
        
        \node[nN,inner sep=2pt] (ll) at (0,0) {2};
        \node[nN,inner sep=2pt] (lt) at (30:\len) {3};
        \node[nN,inner sep=2pt] (lb) at (-30:\len) {3};
        \node[nN,inner sep=2pt] (rt) at ($(lt.center)+(\len,0)$) {3};
        \node[nN,inner sep=2pt] (rb) at ($(lb.center)+(\len,0)$) {3};
        \node[nN,inner sep=2pt] (rr) at ($(rb.center)+(30:\len)$) {2};
        
        %\node (lll) at ($(ll.west)+(-1,0)$) {};
        %\node (rrr) at ($(rr.east)+(1,0)$) {};
            
        \draw[dashed] (ll)--(lt)--(rt) (lb)--(rb)--(rr);
        
        \draw[<-,very thick,clBlue] (ll)--(lb);
        \draw[<-,very thick,clBlue] (lb)--(lt);
        \draw[<-,very thick,clBlue] (lt)--(rb);
        \draw[<-,very thick,clBlue] (rb)--(rt);
        \draw[<-,very thick,clBlue] (rt)--(rr);
    \end{tikzpicture}}
    %\hspace*{\fill}\\
    \hspace*{\fill}
    \subfloat[$G^2$ nach dem Algorithmus mit markierten Knoten (grün), gewählten Knoten (sechseckig) und relevante Gewichte.\label{pic:bsp_Algo_mwIS_b}]
    {\begin{tikzpicture}
        
        \def\len{1.8}
        
        \node[nN,inner sep=2pt] (ll) at (0,0) {0};
        \node[nN,inner sep=2pt,fill=clLight40Green] (lt) at (30:\len) {2};
        \node[nN,inner sep=2pt,fill=clLight40Green,regular polygon,regular polygon sides=6] (lb) at (-30:\len) {1};
        \node[nN,inner sep=2pt,fill=clLight40Green,regular polygon,regular polygon sides=6] (rt) at ($(lt.center)+(\len,0)$) {1};
        \node[nN,inner sep=2pt] (rb) at ($(lb.center)+(\len,0)$) {0};
        \node[nN,inner sep=2pt,fill=clLight40Green] (rr) at ($(rb.center)+(30:\len)$) {2};
            
        \draw[] (ll)--(lt)--(rt) (lb)--(rb)--(rr) (ll)--(lb) (lb)--(lt) (lt)--(rb) (rb)--(rt) (rt)--(rr);

    \end{tikzpicture}}
%    \hspace*{\fill}
%    \subfloat[$G^2$ nach der zweiten Schleife\label{pic:bsp_Algo_mwIS_c}]
%    {\begin{tikzpicture}
%        
%        \def\len{1.8}
%        
%        \node[nN,inner sep=2pt] (ll) at (0,0) {0};
%        \node[nN,inner sep=2pt] (lt) at (30:\len) {2};
%        \node[nN,inner sep=2pt,fill=clLight40Green] (lb) at (-30:\len) {1};
%        \node[nN,inner sep=2pt,fill=clLight40Green] (rt) at ($(lt.center)+(\len,0)$) {1};
%        \node[nN,inner sep=2pt] (rb) at ($(lb.center)+(\len,0)$) {0};
%        \node[nN,inner sep=2pt] (rr) at ($(rb.center)+(30:\len)$) {2};
%            
%        \draw[] (ll)--(lt)--(rt) (lb)--(rb)--(rr) (ll)--(lb) (lb)--(lt) (lt)--(rb) (rb)--(rt) (rt)--(rr);
%    \end{tikzpicture}}
    %\hspace*{\fill}
    \caption[Beispiel für Algorithmus~\ref{algo:mwis}]{Beispiel für Algorithmus~\ref{algo:mwis}. Der Algorithmus wird auf den Graphen~$G^2$ angewendet (blaue und gestrichelte Kanten in \subref{pic:bsp_Algo_mwIS_a}). In \subref{pic:bsp_Algo_mwIS_b} ist $G^2$ nach dem Algorithmus dargestellt. Die Beschriftung der Knoten ist das Gewicht der Knoten, wenn es überprüft wird (negative Gewichte sind als $0$ dargestellt). Vom Algorithmus in der ersten Schleife markierte Knoten sind grün hinterlegt. In der zweiten Schleife gewählte Knoten sind zusätzlich sechseckig. }
    \label{pic:bsp_Algo_mwIS}
\end{figure}


%\vspace*{+10pt}
\subsection{Vermeiden des Quadrats}\label{sec:VermeidenDesQuadrats}
Ein Problem beim Bilden des Quadrats ist es, dass sich damit auch die Anzahl der Kanten quadriert. Gleiches gilt somit auch für die Laufzeit. Ziel dieses Abschnitts ist es nun, auf die Bildung des Quadrats zu verzichten, um eine Laufzeit zu erreichen, die linear zur Größe des Linegraphen ist.

\subsubsection{Reihenfolge der Knoten}
Ein Algorithmus, der auf das Quadrat des Graphen verzichtet, kann durch Modifikation von Algorithmus~\ref{algo:mwis} erreicht werden. Der erste Punkt dabei ist die perfekte Eliminationsordnung. Wie bereits erwähnt, ist hierfür Satz~\ref{theo:dcSquareChordal} hilfreich. Eine maximale Nachbarschaftsordnung für einen dually chordalen Graphen~$G$ ist auch eine perfekte Eliminationsordnung für dessen Quadrat. Sie kann in Linearzeit mit Algorithmus~\ref{algo:mno} ermittelt werden.

\subsubsection{Verrechnung der Gewichte}
Der nächste Punkt ist das Subtrahieren des Gewichts des aktuellen Knotens~$v_i$ von dessen Nachbarn~$N(v_i)$. Das Problem dabei ist, dass die Nachbarschaften in $G$ und $G^2$ unterschiedlich sind. Eine einfache Lösung wäre es, nicht nur die direkten Nachbarn von $v_i$ zu ermitteln sondern auch die Nachbarn der Nachbarn. Dies führt jedoch wieder zu einer quadratischen Laufzeit. Es kann allerdings ausgenutzt werden, dass $v_i$ immer einen maximalen Nachbarn~$m_i$ besitzt.

\begin{Theorem}\label{theo:ijInE_iff_mjInE2}
    Es seien $G=(V,E)$ ein Graph, $G^2=(V,E^2)$ dessen Quadrat und $(v_1,\ldots,v_n)$ eine maximale Nachbarschaftsordnung für $G$, wobei $m_i$ der maximale Nachbar von $v_i$ ist. Außerdem seien $1 \leq i < j \leq n$ und $v_i \neq m_i$. Für alle Knoten $v_j$ gilt nun:
    $$v_iv_j \in E^2 \Leftrightarrow m_iv_j \in E$$
\end{Theorem}

\begin{Proof}
    \prL $v_j$ ist mit dem maximalen Nachbarn von $v_i$ verbunden ($m_iv_j \in E$). Somit ist der Abstand von $v_i$ zu $v_j$ in $G$ höchstens $2$. Also sind $v_i$ und $v_j$ auch in $G^2$ miteinander verbunden ($v_iv_j \in E^2$).
    
    \prR $v_i$ und $v_j$ sind miteinander in $G^2$ verbunden ($v_iv_j \in E^2$). Es kann nun \oBdA angenommen werden, dass $v_iv_j \notin E$ ist. Folglich existiert ein Knoten~$v_k$, der mit $v_i$ und $v_j$ verbunden ist ($v_iv_k,v_kv_j \in E$). Nun gilt es zwei Fälle zu unterscheiden:
    \begin{enumerate}
        \item \label{case:ProofSquare_c1} $i < k$. Per Definition ist der maximale Nachbar $m_i$ mit allen Nachbarn von $v_k$ verbunden. Somit auch mit $v_j$.
        
        \item \label{case:ProofSquare_c2} $k < i$. Da $v_k$ vor $v_i$ in der Nachbarschaftsordnung liegt, besitzt $v_k$ einen maximalen Nachbarn $m_k$, der sowohl mit $v_i$ also auch $v_j$ verbunden ist. Somit kann die Fallunterscheidung für $m_k$ wiederholt werden.
    \end{enumerate}
    
    Per Voraussetzung ist ein Knoten nie sein maximaler Nachbar (außer $v_n$). Daraus folgt, dass sich Fall~\ref{case:ProofSquare_c2} nun so lange wiederholt, bis $i<k$ ist und Fall~\ref{case:ProofSquare_c1} eintritt.
    \qed
\end{Proof}

Anmerkung: Damit Satz~\ref{theo:ijInE_iff_mjInE2} gültig ist, darf (bis auf den letzten Knoten) kein Knoten sein eigener maximaler Nachbar sein. Allerdings erfüllt eine von Algorithmus~\ref{algo:mno} berechnete Nachbarschaftsordnung diese Bedingung.

Satz~\ref{theo:ijInE_iff_mjInE2} erlaubt es nun, Algorithmus~\ref{algo:mwis} so zu ändern, dass die Gewichte auch in Linearzeit korrekt berechnet werden. Dazu erhält jeder Knoten~$v$ neben seinem eigenen Gewicht~$\omega$ ein zusätzliches Gewicht~$\omega_p$, welches mit dem Wert~$0$ initialisiert wird. Dieses Zusatzgewicht dient nun als Zwischenspeicher.

Das Gewicht eines Knotens wird nun nicht mehr direkt abgezogen, sondern erst in dessen maximalen Nachbarn zwischengespeichert. Von dort wird es dann zu einem späteren Zeitpunkt weiterverteilt. Dies stellt sicher, dass über jede Kante höchstens zweimal ein Gewicht "`wandert"' (je einmal $\omega$ und $\omega_p$).

\subsubsection{Sperrung von Nachbarn}
Der letzte Punkt beim Modifizieren von Algorithmus~\ref{algo:mwis} ist das Sperren von Knoten, die zu nahe an einem gewählten Knoten liegen. In Algorithmus~\ref{algo:mwis} werden dazu (in der ersten Schleife) Knoten als Kandidaten markiert. Wird nun in der zweiten Schleife ein Knoten gewählt, werden anschließend die Markierungen sämtlicher Nachbarn entfernt. Sie fallen damit als Kandidaten für die gesuchte Knotenmenge weg.

Auch in diesem Fall würde es zu quadratischer Laufzeit führen, wenn man die Nachbarn der Nachbarn ermittelt. Das Problem lässt sich dadurch lösen, dass man eine zusätzliche Markierung einführt. Für die Markierung eines Knotens~$v$ gibt es dann drei Möglichkeiten: $v$ ist unmarkiert, $v$ ist Kandidat oder $v$ ist gesperrt.

Ein Knoten~$v$ wird nun zur Lösungsmenge hinzugefügt, wenn er als Kandidat markiert ist und weder $v$ noch einer seiner Nachbarn gesperrt wurden. Außerdem werden in diesem Fall alle Nachbarn als gesperrt markiert.

\subsection{Zusammenfassung}
Das oben Beschriebene führt nun zu folgendem Algorithmus, der ein efficient dominating set für dually chordale Graphen berechnet (falls es existiert):

\begin{Algorithm}[dcED]\label{algo:dcED}
    \textbf{Eingabe:} Ein dually chordaler Graph $G=(V,E)$.\\
    \textbf{Ausgabe:} Ein efficient dominating set~$D$ (falls existent).\\

    \begin{codeLine}
        \item $D := \emptyset$
        
        \item \textbf{For All} $v \in V$
              \begin{innerCodeLine}
                  \item[] $\omega(v) := |N[v]|$
                  \item[] $\omega_p(v) := 0$
              \end{innerCodeLine}
        
        \item Ermittle eine maximale Nachbarschaftsordnung $(v_1,\ldots,v_n)$ sowie die dazugehörigen maximalen Nachbarn $(m_1,\ldots,m_n)$ mittels Algorithmus~\ref{algo:mno}.
        
        \item \textbf{For} $i:= 1$ \textbf{To} $n$
              \begin{innerCodeLine}
                  \item Ziehe das Gewicht~$\omega_p(u)$ aller Knoten $u \in N[v_i]$ vom Gewicht~$\omega(v_i)$ ab ($\forall\,u \in N(v_i): \omega(v_i):=\omega(v_i)-\omega_p(u))$
                  
                  \item Wenn $\omega(v_i)>0$ ist, dann markiere $v_i$ als Kandidat und setze $\omega_p(m_i) := \omega_p(m_i) + \omega(v_i)$.
              \end{innerCodeLine}
              
        \item \textbf{For} $i:= n$ \textbf{DownTo} $1$\nopagebreak
              \begin{innerCodeLine}
                  \item[] Wenn $v_i$ als Kandidat markiert ist und kein Knoten $u \in N[v_i]$ gesperrt ist, dann setze $D := D \cup \{v_i\}$ und sperre alle Knoten $u \in N(v_i)$.
              \end{innerCodeLine}
              
        \item  $D$ ist ein efficient dominating set genau dann, wenn $\sum_{v \in D}|N[v]|=|V|$.
    \end{codeLine}
\end{Algorithm}

\begin{Theorem}\label{theo:dcED}
    Algorithmus~\ref{algo:dcED} findet ein efficient dominating set für dually chordale Graphen in Linearzeit.
\end{Theorem}

Die Korrektheit ergibt sich aus der oberen Argumentation (besonders Satz~\ref{theo:ijInE_iff_mjInE2}). Daher wird an dieser Stelle auf einen Beweis verzichtet.

\subsection{Der gewichtete Fall}
Mit Algorithmus~\ref{algo:dcED} ist es nun möglich, ein efficient dominating set für einen dually chordalen Graphen zu finden. Somit auch ein dominating induced Matching für $\alpha$-azyklische Hypergraphen. Das Problem ist damit jedoch nur für den ungewichteten Fall gelöst.

Angenommen, für die Kanten eines Hypergraphen bzw. die Knoten des Linegraphen sei auch eine Gewichtsfunktion~$\alpha$ gegeben. Es soll nun ein dominating induced Matching mit maximalem Gewicht ermittelt werden. Da für das ursprüngliche Problem bereits eine Knotenmenge mit maximalem Gewicht gesucht wird, müssen die Gewichte nur angepasst werden. Nachfolgend werden zwei Varianten dafür vorgestellt.

\subsubsection{Multiplikator-Variante}
Eine Möglichkeit, um sowohl die Anzahl der Nachbarn als auch das gegebene Gewicht einer Hyperkante (bzw. Knotens~$v$ im Linegraphen) zu berücksichtigen, ist es, die Summe von beiden zu bilden. Dabei wird die Anzahl der Nachbarn allerdings vorher mit einem hinreichend großen Multiplikator~$m$ verrechnet.
\[ \omega(v) = m \cdot N[v] + \alpha(v) \]

Die Variante hat zwei Nachteile. Zum einen muss ein Multiplikator gewählt werden, der groß genug ist, damit die Anzahl der Nachbarn die dominante Größe bleibt. Zum anderen können negative Gewichte dazu führen, dass sich am Gesamtgewicht der gefundenen Knotenmenge  nicht mehr direkt ablesen lässt, ob sie dominierend ist.

\subsubsection{Vektor-Variante}
Alternativ zu einem Gesamtgewicht, kann man die Gewichte als Vektoren betrachten. Ein solches Vektorgewicht besteht aus dem ursprünglichen Gewicht und der Anzahl der Nachbarn. 
\[ \omega(v) =\binom{N[v]}{\alpha(v)} \]

Allerdings ist es notwendig, zu definieren, wann ein Vektor größer ist als ein anderer.
    \begin{align*}
        \binom{a}{b} = \binom{x}{y} &\Leftrightarrow a=x \land b =y \\
        \binom{a}{b} > \binom{x}{y} &\Leftrightarrow a>x \lor (a=x \land b>y)
    \end{align*}

Der Vorteil besteht darin, dass die Anzahl der Nachbarn immer vom ursprünglichen Gewicht getrennt ist. Somit muss kein Multiplikator errechnet werden. Außerdem lässt sich am Gesamtgewicht sofort ablesen, ob es sich um eine dominierende Menge handelt.
\[ D \subseteq V \text{ ist ein efficient dominating set } \Leftrightarrow \sum_{d \in D} \omega(d) > \binom{|V|}{-\infty} \]

\chapter{Abschluss}

Diese Arbeit untersuchte das Dominating Induced Matching Problem. Dabei stellte sich heraus, dass es sich (unabhängig von der Graphenklasse) auf die gewichtete Suche nach einer dominierenden Menge im Linegraphen bzw. auf die Suche nach einer gewichteten unabhängigen Menge im Quadrat des Linegraphen reduzieren lässt (Satz~\ref{theo:EEDiffEDiffMWDiffMWIS}, S.~\pageref{theo:EEDiffEDiffMWDiffMWIS}).

Darauf basierend konnte gezeigt werden, dass sich das Problem für die Klasse der $\alpha$-azyklischen Hypergraphen in Polynomialzeit lösen lässt. Dazu wurde ein Algorithmus entwickelt, der in Linearzeit ein efficient dominating set für dually chordale Graphen findet (Algorithmus~\ref{algo:dcED}, S.~\pageref{algo:dcED}), welches die Linegraphen der $\alpha$-azyklischen Hypergraphen sind (Satz~\ref{theo:AlphaAzyklDuallyChordal}, S.~\pageref{theo:AlphaAzyklDuallyChordal}).

Offen bleibt die Frage, ob es in Linearzeit möglich ist, ein dominating induced Matching für $\alpha$-azyklischen Hypergraphen zu finden. Die im vorherigen Kapitel vorgestellte Lösung kann quadratische Laufzeit erreichen, da der Linegraph des Hypergraphen gebildet werden muss. Für eine Lösung mit linearem Aufwand muss das vermieden werden.

Da Satz~\ref{theo:EEDiffEDiffMWDiffMWIS} unabhängig von einer Graphenklasse ist, stellen sich außerdem die Fragen: Gibt es weitere Graphen- oder Hypergraphenklassen, für die Satz~\ref{theo:EEDiffEDiffMWDiffMWIS} zu einer Lösung führt? Gibt es Klassen, für welche die Suche nach einer dominierenden Menge im Linegraphen eine bessere Lösung bringt als die Suche nach einer unabhängigen Menge im Quadrat des Linegraphen?

%\chapter{Prädikatenlogik}

bla bla

\section{Logik erster Stufe}\index{FOL|see{Prädikatenlogik erster Stufe}}\index{Prädikatenlogik!erster Stufe}

bla bla

Die Definitionen in diesem Abschnit basieren auf den entsprechenden Definitionen in \cite{kreuzer2006logik} und \cite{wikiD:FOL}.

\subsection{Syntax}
Die Syntax der Prädikatenlogik unterteilt sich in drei Ebenen: Symbole, Terme und Formeln. Dabei bilden Symbole die unterste Ebene und Formeln die höchste.

\subsubsection{Symbole}
Symbole sind die Elemente für jede prädikatenlogische Aussage. Sie teilen sich in fünf Teilmengen ein: Quantoren, Variablen, Funktionssymbole, Prädikatensymbole und logische Symbole.

\begin{mydef}[Symbole]
    Die Symbole der Prädikatenlogik ergeben sich aus den paarweise disjunkten Mengen
   \begin{itemize}
        \item der Variablen $\mV$,
        \item der Funktionssymbole $\mF$,
        \item der Prädikatensymbole $\mP$,
        \item der Quantoren $\{\forall,\exists\}$ und
        \item der logischen Symbole $\{\land,\lor,\lnot\}$.
    \end{itemize}
\end{mydef}

\subsubsection{Terme}
Terme werden aus Variablen und Funktionssymbolen oder bereits vorhanden Termen gebildet.
\begin{mydef}[Terme]
    Es seien $v$ eine Variable, $f$ ein Funktionssymbol und $t_1,\ldots,t_k$ ($k\geq 1$) Terme. Dann gilt:
    \begin{itemize}
       \item  $v$ und $f$ sind Terme.
       \item  $f(t_1,\ldots,t_k)$ ist ein Term.
    \end{itemize}
    Im ersten Fall sagt man, dass $f$ eine konstante (oder $0$-stellige) Funktion ist. Im zweiten Fall spricht man von einer $k$-stelligen Funktion.
\end{mydef}

\subsubsection{Formeln}
Aus Symbolen und Termen lassen sich nun Formeln bilden. Genau wie bei Termen erfolgt dies wieder induktiv.
\begin{mydef}[Formeln]
    Es seien $v$ eine Variable, $P$ ein Prädikatensymbol, $Q$ ein Quantor, $t_1,\ldots,t_k$ ($k\geq 1$) Terme und $F$ und $G$ Formeln. Dann sind die folgenden Ausdrücke Formeln:
    \begin{itemize}
       \item  $P(t_1,\ldots,t_k)$
       \item  $Qv\ F$
       \item  $\lnot\,F$, $(F\land G)$, $(F\lor G)$
    \end{itemize}
    Im ersten Fall sagt man, dass $P$ ein $k$-stelliges Prädikat ist. Prädikate können nicht $0$-stellig sein.
\end{mydef}

\subsubsection{Anmerkungen}
In dieser Arbeit wird davon ausgegangen, dass die Stelligkeit von Prädikaten- und Funktionssymbolen eindeutig ist. Im Allgemeinen ist das aber nicht nötig. Hat ein Prädikaten- oder Funktionssymbol $S$ verschiedene Stelligkeiten (z. B. $j$ und $k$), kann es durch die Symbole $S_j$ und $S_k$ ersetzt werden. Aufgrund der Syntax ist die Ersetzung für jedes Vorkommen von $S$ eindeutig.
\todo{Überprüfen, ob nicht doch irgendwo in der Arbeit mehrdeutig}

Folgende Notation wird in dieser Arbeit verwendet:
\begin{itemize}
    \item  kleine lateinische Buchstaben für Variablen und $k$-stellige Funktionen ($k\geq 1$)
    \item  kleine grichische Buchstaben für konstante Funktionen
    \item  große lateinische Buchstaben für Prädikate
\end{itemize}

\todo{Werden Konstanten überhaubt benötigt?}

\subsection{Semantik}

\begin{mydef}[Struktur]
    Es sei $F$ eine Formel in FOL. Des Weiteren sei $P$ ein $k$-stelliges Prädikatensymbol, $f$ ein $k$-stelliges Funktionssymbol und $v$ eine Variable in $F$.
    
    Eine zu $F$ passende Struktur $\mS$ ist ein 4-Tupel $\mS=(\mU,\varphi,\psi,\chi)$ aus einer nicht leeren Menge $\mU$ (\emph{Universum} \index{Universum} genannt) und den Abbildungen $\varphi$, $\psi$ und $\chi$. Dabei gilt:
    \begin{itemize}
        \item  $\varphi(f):\mU^k\rightarrow\mU$
        \item  $\psi(P)\subseteq\mU^k$
        \item  $\chi(v)\in\mU$
    \end{itemize}
\end{mydef}

\section{Logik zweiter Stufe}\index{SOL}\index{Prädikatenlogik!zweiter Stufe}
Quantoren über prädikate

\begin{mydef}[Formeln in SOL]
    Es seien $v$ eine Variable, $P$ ein Prädikatensymbol, $Q$ ein Quantor, $t_1,\ldots,t_k$ ($k\geq 1$) Terme und $F$ und $G$ Formeln. Dann sind die folgenden Ausdrücke Formeln:
    \begin{itemize}
       \item  $P(t_1,\ldots,t_k)$
       \item  $Qv\ F$ und $QP\ F$
       \item  $\lnot\,F$, $(F\land G)$ und $(F\lor G)$
    \end{itemize}
\end{mydef}

\url{http://www2.informatik.hu-berlin.de/logik/lehre/WS07-08/Logik/HU/kap8.pdf}
@book\{gurski2010exakte,\\
  title=\{Exakte Algorithmen für schwere Graphenprobleme\},\\
  author=\{Gurski, F. and Rothe, I. and Rothe, J. and Wanke, E.\},\\
  isbn=\{978-3-642-04499-1\},\\
  year=\{2010\},\\
  publisher=\{Springer Berlin Heidelberg\}\\
\}

\section{Monadische Logik}
Nur einstellige Prädikate

\section{Anwendung bei Graphen}

%\part{Nicht in der endgültigen Arbeit}

\chapter{Seitenränder}
\url{http://www.latex-project.org/guides/lb2-ch4.pdf}  aus "`Der \LaTeX-Begleiter"' (978-3827370440 -- \url{http://www.amazon.de/LaTeX-Begleiter-Michel-Goossens/dp/3827370442})

\the\oddsidemargin \\
\the\evensidemargin \\
\the\hoffset \\
\the\footskip


\chapter{Sonstiges}

\todo{Dieses Kapitel dient nur dazu unsortierte Textabschnitte aufzunehmen. Es wird nicht Teil der Arbeit.}

\cite{brandstaedt1999graph}
\cite{Cardoso2011}
\cite{ClWidthSOL}
\rule{\linewidth}{1pt}
\clearpage

\section{Pflichtkanten}
\todo{Eventuell andere Übersetzung oder engl Begriff}

Für einen Graphen ist ein dim nicht immer eindeutig. Es kann mehrere mögliche Varianten in einem Graphen geben. Es ist jedoch so, dass in einigen Fällen auch bei verschiedenen dims, einzelne Kanten zwingend in jedem dim des Graphen enthalten sein müssen.

\begin{mydef}[Pflichtkante]
Gegeben sei ein Graph $G=(V,E)$. Eine Kante $m\in E$ ist eine \emph{Pflichtkante} (engl.: mandatory edge), wenn für alle dims $M$ von $G$ gilt: $m\in M$.
\end{mydef}

In einem Diamond $D$ ist beispielsweise die mittlere Kante eine Pflichtkante. Wählt man eine andere Kante (am Rand), so ist die gegenüberliegende Kante zu nahe an der gewählten um zum dim zu gehören. Wenn $D$ Teilgraph eines größeren Graphen ist, bliebe noch die Möglichkeit, dass die gegenüberliegende Kante mit einer Kante des dims verbunden ist. Auch das ist nicht möglich, da diese Kanten ebenfalls zu nahe wären.

\todo{alternativ über Knotenklassen beschrieben, oder vorher zeigen, dass in einem Dreieck eine Kante enthalten sein muss.}

\begin{figure}[htb]
\centering
\hspace*{\fill}
\subfloat[Diamond]{\begin{tikzpicture}
  [thick,
   normalN/.style={circle,draw,minimum size=0.25cm,inner sep=0pt,fill=white},
   colE/.style={draw,-,line width=0.075cm}
  ]

  \foreach \x/\y in {0/a,90/b,180/c,-90/d} {
     \node[normalN] (\y) at (\x:0.75cm) {};
  }

  \node (e) at (0:1.5cm) {};
  
    \begin{pgfonlayer}{background}
        \draw [colE,blue] (c.center) -- (b.center);
        
        \draw [colE,darkgreen] (a.center) -- (b.center);
        \draw [colE,darkgreen] (a.center) -- (c.center);
        \draw [colE,darkgreen] (c.center) -- (d.center);
        
        \draw [colE,red] (a.center) -- (d.center);

        \draw [colE,red, dashed] (a.center) -- (e.center);
    \end{pgfonlayer}
   
 
\end{tikzpicture}}
\hspace*{\fill}
\subfloat[Gem]{\begin{tikzpicture}
  [thick,
   normalN/.style={circle,draw,minimum size=0.25cm,inner sep=0pt,fill=white}
  ]

  \foreach \x/\y in {0/a,90/b,180/c,-90/d} {
     \node[normalN] (\y) at (\x:0.75cm) {};
  }
  
  \foreach \x/\y in {a/b,b/c,c/d,d/a,a/c} {
      \draw (\x) -- (\y);
  }

 
\end{tikzpicture}}
\hspace*{\fill}
\caption{TODO}
\label{pic:bsp_mandEdge}
\end{figure}


\rule{\linewidth}{1pt}
\clearpage

\begin{Lemma}\label{lem:simplClique}Es sei $H$ ein Hypergraph mit $\mL=L(H)$ und einem dim $M$. Für jede Clique $K$ in $\mL$ mit einem simplizialen Knoten $s$ gilt:
\[\exists!\,v\in K:v\in M\]
\end{Lemma}

\begin{Proof}
Angenommen es gehört kein Knoten aus $K$ zu $M$. Dann ist $s$ nicht mit einem Knoten aus $M$ benachbart. Somit ist $M$ auch kein dim von $H$. Folglich gibt es einen Knoten $v\in K$.

Offensichtlich ist es genau ein Knoten, da in einer Clique von $\mL$ höchstens ein Knoten zu $M$ gehören kann. Andernfalls wären zwei Knoten aus $M$ miteinander benachbart.
\qed
\end{Proof}

\rule{\linewidth}{1pt}
\clearpage

\begin{Lemma}
Es sei $H$ ein Hypergraph und $\mL$ dessen Linegraph. Des weiteren sei $N_B(v)=\{b\ |\ b\in N(v)\wedge b\text{ ist Blatt}\}$ die Menge der benachbarten Blätter eines Knotens $v$ aus $\mL$.

Wenn $\left|N_B(v)\right|\geq2$, dann ist $v$ eine Pflichtkante in $H$.
\end{Lemma}

Ergibt sich aus Lemma \ref{lem:simplClique}

\rule{\linewidth}{1pt}
\clearpage

\section{Bereinigung von Zwillingen}

\begin{mydef}[Nachbarschaft eines Knotens]
    Gegeben sei ein Graph $G=(V,E)$. Die \emph{offene} Nachbarschaft $N(v)$ bzw. die \emph{abgeschlossene} Nachbarschaft $N[v]$ von $v$ seien dann wie folgt definiert:
    \begin{align*}
        N(v)&:=\{u\ |\ uv \in E\} \\
        N[v]&:=N(v) \cup \{v\}
    \end{align*}
\end{mydef}

\begin{mydef}[Zwillinge]
    In einem Graphen werden zwei Knoten $u$ und $v$ ($u\neq v$) als (echte) Zwillinge bezeichnet (engl. true twins), wenn $N[u]=N[v]$.
\end{mydef}

Bei Zwillingen gilt es zu beachten, dass es sich um eine abgeschlossene Nachbarschaft handelt. Somit müssen zwei Knoten $u$ und $v$ nicht nur die selben Nachbarn haben, sondern auch selbst benachbart sein. Es muss also eine Kante $uv$ vorhanden sein.

\subsection{Die Funktion $t()$}
\todo{Funktion besser beschreiben}
Gegeben ein Graph $G=(V,E)$.
\begin{description}
\item[Schritt 1] Für alle $v\in V$: Wenn $v$ einen Zwilling $u$ besitzt, dann $V:=V\,\backslash\,\{v\}$

\item[Stopp]
\end{description}

Der so erzeugte Graph $t(G)$ sei der um Zwillinge bereinigte Graph von $G$.

\begin{Lemma}
    Ein Graph $G$ hat ein vDim genau dann, wenn $t(G)$ ein vDim hat.
\end{Lemma}
\todo{Eventuell Abb. für Beweis}
\begin{Proof}
	Es seien $u$ und $v$ Zwillinge in $G$, wobei $u$ kein Knoten in $t(G)$ ist.
	
    \prR $G$ habe ein vDim $D$. Es gibt nun zwei mögliche Fälle:
    \begin{enumerate}
        \item $u$ gehört nicht zu $D$. Durch das entfernen von $u$ bleibt $D$ ein gültiges vDim in $t(G)$.
        \item $u$ gehört zu $D$. In diesem Fall ist $(D\,\backslash\,\{u\})\cup\{v\}$ ein gültiges vDim in $t(G)$. Da $u$ und $v$ Zwillinge sind, werden nun alle Knoten durch $v$ gematcht, die vorher durch $u$ gematcht wurden.
    \end{enumerate}

    \prL $t(G)$ habe ein vDim $D$. Es sei $d$ der Knoten, durch den $v$ gematcht wird. $d$ matcht aufgrund der identischen Nachbarschaft auch $u$ in $G$. Somit ist $D$ auch ein gültiges vDim in $G$.
    \qed
\end{Proof}

Somit kann o. B. d. A. angenommen werden, dass ein Linegraph, in dem ein vDim gesucht wird, keine Zwillinge mehr besitzt.

\rule{\linewidth}{1pt}
\clearpage


\section{Vereinfachung eines Hypergraphen}

\todo{Darstellung für Algorithmen}

Es sein $E(v):=\{e\in \mE \ |\ v\in e\}$ die Menge der Hyperkanten, in denen der Knoten $v$ ist.

\subsection{Die Funktion $m(H)$}
\todo{eventuell besser formulieren}
Gegeben ein Hypergraph $H=(V,\mE)$.
\begin{description}
\item[Schritt 1] Für alle $v\in V$: Wenn $E(v)=\{e\}$ und $|e|>1$, dann $H:=H\backslash\{v\}$.

\item[Schritt 2] Für alle $v\in V$: Wenn $\exists\,u:u\neq v$ und $E(u)=E(v)$, dann $H:=H\backslash\{v\}$.

\item[Stopp]
\end{description}

Der so erzeugte Graph sei $m(H)$.

\subsection{Beibehalten relevanter Eigenschaften}
\subsubsection{Beibehalten der Dominating Induced Matchings}
\begin{Theorem}
$H$ hat dim $\Leftrightarrow$ $m(H)$ hat dim.
\end{Theorem}

\begin{Proof}
Es werden weder Kanten entfernt, noch hinzugefügt. Auch die Nachbarschaft der Kanten wird nicht verändert. \qed
\end{Proof}

\subsubsection{Beibehalten der Chordalität}
\todo{Gibts das Wort Chordalität? -- Eventuell andere Formulierung}

\begin{Lemma}\label{lem:JedeKanteAndereHyperkante}
Es sei $C$ ein $C_k$ ($k\geq 4$) und induzierter Teilgraph von $2Sec(H)$, dann gilt: Jede Kante aus $C$ gehört zu einer anderen Hyperkante.
\end{Lemma}

\begin{Proof}
Angenommen zwei Kanten $e_1=u_1v_1$ und $e_2=u_2v_2$ von $C$ mit $u_1\neq v_2$ gehören zur gleichen Hyperkante von $H$. Dann ist $u_1v_2$ eine Kante in $2Sec(H)$. Somit ist $C$ kein induzierter Teilgraph von $2Sec(H)$. \qed
\end{Proof}

\begin{Theorem}\label{theo:ChordalM}
$2Sec(H)$ ist chordal $\Leftrightarrow$ $2Sec(m(H))$ ist chordal.
\end{Theorem}

\todo{Alternativ für $\Rightarrow$ zeigen, dass: $\forall\,u,v \in m(H)$ gilt: $uv\in 2Sec(H) \Leftrightarrow uv\in 2Sec(m(H))$}

\begin{Proof} Es sei $k\geq4$. Index Arithmetik modulo $k$.

\prR Angenommen, $2Sec(m(H))$ ist nicht chordal. Dann existiert in $2Sec(m(H))$ ein Kreis $C$ der Länge $k$ mit den Knoten $v_0,\ldots,v_{k-1}$. 
Aufgrund der Arbeitsweise von $m()$ und der Definition von 2-Section-Graphen sind alle Knoten von $C$ auch Knoten in $2Sec(H)$. Somit wurde entweder eine Kante $v_iv_j$ ($i+1=j$) hinzugefügt, oder eine Kante $v_iv_j$ ($i+1\neq j$) wurde entfernt.
Dies bedeutet, dass es in $m(H)$ (bzw. $H$) eine Hyperkante gibt, die $v_i$ und $v_j$ enthält, jedoch nicht in $H$ (bzw. $m(H)$). Dies steht im Widerspruch zur Arbeitsweise von $m()$.

\prL Es sei $2Sec(H)$ nicht chordal. Es ist nun zu zeigen, dass $2Sec(m(H))$ nicht chordal ist.

Da $2Sec(H)$ nicht chordal ist, besitzt er einen $C_k$. Aus Lemma \ref{lem:JedeKanteAndereHyperkante} folgt, dass es in $H$ einen $\mathcal{C}_k$  gibt mit den Kanten $e_0,\ldots,e_{k-1}$ und den Knoten $v_0,\ldots,v_{k-1}$ mit $v_i \in e_i\cap e_{i+1}$.

Wird kein $v_i$ durch $m()$ entfernt, so bleiben alle Kanten $v_iv_{i+1}$ enthalten und somit auch der $C_k$ in $2Sec(m(H))$. Somit ist $2Sec(m(H))$ nicht chordal.

Wird $v_i$ entfernt, dann folgt aus der Arbeitsweise von $m()$, dass ein Knoten $v^*$ existiert mit $v^* \in e_i\cap e_{i+1}$. 
%Würde $v^*$ nicht existieren, könnte $v_i$ nicht entfernt werden. 
Somit sind $v_{i-1}v^*$ und $v^*v_{i+1}$ Kanten in $2Sec(m(H))$, wodurch die Knoten $v_0,\ldots,v_{i-1},v^*,v_{i+1},\ldots,v_{k-1}$ einen $C_k$ in $2Sec(m(H))$ bilden. Folglich ist $2Sec(m(H))$ nicht chordal. \qed
\end{Proof}

\tikzfading[name=fade right,
left color=transparent!0,
right color=transparent!100]
\floatimage{pic:ProofChordalM}{Beweisskitze zum Beweis ($\Leftarrow$) von Satz \ref{theo:ChordalM}}{\begin{tikzpicture}
  [thick,node distance=2.5cm,lbl/.style={below}]
  
  \node[hN] (center) at (0,0) {};

  \node[nN,dotted] (vi) at (0,.5) {}; \node[lbl] (lblvi) at (vi.south) {$v_i$};
  \node[nN] (vs) at (0,-.5) {}; \node[lbl] (lblvs) at (vs.south) {$v^*$};
  
  \node[nN] (vim) [left=of center] {};
  \node[lbl] (lblvim) at (vim.south) {$v_{i-1}$};
  
  \node[nN] (vip) [right=of center] {};
  \node[lbl] (lblvip) at (vip.south) {$v_{i+1}$};
    
  \node[hN] (vimm) [left=of vim] {};
  %\node[lbl] (lblvimm) at (vimm.south) {$v_{i-2}$};
  
  \node[hN] (vipp) [right=of vip] {};
  %\node[lbl] (lblvipp) at (vipp.south) {$v_{i+2}$};
  
  \begin{pgfonlayer}{background}
      \node [fill=clRed,ellipse,fill opacity=.5] (ei)
            [fit=(vi) (vs) (vim) (lblvi) (lblvs) (lblvim)] {};
      \node [lbl] at (ei.north) {$e_i$};
      
      %\node [fill=clBlue,ellipse,fill opacity=.5]
            [fit=(vip) (vipp) (lblvip)] {};
            
      \node [fill=clGreen,ellipse,fill opacity=.5] (eip)
            [fit=(vi) (vs) (vip) (lblvi) (lblvs) (lblvip)] {};
      \node [lbl] at (eip.north) {$e_{i+1}$};
                        
  \end{pgfonlayer}
  
  \draw[dashed] (vim) -- (vi) (vi) -- (vip);
  \draw (vim) -- (vs) (vs) -- (vip);
  
  \path[scope fading=east] (vip.north east) rectangle (vipp.south west);
  \draw [] (vip) -- (vipp);

  \path[scope fading=west] (vim.north west) rectangle (vimm.south east);
  \draw [] (vimm) -- (vim);
    
  
\end{tikzpicture}}

\subsubsection{conformal}
\todo{Überschrift}


\begin{Theorem}\label{theo:ConformalM}
$H$ ist conformal $\Leftrightarrow$ $m(H)$ ist conformal
\end{Theorem}

\begin{Proof}
Es seien $e_1,e_2,e_3,e$ Kanten von $H$, die sich entsprechend Satz~\ref{theo:GilmoreTheorem} zueinander verhalten mit $e_{ij}:=e_i\cap e_j$. Zusätzlich sei $v\in V$ ein Knoten von $H$, der durch $m()$ entfernt wird. O. B. d. A. sei $v \in e_{12}$ und $v\in e \Leftrightarrow$ $H$ ist conformal.

\prR
Offensichtlich gilt:
$(e_{12} \cup e_{13} \cup e_{23})\backslash\{v\} \subseteq e\backslash\{v\}$

\prL
$H$ sei nicht conformal. Somit gilt: $v\in e_{12} \cup e_{13} \cup e_{23} $ und $v\notin e$. Da $v$ entfernt wird, existiert ein weiterer Knoten $v^*$ mit $E(v)=E(v^*)$. Es ist somit $v^*\in e_{12}$, jedoch $v^*\notin e$. Also gilt:
$e_{12} \cup e_{13} \cup e_{23}\nsubseteq e$\qed
\end{Proof}

Aus Satz \ref{theo:ChordalM} und \ref{theo:ConformalM} folgt nun unmittelbar:

\begin{Theorem}Ein Hypergraph $H$ ist $\alpha$-azyklisch $\Leftrightarrow$ $m(H)$ ist $\alpha$-azyklisch.
\end{Theorem}

Das Anwenden von $m()$ hat somit weder Auswirkungen auf die dims des Hypergraphen noch darauf, ob der Hypergraph $\alpha$-azyklisch ist.


\rule{\linewidth}{1pt}
\clearpage


\subsection{Hierarchische Linegraphen}
Bei Hypergraphen kann es (im Gegensatz zu normalen Graphen) den Fall geben, dass eine Kante $e$ vollständig in einer anderen Kante $f$ enthalten ist ($e\subseteq f$). Daraus folgt eine Hierarchie innerhalb der Kanten, die sich auch in einem abgeänderten Linegraph abbilden lässt.

\begin{mydef}[hierarchischer Linegraph]
Gegeben sei ein Hypergraph $H=(V,\mE)$. Der hierarchische Linegraph $L^*(H)=(\mE,E_l,E_d)$ ist dann wie folgt definiert:
\begin{align*}
E_l &= \{ef\ |\ e\neq f\in\mE  \wedge e\cap f\neq\emptyset \} \\
E_d &= \{(e,f)\ |\ e,f\in\mE  \wedge e \subseteq f \}
\end{align*}
\end{mydef}

Prinzipiell könnte für einen Hypergraphen der Fall auftreten, dass zwei (oder mehr) Kanten $e$ und $f$ die gleichen Knoten haben, also $e\subseteq f$ und $f\subseteq e$. Da die Kantenmenge eines Hypergraphen nicht als Multimenge definiert wurden, sind dann $e$ und $f$ streng genommen ein und die selbe Kante. Jedoch kann es erst durch die Minimierung eines Hypergraphen dazu gekommen sein. Eventuell waren $e$ und $f$ ursprünglich unterschiedliche Kanten. In diesem Fall können $e$ und $f$ auch als eine Kante betrachtet werden. Auf die Suche nach einem dim hat dies keine Auswirkungen, da die Nachbarschaft von beiden Kanten identisch ist. Gehört eine der Kanten zum Matching, ist es egal welche. Gehört keine der Kanten zum Matching ist die Entfernung von allen zur nächsten Kante gleich. Somit kann davon ausgegangen werden, dass es keine verschiedenen Kanten $e$ und $f$ gibt, die die gleiche Knotenmenge besitzen.

Offensichtlich ist der Graph $L^*_d(H)=(\mE,E_d)$ ein gerichteter azyklischer Graph 
(engl. directed acyclic graph; kurz DAG). Gäbe es einen Zyklus, wären die darin enthaltenen Knoten jeweils Kanten von $H$, die ineinander enthalten sind. Diese werden dann jedoch als eine Kante und somit nur ein Knoten betrachtet.

\rule{\linewidth}{1pt}
\clearpage

\begin{Theorem}
    Es sei $H$ ein $\beta$-azyklischer Hypergraphen und $\mL=L(H)$ dessen Linegraph. Dann gilt: Für jede (maximale) Clique in $\mL$ haben die entsprechenden Kanten einen gemeinsamen Knoten in $H$.
\end{Theorem}
\todo{Formatierung \textbackslash paragraph\{\}}
\todo{Definition $\beta$-azyklischer Hypergraph}

\begin{Proof}BlaBla
	\paragraph{$n=2$:} Erfüllt aufgrund der Definition für Linegraphen (Definition \ref{def:Linegraph}).
	\paragraph{$n=n+1$:} Es seine $K$ eine Menge von Hyperkanten in $H$, die eine Clique in $\mL$ bilden und einen gemeinsamen Knoten $v$ haben. Außerdem sei $e$ eine Hyperkante, die mit allen Kanten aus $K$ benachbart ist. Somit ist $K^*=K\cup \{e\}$ eine Clique in $\mL$.
	
	Angenommen, es gibt keinen gemeinsammen Knoten aller Kanten in $K^*$. Dann ist $v$ kein Knoten von $e$. Des Weiteren existieren zwei Kanten $a,b \in K$, die jeweils einen gemeinsamen Knoten mit $e$ haben. 
	
	\todo{Abbildung}
	
	Aufgrund von Lemma TODO (Gilmore Theorem für $\beta$-azkl. HyGr) kann $H$ kein $\beta$-azyklischer Hypergraph sein, da $a$, $b$ und $e$ Lemma TODO nicht erfüllen. Dies steht im Widerspruch zur Annahme.
\qed	
\end{Proof}

\rule{\linewidth}{1pt}
\clearpage

\begin{Lemma}\label{lem:LineConformalS3}
Es existiert kein conformaler Hypergraph, dessen Linegraph ein $S_3$ ist.
\end{Lemma}

\begin{Proof}
Gegeben sei ein $S_3$ $\mS$ mit den Knoten $a$, $b$, $c$, $d$, $e$ und $f$. $\mS$ besitzt die Cliquen $K_1=\{a,b,c\}$, $K_2=\{b,c,e\}$, $K_3=\{b,d,e\}$, und $K_4=\{c,e,f\}$. Des weiteren sei $uv:=u\cap v$.

\floatimage{pic:ProofLineConformalS3}{Der Grpah $\mS$}
{\begin{tikzpicture}
  [thick,
   lbl/.style={font=\small}]
   
  \node [nN] (a) at ($2*(60:1.5)$)       {}; \node[lbl,anchor=-90] at (a.90)  {$a$};
  \node [nN] (b) at (60:1.5)             {}; \node[lbl,anchor=-30] at (b.150) {$b$};
  \node [nN] (c) at ($(60:1.5)+(0:1.5)$) {}; \node[lbl,anchor=210] at (c.30)  {$c$};
  \node [nN] (d) at (0,0)                {}; \node[lbl,anchor=30]  at (d.210) {$d$};
  \node [nN] (e) at (0:1.5)              {}; \node[lbl,anchor=90]  at (e.-90) {$e$};
  \node [nN] (f) at (0:3)                {}; \node[lbl,anchor=150] at (f.-30) {$f$};

  \draw (a)--(b)--(d)--(e)--(b)--(c)--(e)--(f)--(c)--(a);
  
  \node [lbl] at ($(60:1.5)+(30:0.866)$) {$K_1$};
  \node [lbl] at ($2*(30:0.866)$)        {$K_2$};
  \node [lbl] at (30:0.866)              {$K_3$};
  \node [lbl] at ($(0:1.5)+(30:0.866)$)  {$K_4$};
  
\end{tikzpicture}}

Angenommen es existiert ein conformaler Hypergraph $H$, dessen Linegraph $\mS$ ist. Dann existiert aufgrund der Conformalität (Definition~\ref{def:conformal}) für jede Clique $K_i=\{u,v,w\}$ in $\mS$ eine Kante $\varepsilon_i$ in $H$, so dass $\varepsilon_i \supseteq uv \cup uw \cup vw$. Für alle vier Cliquen von $\mS$ muss es sich bei $\varepsilon_i$ um einen Knoten der Clique handeln, da $\varepsilon_i$ sonst in $H$ gemeinsame Knoten mit $u$, $v$ und $w$ hätte und somit $\{\varepsilon_i,u,v,w\}$ eine Clique in $\mS$ bilden würde. Es ist jedoch keine Clique dieser Größe in $\mS$ vorhanden.

Aufgrund der Symmetrie von $\mS$ kann \oBdA angenommen werden, dass $\varepsilon_2=b$ ist. Somit gilt:
\[bc \cup be \cup ce \subseteq b\]

Für $K_4$ gilt somit: $\varepsilon_4\neq f$. Andernfalls wäre $ce \subseteq f$ und somit $bf\neq\emptyset$. Dann müsste es in $\mS$ eine Kante zwischen $b$ und $f$ geben. 

Es sei nun $\varepsilon_4=c$ ($\varepsilon_4=e$ würde sich aufgrund der Symmetrie analog verhalten). Somit gilt:
\[ce \cup cf \cup ef \subseteq c\]

Folglich ist die Schnittmenge aus $c$, $e$ und $f$ nicht leer ($cef:=c \cap e \cap f$, $cef \neq \emptyset$). Da $cef$ eine Teilmenge von $ce$ und $ce$ eine Teilmenge von $b$ ist ($cef \subseteq ce \subseteq b$), kann die Schnittmenge von $b$ und $f$ nicht leer sein.
Somit muss es entweder eine Kante zwischen $b$ und $f$ in $\mS$ geben, oder $H$ ist nicht conformal. Beides steht im Widerspruch zur Annahme.
\qed
\end{Proof}

Aus Lemma \ref{lem:LineConformalS3} folgt unmittelbar, dass der Linegraph eines $\beta$-azyklischen Hypergraphen $S_3$ frei ist.

\rule{\linewidth}{1pt}
\clearpage

\todo{- Quelle (\url{http://epubs.siam.org/sicomp/resource/1/smjcat/v13/i3/p566_s1?isAuthorized=no} und \url{http://dx.doi.org/10.1137/0213035}) downloaden\\
- eventuell Alogrithmus vorstellen (wenn noch Zeit ist)}

\rule{\linewidth}{1pt}
\clearpage

\begin{Theorem}
	DIM ist NP-vollständig für $\alpha$-azyklische Hypergraphen.
\end{Theorem}
\todo{Entstehender Graph ist nicht $\alpha$-azyklisch. $S_3$ funktioniert nicht.}
Reduktion von indipendent perfect dominating set für chordale Graphen (dessen NP-Vollständigkeit in \cite{ChainChin1996})

Gegeben sei ein chordaler Graph $G=(V,E)$. Es wird nun der Hypergraph $H=(\mV,\mE)$ erzeugt.

Zu jedem Knoten $v$ in $G$ gibt es ein Knoten $\nu_v$ in $H$. $$\mV := \{\nu_v\ |\ v \in V\}$$

Zu jeder Kante $e$ in $G$ gibt es einen Knoten $\nu_e$ in $H$. $$\mV := \mV \cup \{\nu_e\ |\ e \in E\}$$

Für jeden Knoten $v$ in $G$ gibt es eine Hyperkante $\varepsilon_v$ in $H$, welche den Knoten $\nu_v$ und die Knoten $\nu_e$ aller mit $v$ verbundenen Kanten $e$ enthällt. $$\varepsilon_v := \{\nu_v\} \cup \{\nu_e\ |\ v\in e\}$$

Für jede max. Clique $K$ in $G$ gibt es ein Hyperkante $\varepsilon_K$ in $H$, welche die Konten $\nu_v$ und $\nu_e$ aller Knoten $v$ und Kanten $e$ der Clique einschließt. $$\varepsilon_K := \{\nu_v\ |\ v \in K\} \cup \{\nu_e\ |\ \text{$e$ ist Kante in $K$}\}$$ $$\mE:=\mE\cup \{\varepsilon_K\ |\ \text{$K$ ist max. Clique in $G$} \}$$

Zusätzlich gint es für jede max. Clique $K$ die Knoten $\nu_K^i$ und $\nu_K^o$ sowie die Hyperkanten $\varepsilon_K^i$ und $\varepsilon_K^o$. Dabei ist $\nu_K^i$ in $\varepsilon_K^i$ und $\varepsilon_K$ enthalten. $\nu_K^o$ ist in $\varepsilon_K^i$ und $\varepsilon_K^o$. Diese Konstruktion sorgt dafür, dass $\varepsilon_K$ immer gematcht werden kann, aber selber nie zum Matching gehört (anderfalls wird $\varepsilon_K^o$ nicht gematcht).
$$\varepsilon_K:=\varepsilon_K\cup\{\nu_K^i\}$$ \\
$$\varepsilon_K^i:=\{\nu_K^i,\nu_K^o\}$$ \\
$$\varepsilon_K^o:=\{\nu_K^o\}$$ \\
$$\mV:=\mV\cup \{\nu_K^i,\nu_K^o\}$$ \\
$$\mE:=\mE\cup\{\varepsilon_K^i,\varepsilon_K^o\}$$

\rule{\linewidth}{1pt}
\clearpage

Dafür wird die im Rahmen dieser Arbeit entwickelte Klasse der $\alpha L$-Graphen vorgestellt. Sie stellen die Linegraphen der $\alpha$-azyklischen Hypergraphen dar.

\subsection{$\alpha L$-Graphen}
Ausgangspunkt für die $\alpha L$-Graphen ist die Graham-Reduktion (siehe Abschnitt~TODO). Bei dieser werden die Hyperkanten eines Hypergrahen nacheinander in eine bestehende Hyperkante eingefügt. Somit ergibt sich für die Hyperkanten eine (gerichtete) Baumstruktur, welche die Reihenfolge darstellt, mit der die Hyperkanten eingefügt wurden. Die Wurzel ist dabei die erste Hyperkante und Blätter die zuletzt eingeführten Hyperkanten. Abbildung~\ref{pic:bsp_BaumstrukturGraham} stellt dies an einem Beispiel dar.
%
%\begin{figure}[htbp]
%    \centering
%    \hspace*{\fill}
%    \subfloat[]{\label{pic:bsp_BaumstrukturGraham_HG}
%        \begin{tikzpicture}
%            
%            \coordinate (c) at (0,0);
%            \coordinate (lt) at (150:1);
%            \coordinate (lb) at (210:1);
%            \coordinate (lbb) at ($(210:1)+(-90:.5)$);
%            \coordinate (ll) at ($(210:1)+(150:1)$);
%            \coordinate (rt) at (30:1);
%            \coordinate (rr) at ($(0:1.4)$);
%            
%            %\draw[gray] (lt)--(c)--(lb)--(ll)--(lt)--(lb)--(lbb) (c)--(30:1)--(rr);
%            
%            \node[ellipse,thick,draw=clGreen,fit=(rr),inner sep=8pt] (e3) {};
%                \node[lbl,black,inner sep=2pt,fill=white,circle] at (e3.west) {3};
%                
%            \node[ellipse,thick,draw=clBlue,fit=(e3)(rt)(c),inner sep=1pt] (e1) {};
%                \node[lbl,black,inner sep=2pt,fill=white,circle] at (e1.north) {1};
%                
%            \node[ellipse,thick,draw=clOrange,fit=(c)(lb)(lt)(ll),inner sep=0pt] (e4) {};
%                \node[lbl,black,inner sep=2pt,fill=white,circle] at (e4.north) {4};
%                
%            \node[ellipse,thick,draw=clRed,fit=(lb)(lbb),inner sep=7pt] (e5) {};
%                \node[lbl,black,inner sep=2pt,fill=white,circle] at (e5.north) {5};
%                
%            \node[ellipse,thick,draw=clViolet,fit=(e4)(e5),inner sep=-3pt] (e2) {};
%                \node[lbl,black,inner sep=2pt,fill=white,circle] at (e2.west) {2};
%                
%            
%        \end{tikzpicture}
%    }
%    \hspace*{\fill}
%    \subfloat[]{\label{pic:bsp_BaumstrukturGraham_T}
%        \begin{tikzpicture}
%        [lbl/.style={font=\small},
%         llbl/.style={left,lbl},rlbl/.style={lbl,right},tlbl/.style={lbl,above}]
%            
%            \node[nN] (1) at ( 0, 0) {}; \node[tlbl] (lbl1) at (1.north) {$1$};
%            \node[nN] (2) at (-1,-1) {}; \node[llbl] (lbl2) at (2.west) {$2$};
%            \node[nN] (3) at ( 1,-1) {}; \node[rlbl] (lbl3) at (3.east) {$3$};
%            \node[nN] (4) at (-2,-2) {}; \node[llbl] (lbl4) at (4.west) {$4$};
%            \node[nN] (5) at ( 0,-2) {}; \node[rlbl] (lbl5) at (5.east) {$5$};
%  
%            \begin{pgfonlayer}{background}
%                \foreach \c/\p in {4/2,5/2,2/1,3/1}
%                {
%                    \draw[->,very thick,clBlue,decoration={snake,amplitude=1},decorate] (\c.center) -- (\p);
%                }
%            \end{pgfonlayer}
%        \end{tikzpicture}
%    }
%    \hspace*{\fill}
%    \caption[Beispeil für die Baumstruktur der Graham-Reduktion]
%    {Beispeil für die Baumstruktur der Graham-Reduktion: Der Hypergraph~\subref{pic:bsp_BaumstrukturGraham_HG} kann erzeugt werden, indem die Hyperkanten in der Reihenfolge ihrer Nummerierung ineinander eingefügt werden ($2$ in $1$, $3$ in $1$, $4$ in $2$, $5$ in $2$). Der Baum~\subref{pic:bsp_BaumstrukturGraham_T} gibt diese Ordnung wieder.}
%    \label{pic:bsp_BaumstrukturGraham}
%\end{figure}
%
%Ein solcher Baum $T$ ist nun ein erster Ansatz für den gesuchten Linegraphen. Zwar ist $T$ ein Spannbaum des gesuchten Linegraphen, allerdings werden nicht alle möglichen Nachbarschaften der Hyperkanten wiedergegeben.
%
%Zwei Knoten $a$ und $b$ sind benachbart in einem Linegraphen, wenn ihre entsprchenden Hyperkanten einen gemeinsamen Knoten $v$ besitzen. Die Graham-Reduktion erlaubt das Einfügen von Knoten jedoch nur in genau eine Hyperkante. Alle anderen Knoten werden geerbt. Eine Hyperkante kann dabei Knoten nur von der Hyperkante erben, in die sie eingefügt wurde. Daraus ergeben sich nun die zwei folgenden Möglichkeiten, wenn $v$ sowohl in $a$ als auch in $b$ ist:
%
%\begin{enumerate}
%    \item  \label{case:a_in_b} $a$ wurde in $b$ eingefügt oder umgekehrt.
%    \item  \label{case:ab_in_e} Es gibt einen gemeinsamen Eltenknoten $e$ in $T$, wobei $v$ in die entsprechende Hyperkante eingefügt wurde.
%\end{enumerate}
%
%Für den Fall~\ref{case:ab_in_e} bedeutet dies, dass auch alle Hyperkanten, deren Knoten auf dem Pfad (in $T$) von $e$ zu $a$ und von $e$ zu $b$ liegen, den Knoten $v$ besitzen. Andernfalls ließe sich $v$ nicht von $e$ auf $a$ und $b$ vererben. Somit sind $a$ und $b$ auch mit allen Knoten auf diesem Pfad benachbart. Abbildung~\ref{pic:bsp_NachbarschaftPfad} stellt dies dar.
%
%\begin{figure}[htbp]
%    \centering
%    \begin{tikzpicture}
%        [lbl/.style={font=\small},
%         llbl/.style={left,lbl},rlbl/.style={lbl,right},tlbl/.style={lbl,above}]
%       
%       \foreach \name/\ang in {a/210,a_t/170,e_l/130,e/90,e_r/50,b_t/10,b/-30}
%       {
%           \node[nN] (\name) at (\ang:2cm) {};
%       }
%    
%       \node[llbl] (lbla) at (a.west) {$a$};
%       \node[rlbl] (lblb) at (b.east) {$b$};
%       \node[tlbl] (lble) at (e.north) {$e$};
%       
%       \begin{pgfonlayer}{background}
%           \foreach \name in {e_l,e,e_r,b_t,b}
%           {
%               \draw (a.center) -- (\name.center);
%           }
%
%           \foreach \name in {a,a_t,e_l,e,e_r}
%           {
%               \draw (b.center) -- (\name.center);
%           }
%
%           \foreach \f/\t in {a/a_t,b/b_t,e_l/e,e_r/e}
%           {
%               \draw[->,very thick,clBlue,decoration={snake,amplitude=1},decorate]
%                   (\f.center) -- (\t);
%           }
%            
%           \foreach \f/\t in {a_t/e_l,b_t/e_r}
%           {
%               \draw[->,dotted,very thick,clBlue,decoration={snake,amplitude=1},decorate]
%                   (\f.center) -- (\t);
%           }
%           
%           \draw[->,thick,clDark25Green] (100:2.35cm) arc[start angle=100,end angle=195,radius=2.35cm];
%           \draw[->,thick,clDark25Green] (80:2.35cm) arc[start angle=80,delta angle=-95,radius=2.35cm];
%           \node[lbl,above left] (lblvl) at (147.5:2.35cm) {$v$};
%           \node[lbl,above right] (lblvr) at (32.5:2.35cm) {$v$};
%            
%       \end{pgfonlayer}
%
%    
%    \end{tikzpicture}
%    \caption[Die Nachbarschaft zweier Hyperkanten entlang des Spannbaums]
%    {Die Nachbarschaft der Hyperkanten $a$ und $b$ entlang des Spannbaums (blau gewellt): Der Knoten $v$ wird entlang des Spannbaums von $e$ auf $a$ und $b$ vererbt (grün). Somit sind $a$ und $b$ auch mit allen Hyperkanten entlang dieser Pfade benachbart.}
%    \label{pic:bsp_NachbarschaftPfad}
%\end{figure}
%
%
%Entsprechend der obigen Argumentation ergibt sich nun Definition~\ref{def:aLGraph} für $\alpha L$-Graphen.
%
%\begin{mydef}[$\alpha L$-Graph\index{$\alpha L$-Graph}]\label{def:aLGraph}
%    Es sei $P_T(u,v)$ die Menge der Knoten auf dem kürzesten Pfad von $u$ nach $v$ in $T$ ($u,v\in P_T(u,v)$).
%    
%    Ein Graph $G=(V,E)$ ist ein \emph{$\alpha L$-Graph} genau dann, wenn $G$ einen Spannbaum $T$ besitzt, sodass für alle Kanten $uv \in E$ gilt:
%    \[ \forall\, w \in P_T(u,v):uw, vw \in E \]
%\end{mydef}

\subsubsection{Konstruktion- und Eliminationsregeln}
Zu zeigen:\\
- Jeder $\alpha L$-Graph kann so konstruiert werden\\
- Jeder so konstruierte Graph ist $\alpha L$-Graph (dazu zeigen, dass jeder $\alpha L$-Graph min einen Knoten hat, der entfenrt werden kann und das nach der Entfernung, der Graph immernoch ein $\alpha L$-Graph ist)\\
$\Rightarrow$ ein $G$ ist $\alpha L$-Graph $\Leftrightarrow$ $G$ hat eine solche Elimin.-/Konstr.Ordnung\\

\subsubsection{Zusammenhang zu $\alpha$-azyklichen Hypergraphen}
Dieser Abschnitt weißt formal nach, dass die $\alpha L$-Graphen genau die Klasse der Linegraphen der $\alpha$-azyklischen Hypergraphen ist (Satz~\ref{theo:aL_genau_LG_von_aazyk}). Dazu wird gezeigt, dass es zu jedem $\alpha L$-Graphen $G$ einen $\alpha$-azyklischen Hypergraphen $H$ gibt, sodass $L(H)=G$ ist (Lemma~\ref{lem:zu_aLGr_ex_aazykHG}). Außerdem wird bewiesen, dass der Linegraph jedes $\alpha$-azyklischen Hypergraphen ein $\alpha L$-Graph ist (Lemma~\ref{lem:aazykHG_hat_aLGr_LG}).

\begin{Lemma}\label{lem:zu_aLGr_ex_aazykHG}
    Zu jedem $\alpha L$-Graphen $G$ existiert ein $\alpha$-azyklischer Hypergraph $H$, sodass $L(H)=G$ ist.
\end{Lemma}

\begin{Proof}
    Es sei $\mL$ ein $\alpha L$-Graph, mit dem dazugehörigen Spannbaum $T$. Der Beweis erfolgt mittels Induktion über $T$. Dazu wird ein belibiger Knoten als Wurzel gewählt und $T$ (bzw. $\mL$) durch das Hinzufügen von Blättern zu $T$ erneut konstruiert.
    
    Gegeben seien ein $\alpha L$-Graph $\mL$, sein Spannbaum $T$ und ein $\alpha$-azyklischer Hypergraph $H$, mit $L(H)=\mL$. Zusätzlich sei $a$ ein Knoten in $\mL$ bzw. eine Hyperkante in $H$. Es wird nun ein Knoten $n$ zu $\mL$ hinzugefügt. Das hinzufügen erfolgt gemäß der oben genannten Konstruktionsregeln für $\alpha L$-Graphen. Das bedeutet, dass $n$ ein neues Blatt von $T$ ist und die Kante $na$ zu $T$ gehört. $H$ wird daraufhin so modifiziert, dass $H$ weiterhin $\alpha$-azyklisch ist und weiterhin $L(H)=\mL$ gilt.
    
    \paragraph{Fall 1:} $n$ wird nur an $a$ angefügt. In die Hyperkante $a$ wird dazu der Knoten $v$ eingefügt. Die Hyperkannte $n$ erhällt nun $v$ als einzigen Knoten, wodurch $n$ vollständig in $a$ enthalten ist. Da beide Veränderungen von $H$ der Graham-Reduktion entsprechen und $n$ nur Knoten mit $a$ gemeinsam hat, ist $H$ somit weiterhin $\alpha$-azyklisch und $\mL$ weiterhin der Linegraph von $H$.
    
%    \begin{figure}[htbp]
%        \centering
%        
%        \hspace*{\fill}
%        \subfloat[]{
%            \begin{tikzpicture}
%                
%                \node[nN] (n) at (0,0) {}; \node[llbl] (lbl_n) at (n.west) {$n$};
%                \node[nN] (a) at (1.4,0) {}; \node[rlbl] (lbl_a) at (a.east) {$a$};
%                %\node[hN] (at) at (2.2,.75) {};
%                %\node[hN] (ab) at (2.2,-.75) {};
%                
%               \begin{pgfonlayer}{background}
%                   \draw[->,Tedge] (n.center) -- (a);
%                   %\draw[Tedge] (a.center) -- (at);
%                   %\draw (a.center) -- (ab);
%               \end{pgfonlayer}
%            \end{tikzpicture}
%            \vspace*{\fill}
%        }
%        \hspace*{\fill}
%        \subfloat[]{
%            \begin{tikzpicture}
%            \node[nN] (v) at (0,0) {}; \node[llbl] (lbl_v) at (v.west) {$v$};
%            
%            \begin{pgfonlayer}{background}
%                \node[fill=clLight60Blue,ellipse,fill opacity=1] (n)
%                     [fit=(v) (lbl_v)] {};
%                     
%                \node [llbl] (lbl_n) at (n.west) {$n$};
%                
%            \end{pgfonlayer}
%            \begin{pgfonlayer}{lowerBackground}
%                \node[fill=clGreen,ellipse,fill opacity=.5] (a)
%                     [fit=(n) (lbl_n)] {};
%
%                \node [llbl] (lbl_a) at (a.west) {$a$};
%
%            \end{pgfonlayer}
%            \end{tikzpicture}
%        }
%        \hspace*{\fill}
%        
%        \caption{Das Hinzufügen eines Knotens $n$ zu $\mL$ und die Veränderung von $H$}
%    \end{figure}
\newsavebox{\tempboxB}
\newsavebox{\tempboxA}

    \sbox{\tempboxB}
    {
        \begin{tikzpicture}
            \node[nN] (v) at (0,0) {}; \node[llbl] (lbl_v) at (v.west) {$v$};
            
            \begin{pgfonlayer}{background}
                \node[fill=clLight60Blue,ellipse,fill opacity=.5] (n)
                     [fit=(v) (lbl_v)] {};
                     
                \node [llbl] (lbl_n) at (n.west) {$n$};
                
            \end{pgfonlayer}
            \begin{pgfonlayer}{lowerBackground}
                \node[fill=clGreen,ellipse,fill opacity=.5] (a)
                     [fit=(n) (lbl_n)] {};

                \node [llbl] (lbl_a) at (a.west) {$a$};

            \end{pgfonlayer}
        \end{tikzpicture}
    }

    \sbox{\tempboxA}
    {
        \begin{tikzpicture}
                
            \node[nN] (n) at (0,0) {}; \node[llbl] (lbl_n) at (n.west) {$n$};
            \node[nN] (a) at (1.5,0) {}; \node[rlbl] (lbl_a) at (a.east) {$a$};
            %\node[hN] (at) at (2.2,.75) {};
            %\node[hN] (ab) at (2.2,-.75) {};
                
            \begin{pgfonlayer}{background}
                \draw[->,Tedge] (n.center) -- (a);
                %\draw[Tedge] (a.center) -- (at);
                %\draw (a.center) -- (ab);
            \end{pgfonlayer}
        \end{tikzpicture}
    }

    \begin{figure}[htbp]
        \centering
        \hspace*{\fill}
        \subfloat[]
        {
            \raisebox{ (\ht\tempboxB-\ht\tempboxA)/2 }{ \usebox{\tempboxA} }
        }%
        \hspace*{\fill}
        \subfloat[]
        {
            \usebox{\tempboxB}
        }%
        \hspace*{\fill}
        \caption{Das Hinzufügen eines Knotens $n$ zu $\mL$ und die Veränderung von $H$}
    \end{figure}

    \paragraph{Fall 2:} Es sei $b$ ein bereits vorhandener Knoten in $\mL$, der mit $a$ verbunden ist. Außerdem sei $P_{ab}$ die Menge der Knoten (inkl. $a$ und $b$) auf dem kürzesten Pfad von $a$ nach $b$ in $T$. $n$ wird nun mit $a$ und $b$ verbunden. Somit wird $n$ auch mit allen Knoten in $P_{ab}$ verbunden.
    
    \begin{figure}[htbp]
        \centering
        \begin{tikzpicture}
            \foreach \n/\a in {b/-90,bb/-30,aa/30,a/90}
            {
                \node[nN] (\n) at (\a:1.0cm) {};
            }
            
            %\node[nN] (n) at ($(a.center)+(-1.5,0)$) {};
            \node[nN] (n) at (150:1.0cm) {};
            
            \node [llbl] (lbl_n) at (n.west) {$n$};
            \node [blbl] (lbl_b) at (b.south) {$b$};
            \node [tlbl] (lbl_a) at (a.north) {$a$};
            
            \draw (n) -- (aa) -- (b) -- (a) -- (bb) -- (n) -- (b);
            
            \draw[thick,clDark25Green] (80:1.35cm) arc[start angle=80,end angle=-80,radius=1.35cm];
            \node[rlbl] (lbl_P) at (0:1.35cm) {$P_{ab}$};
            
            \begin{pgfonlayer}{background}
                \draw[->,Tedge] (n.center) -- (a);
                \draw[Tedge] (a.center) -- (aa.center);
                \draw[Tedge] (b.center) -- (bb.center);
                \draw[Tedge,dotted] (aa.center) -- (bb.center);
            \end{pgfonlayer}
        \end{tikzpicture}
        \caption{TODO}
    \end{figure}
    
%    \begin{figure}[htbp]
%        \centering
%        \begin{tikzpicture}
%            \foreach \n/\a in {b/-72,e/0,a/72,n/144}
%            {
%                \node[nN] (\n) at (\a:1.0cm) {};
%            }
%            
%            %\node[nN] (n) at ($(a.center)+(-1.5,0)$) {};
%            %\node[nN] (n) at (150:1.0cm) {};
%            
%            \node [llbl] (lbl_n) at (n.west) {$n$};
%            \node [blbl] (lbl_b) at (b.south) {$b$};
%            \node [tlbl] (lbl_a) at (a.north) {$a$};
%            \node [rlbl] (lbl_e) at (e.east) {$e$};
%            
%            \draw (n) -- (e) (n) -- (b) -- (a);
%            
%            \draw[thick,clDark25Green] ($(72:1.35cm)+(0.4cm,0)$) arc[start angle=72,end angle=-72,radius=1.35cm];
%            \node[rlbl] (lbl_P) at ($(0:1.35cm)+(0.4cm,0)$) {$P_{ab}$};
%            
%            \begin{pgfonlayer}{background}
%                \draw[->,Tedge] (n.center) -- (a);
%                \draw[Tedge,dotted] (a.center) -- (e.center);
%                \draw[Tedge,dotted] (b.center) -- (e.center);
%                %\draw[Tedge,dotted] (aa.center) -- (bb.center);
%            \end{pgfonlayer}
%        \end{tikzpicture}
%        \caption{}
%    \end{figure}
    
    Damit nun eine Hyperkante $n$ in $H$ eingefügt werden kann, welche die gleiche Nachbarschaft wie der Knoten $n$ in $\mL$ hat, wird ein Knoten benötigt, der in allen Hyperkanten des Pfades $P_{ab}$ enthalten ist, jedoch in keiner anderen. Zwar sind $a$ und $b$ verbunden, woduch es einen gemeinsamen Knoten gibt, jedoch ist es möglich, dass dieser auch in anderen Hyperkanten außer denen in $P_{ab}$ enthalten ist. Aus diesem Grund wird in $H$ ein neuer Knoten $v_n$ eingefügt.
    
    Damit das Einfügen von $v_n$ der Graham-Reduktion entspricht, darf $v_n$ nur in genau eine Hyperkante eingefügt werden. Um dies zu ermöglichen, wird nun die Konstruktion von $H$ soweit rückgängig gemacht, bis nur noch eine Hyperkante $e$ aus $P_{ab}$ vorhanden ist. Nun wird $v_n$ in $e$ eingefügt. Der nächste Schritt ist das erneute Hinzufügen der zuvor entfernten Hyperkanten und Knoten. Für die Hyperkanten aus $P_{ab}$ gilt jedoch, dass sie zusätlich den Knoten $v_n$ erhalten (direkt oder indirekt von $e$ erben). Abschließend wird nun ein neue Hyperkante $n$ in $a$ eingefügt, die nur den Knoten $v_n$ enthällt.
    
    Der so veränderte Hypergraph $H$ ist nun weiterhin $\alpha$-azyklisch, da sämmtliche Änderungen die Regeln der Graham-Reduktion einhalten. Es gilt auch weiterhin, dass $\mL$ der Linegraph von $H$ ist, denn die Hyperkante $n$ wurde so eingefügt, dass sie nur mit den Hyperkanten auf dem Pfad $P_{ab}$ verbunden ist.
    
    \paragraph{Fall 3:} Es seien $b$ und $c$ bereits vorhandene Knoten in $\mL$, die mit $a$ benachbart sind, jedoch nicht miteinander. $n$ soll nun mit $b$ und $c$ verbunden werden.
    
    \begin{figure}[htbp]
        \centering
        \begin{tikzpicture}
            \foreach \n/\a in {b/-90,bb/-30,aa/30,a/90}
            {
                \node[nN] (\n) at (\a:1.0cm) {};
            }
            
            \foreach \n/\a in {ac/30,cc/90,c/150}
            {
                \node[nN] (\n) at ($(90:1cm)+(150:1.0cm)+(\a:1.0cm)$) {};
            }
            
            %\node[nN] (n) at ($(a.center)+(-1.5,0)$) {};
            \node[nN] (n) at (150:1.0cm) {};
            
            \node [lbl,anchor=30] (lbl_n) at (n.210) {$n$};
            \node [lbl, anchor=210] (lbl_a) at (a.30) {$a$};
            
            \node [blbl] (lbl_b) at (b.south) {$b$};
            \node [lbl,anchor=-30] (lbl_c) at (c.120) {$c$};
            
            \draw (n) -- (aa) -- (b) -- (a) -- (bb) -- (n) -- (b);
            \draw (n) -- (ac) -- (c) -- (a) -- (cc) -- (n) -- (c);
            
            \draw[thick,clDark25Green] (65:1.35cm) arc[start angle=65,end angle=-80,radius=1.35cm];
            \draw[thick,clDark25Green] ($(90:1cm)+(150:1.0cm)+(-5:1.35cm)$) arc[start angle=-5,end angle=135,radius=1.35cm];
            
            \node[rlbl] (lbl_Pab) at (0:1.35cm) {$P_{ab}$};
            \node[lbl, anchor=240] (lbl_Pac) at ($(90:1cm)+(150:1.0cm)+(60:1.35cm)$) {$P_{ac}$};
            
            \begin{pgfonlayer}{background}
                \draw[->,Tedge] (n.center) -- (a);
                
                \draw[Tedge] (a.center) -- (aa.center);
                \draw[Tedge] (b.center) -- (bb.center);
                \draw[Tedge,dotted] (aa.center) -- (bb.center);
                
                \draw[Tedge] (a.center) -- (ac.center);
                \draw[Tedge] (c.center) -- (cc.center);
                \draw[Tedge,dotted] (ac.center) -- (cc.center);
                
            \end{pgfonlayer}
        \end{tikzpicture}
        \caption{TODO}
    \end{figure}
    
    Das Verfahren ist hierbei das gleiche wie in Fall~2: Die Konstruktion von $H$ wird rückgängig gemacht. Diesmal werden jedoch zwei Knoten $v_{nb}$ und $v_{nc}$ in die entsprechenden Hyperkanten eingefügt.
    \qed
\end{Proof}

\begin{Lemma}\label{lem:aazykHG_hat_aLGr_LG}
    $H$ ist $\alpha$-azyklisch $\Rightarrow$ $L(H)$ ist ein $\alpha L$-Graph.
\end{Lemma}

\begin{Proof}
    Der Beweis erfolgt mittels Induktion über die Konstruktion eines $\alpha$-azyklischen Hypergraphen gemäß der Graham-Reduktion.
    
    Es seien $H$ ein $\alpha$-azyklsicher Hypergraph und $\mL=L(H)$ sein Linegraph. Dabei sei $\mL$ ein $\alpha L$-Graph mit dem dazugehörigen Spannbaum $T$. Zusätzlich sei $P_{uv}$ die Menge der Knoten auf dem kürzesten Pfad von $u$ nach $v$ in $T$ ($u,v\in P_{uv}$).
    
    Es wird nun (gemäß der Graham-Reduktion) eine neue Hyperkante $n$ in eine bestehende Hyperkante $a$ eingefügt.
    
    \paragraph{Fall 1:} $n$ wird nur in $a$ eingefügt. In $\mL$ hat $n$ somit nur $a$ als Nachbar. Fügt man die Kante $na$ auch zu $T$ hinzu, ist $T$ wieder ein Spannbaum von $\mL$. Da $n$ keine weiteren Knoten hat, ist $\mL$ auch weiterhin ein $\alpha L$-Graph.
    
    \todo{Abbildung für Fall 1}
    
    \paragraph{Fall 2:} $n$ wird in $a$ eingefügt und ist mit $b$ benachbart. Somit muss es in $H$ einen Knoten $v$ geben mit $v\in(a\cap b \cap n)$. Also gibt es auch eine Hyperkante $e$ in $H$, in die $v$ eingefügt wurde.
    
    $\mL$ wird nun folgendermaßen erweitert: $n$ wird mit allen Knoten auf den Pfaden $P_{ae}$, $P_{be}$ und $P_{ab}$ verbunden. Außerdem wird die Kante $na$ zu $T$ hinzugefügt.
    
    \todo{Abbildung für Fall 2}
    
    Da keine Kante zwischen bereits vorhanden Knoten hinzugefügt wurde, ist $T$ mit der Kante $na$ weiterhin ein Spannbaum von $\mL$.
    
    Per Vorraussetzung sind alle Knoten $w$ auf den Pfaden $P_{ae}$, $P_{be}$ und $P_{ab}$ mit allen Knoten auf den Pfad $P_{aw}$ verbunden (ihre Hyperkanten enthalten alle $v$). Da $n$ mit allen Knoten $w$ verbunden wurde und $n$ in $T$ mit $a$ benachbart ist, ist somit auch $w$ mit allen Knoten auf $P_{nw}$ verbunden. Bereits bestehnde Nachbarschaften wurden nicht verändert. Somit ist $\mL$ weiterhin ein $\alpha L$-Graph.
    \qed
\end{Proof}

Aus den Lemmata~\ref{lem:zu_aLGr_ex_aazykHG} und \ref{lem:aazykHG_hat_aLGr_LG} ergibt sich nun Satz \ref{theo:aL_genau_LG_von_aazyk}.

\begin{Theorem}\label{theo:aL_genau_LG_von_aazyk}
    Die Klasse der $\alpha L$-Graphen ist genau die Klasse der Linegraphen von $\alpha$-azyklischen Hypergraphen.
\end{Theorem}

\subsection{Linegraphen $\beta$-azyklischer Hypergraphen}

- $H$ $\beta$-azyk $\Rightarrow$ $L(H)$ strongly chordal \\
- eventuell beweisen mit:  hereditary dually chordal = strongly chordal (siehe \url{http://www.graphclasses.org/classes/gc_318.html})


\begin{Lemma}\label{lem:chordalWkFrei}
    Für einen Hypergraph $H$, dessen 2-Section Graph chordal ist, gilt: $L(H)$ ist genau dann chordal, wenn $L(H)$ $W_k$-frei ($k\geq4$) ist.
\end{Lemma}

\begin{Proof}
Es seien $H=(V,\mE)$, $\mS=2Sec(H)$, $\mL=L(H)=(\mE,E)$ und $k\geq4$. Des weiteren sei $n_{xy}\in x\cap y$ ($x,y\in\mE$) ein gemeinsamer Knoten der Kanten $x$ und $y$ in $H$. 

\prR Wenn $\mL$ chordal ist, dann existiert kein $C_k$ in $\mL$. Somit kann auch kein $W_k$ vorhanden sein.

\prL $\mL$ ist nicht chordal. Es sei $\mC$ der kleinste $C_k$ in $\mL$. $\mC$ besitzt dann die Knoten $u$, $v$, $w$ und $i$, $j$, $k$ mit $\{uv,vw,ij,jk\}\subseteq E$, $u\notin\{j,k\}$, $v\notin\{i,j,k\}$, $w\notin\{i,j\}$, $i\notin\{v,w\}$, $j\notin\{u,v,w\}$ und $k\notin\{u,v\}$ (siehe Abbildung \ref{pic:ProofLineChordalWheel_Circle}). Ist $\mC$ ein $C_4$, dann ist $u=i$ und $w=k$.

\floatimage{pic:ProofLineChordalWheel_Circle}{Skizze für den Kreis $\mC$}
{\begin{tikzpicture}
  [thick,node distance=2.5cm,
   lbl/.style={font=\small},
   llbl/.style={left,lbl},rlbl/.style={lbl,right}]
  
  \node [nN] (u) at (120:1.5) {}; \node[llbl] (lblu) at (u.west) {$u$};
  \node [nN] (v) at (180:1.5) {}; \node[llbl] (lblv) at (v.west) {$v$};
  \node [nN] (w) at (240:1.5) {}; \node[llbl] (lblw) at (w.west) {$w$};
  
  \draw (u)--(v)--(w);
  
  \node [nN] (i) at ( 60:1.5) {}; \node[rlbl] (lbli) at (i.east) {$i$};
  \node [nN] (j) at (  0:1.5) {}; \node[rlbl] (lblj) at (j.east) {$j$};
  \node [nN] (k) at (-60:1.5) {}; \node[rlbl] (lblk) at (k.east) {$k$};
  
  \draw (i)--(j)--(k);
  
  \draw [dashed] (u)--(i) (w)--(k);
\end{tikzpicture}}

Entsprechend sind $u$, $v$, $w$ und $i$, $j$, $k$ Kanten und $n_{uv}$, $n_{vw}$, $n_{ij}$ und $n_{jk}$ Knoten in $H$, die in $\mS$ einen Kreis $\mN$ bilden (siehe Abbildung \ref{pic:ProofLineChordalWheel_Hyper}).

\floatimage{pic:ProofLineChordalWheel_Hyper}{Skizze für $\mC$ in $H$ und $\mS$}{
\begin{tikzpicture}
  [thick,node distance=2.5cm,
   lbl/.style={},
   llbl/.style={left,lbl},rlbl/.style={lbl,right}]
     
  \node [hN] (iu) at ( 90:1.6) {}; 
  \node [nN] (uv) at (150:1.5) {}; \node[rlbl,inner sep=10pt] at (uv.east) {$n_{uv}$}; 
  \node [nN] (vw) at (210:1.5) {}; \node[rlbl,inner sep=10pt] at (vw.east) {$n_{vw}$}; 

  \node [hN] (wk) at (-90:1.5) {}; 
  \node [nN] (kj) at (-30:1.5) {}; \node[llbl,inner sep=10pt] at (kj.west) {$n_{jk}$}; 
  \node [nN] (ji) at ( 30:1.5) {}; \node[llbl,inner sep=10pt] at (ji.west) {$n_{ij}$}; 

  \node [hN] (iul) at ($(uv)!1.cm!30:(iu)$) {}; 
  \node [hN] (iur) at ($(ji)!1.cm!-30:(iu)$) {}; 

  \node [hN] (wkl) at ($(vw)!1.cm!-30:(wk)$) {}; 
  \node [hN] (wkr) at ($(kj)!1.cm!30:(wk)$) {}; 

  
  \begin{pgfonlayer}{background}
      \node [fill=clRed,ellipse,fill opacity=.5,rotate fit= 60] (u)
            [fit=(iul) (uv)] {};
      \node [lbl, above left] at (u.north) {$u$};
      
      \node [fill=clGreen,ellipse,fill opacity=.5,rotate fit= 90] (v)
            [fit=(uv) (vw)] {};
      \node [lbl, left] at (v.north) {$v$};
      
      \node [fill=clBlue,ellipse,fill opacity=.5,rotate fit=120] (w)
            [fit=(vw)(wkl)] {};
      \node [lbl, below left] at (w.north) {$w$};
      

      \node [fill=clRed,ellipse,fill opacity=.5,rotate fit=-60] (i)
            [fit=(iur)(ji)] {};
      \node [lbl, above right] at (i.north) {$i$};
      
      \node [fill=clGreen,ellipse,fill opacity=.5,rotate fit=-90] (j)
            [fit=(kj)(ji)] {};
      \node [lbl, right] at (j.north) {$j$};
      
      \node [fill=clBlue,ellipse,fill opacity=.5,rotate fit=-120] (k)
            [fit=(wkr)(kj)] {};
      \node [lbl, below right] at (k.north) {$k$};
      
                       
  \end{pgfonlayer}

  \draw (uv)--(vw) (ji)--(kj);
  \draw [dashed] (uv) to [out=60,in=120] (ji) (vw) to [out=-60,in=-120] (kj);
  
\end{tikzpicture}}

Da $\mS$ chordal ist, kann $\mN$ kein induzierter Teilgraph von $\mS$ sein. Folglich gibt es Sehnen in $\mN$. \OBdA sei $n_{uv}n_{jk}$ eine solche Sehne. Aufgrund der Definition für 2-Section Graphen muss es auch eine Hyperkante $e$ in $H$ geben mit $\{n_{uv},n_{jk}\}\subseteq e$. Somit ist $e$ auch ein Knoten in $\mL$, der mit $u$, $v$, $j$ und $k$ verbunden ist.

\floatimage{pic:ProofLineChordalWheel_Wheel}{Skizze für den Kreis $\mC$ und den Knoten $e$}
{\begin{tikzpicture}
  [thick,node distance=2.5cm,
   lbl/.style={font=\small},
   llbl/.style={left,lbl},rlbl/.style={lbl,right}]
  
  \node [nN] (u) at (120:1.5) {}; \node[llbl] (lblu) at (u.west) {$u$};
  \node [nN] (v) at (180:1.5) {}; \node[llbl] (lblv) at (v.west) {$v$};
  \node [nN] (w) at (240:1.5) {}; \node[llbl] (lblw) at (w.west) {$w$};
  
  \draw (u)--(v)--(w);
  
  \node [nN] (i) at ( 60:1.5) {}; \node[rlbl] (lbli) at (i.east) {$i$};
  \node [nN] (j) at (  0:1.5) {}; \node[rlbl] (lblj) at (j.east) {$j$};
  \node [nN] (k) at (-60:1.5) {}; \node[rlbl] (lblk) at (k.east) {$k$};

  \draw (i)--(j)--(k);
 
  \draw [dashed] (u)--(i) (w)--(k);

  \node [nN] (e) at (0,0) {}; \node[lbl,below] (lble) at (e.south) {$e$};
  \foreach \n in {u,v,j,k}
      \draw (\n)--(e);
\end{tikzpicture}}

Nun gibt es zwei Fälle:
\begin{enumerate}
    \item $\mC$ und $e$ bilden ein $W_k$.
    \item $\mC$ und $e$ bilden kein $W_k$. Dann bilden die Knoten $e,u,\ldots,i,j,e$ bzw.  $e,v,w,\ldots,k,e$ einen Kreis. Dies widerspricht der Voraussetzung, dass $\mC$ der kleinste Kreis in $\mL$ ist.
\end{enumerate}
Somit gibt es einen $W_k$ in $\mL$, falls dieser nicht chordal ist. \qed
\end{Proof}

- $H$ $\gamma$-azyk $\Rightarrow$ $L(H)$ Gem-free chordal (Gem erzeugt $\gamma$-cycle)

\rule{\linewidth}{1pt}
\clearpage


Eine Matching ist eine Teilmenge von Kanten eines (Hyper)Graphen, wobei die Kanten keinen gemeinsamen Knoten besitzen.

\begin{mydef}[Matching\index{Matching}]\label{def:Matching}
Gegeben sei ein Hypergraph $H=(V,\mE)$. Eine Menge $M \subseteq \mE$ heißt \emph{Matching}, wenn für alle $e,e' \in M$ gilt: $e\cap e' = \emptyset$.
\end{mydef}

Ein dominating induced Matching (dim) zeichnet sich dadurch aus, dass jede Kante, die nicht zum Matching gehört, einen gemeinsamen Knoten mit einer Kante des Matchings besitzt (dominating), und dass zwischen zwei Kanten des Matchings immer mindestens zwei weitere Kanten liegen (induced).

Gehört eine Kante $m$ zum Matching oder ist sie mit einer Kante des Matchings verbunden, so wird gesagt: $m$ wird gematcht.

\begin{mydef}[dominating induced Matching\index{dominating induced Matching}]
Gegeben sei ein Hypergraph $H=(V,\mE)$ und ein Matching $M$. $M$ ist ein \emph{dominating induced Matching}, wenn gilt:
%%% Original-Definition
% \forall\ e,e' \in M&: dist_G(e,e') \geq 2 & \text{\emph{induced}} \\
% \forall\ e \in \mE\, \backslash\, M &: \exists\ \varepsilon \in M \text{ mit } \varepsilon\cap e \neq \emptyset & \text{\emph{dominating}}
\[ \forall\,e \in \mE : \exists!\, m \in M \text{ mit } m\cap e \neq \emptyset \]
\end{mydef}

%\begin{mydef}[induced Matching]
%Gegeben sei ein Graph $G=(V,E)$ und ein Matching $M$. $M$ ist ein \emph{induced Matching}, wenn für alle $e,e' \in M$ gilt: $dist_G(e,e') \geq 2$.
%\end{mydef}

%\begin{mydef}[dominating Matching]\label{def:dominatingMatching}
%Gegeben sei ein Graph $G$ und ein Matching $M$. $M$ ist ein \emph{dominating Matching}, wenn für alle $e \in E \backslash M$ gilt: Es existiert ein $e' \in M$ mit $e\cap e' \neq \emptyset$.
%\end{mydef}

Das dazugehörige Entscheidungsproblem (Dominating Induced Matching Problem; kurz~DIM\index{DIM}) fragt, ob ein Graph ein solches Matching besitzt. Für Graphen wird es auch als Efficient Edge Domination Problem\index{Efficient Edge Domination} (EED\index{EED}) bezeichnet und ist NP-vollständig \cite{dimNPv}. Dies gilt auch für Hypergraphen, da jeder Graph auch ein Hypergraph ist.

Das Besondere an DIM ist, dass es sowohl ein Pack- als auch ein Abdeckungsproblem darstellt. Es müssen genug Kanten gewählt werden, damit alle Kanten, die nicht zum Matching gehören mit einer des Matchings verbunden sind. Zwischen den Kanten des Matching müssen sich aber gleichzeitig immer mindestens zwei weitere Kanten befinden. Somit muss jede Kante genau ein mal gematcht werden.

\todo{Beispiel}
\begin{figure}[htb]
\subfloat[]{\begin{tikzpicture}
  [thick,node distance=.5cm]

  \node[mN] (a) at (-1.125,0) {};
  \node[mN] (b) [right=of a] {};
  \node[nN] (c) [right=of b] {};
  \node[mN] (d) [right=of c] {};
  \node[mN] (e) [right=of d] {};
  \node[nN] (f) [below=of c] {};
    
    \begin{pgfonlayer}{background}
        \foreach \x/\y in {a/b,d/e} {
            \draw [very thick,-,darkgreen] (\x.center) -- (\y.center);
        }

        \foreach \x/\y in {b/c,c/d} {
            \draw [very thick,-,blue] (\x.center) -- (\y.center);
        }
        
        \draw [very thick,-,red] (c.center) -- (f.center);
        
    \end{pgfonlayer}

 \foreach \x/\y in {a/b,b/c,c/d,d/e,c/f} {
     %\draw (\x) -- (\y);
 }
  
\end{tikzpicture}}
\hspace*{\fill}
\subfloat[]{\begin{tikzpicture}
  [thick,node distance=.5cm]

  \node[nN] (a) at (-1.125,0) {};
  \node[nN] (b) [right=of a] {};
  \node[nN] (c) [right=of b] {};
  \node[nN] (d) [right=of c] {};
  \node[nN] (e) [right=of d] {};
  \node[nN] (f) [below=of c] {};
    
    \begin{pgfonlayer}{background}
        \foreach \x/\y in {b/c} {
            \draw [line width=0.075cm,-,darkgreen] (\x.center) -- (\y.center);
        }

        \foreach \x/\y in {a/b,c/d,c/f} {
            \draw [line width=0.075cm,-,orange] (\x.center) -- (\y.center);
        }
        
        \draw [line width=0.075cm,-,violet] (d.center) -- (e.center);
        
    \end{pgfonlayer}

 \foreach \x/\y in {a/b,b/c,c/d,d/e,c/f} {
     %\draw (\x) -- (\y);
 }
  
\end{tikzpicture}}
\hspace*{\fill}
\subfloat[]{\begin{tikzpicture}
  [thick,
   normalN/.style={circle,draw,minimum size=0.25cm,inner sep=0pt,fill=white},
   node distance=.5cm]

  \node[normalN] (a) at (-1.125,0) {};
  \node[normalN] (b) [right=of a] {};
  \node[normalN] (c) [right=of b] {};
  \node[normalN] (d) [right=of c] {};
  \node[normalN] (e) [right=of d] {};
  \node[normalN] (f) [below=of c] {};

    
    \begin{pgfonlayer}{background}
        \foreach \x/\y in {a/b,d/e} {
            \draw [line width=0.075cm,-,bgred] (\x.center) -- (\y.center);
        }

        \foreach \x/\y in {b/c,c/d} {
            \draw [line width=0.075cm,-,bgblue] (\x.center) -- (\y.center);
        }
        
        \draw [line width=0.075cm,-,darkgreen] (c.center) -- (f.center);
        
    \end{pgfonlayer}

 \foreach \x/\y in {a/b,b/c,c/d,d/e,c/f} {
     %\draw (\x) -- (\y);
 }
\end{tikzpicture}}
\caption{Die Graphen $P_4$, $C_5$ und $K_6$.}
\label{pic:bsp_DIM}
\end{figure}

\begin{figure}[htb]
\centering
\begin{tikzpicture}
  [thick,
   normalN/.style={circle,draw,minimum size=0.25cm,inner sep=0pt,fill=white},
   greenN/.style={normalN,fill=bggreen},
   node distance=.5cm]

  \node[normalN] (a) at (-1.125,0) {};
  \node[normalN] (b) [right=of a] {};
  \node[normalN] (c) [right=of b] {};
  \node[normalN] (d) [right=of c] {};
  \node[normalN] (e) [right=of d] {};
  \node[normalN] (f) [below=of c] {};
  \node[normalN] (g) [below=of f] {};
  
    \begin{pgfonlayer}{background}
        \foreach \x/\y in {a/b,d/e,f/g} {
            \draw [line width=0.075cm,darkgreen] (\x.center) -- (\y.center);
        }

        \foreach \x/\y in {b/c,c/d,c/f} {
            \draw [line width=0.075cm,-,bgblue] (\x.center) -- (\y.center);
        }
    \end{pgfonlayer}

 \foreach \x/\y in {a/b,b/c,c/d,d/e,c/f,f/g} {
     %\draw (\x) -- (\y);
 }
  
\end{tikzpicture}
\caption{TODO}
\label{pic:bsp_BnB_1}
\end{figure}

\section{DIM für spezielle Graphen-Klassen}



%\chapter{Design}

Das Kapitel dient erstmal als Spielwiese und um ansatzweise die Möglichkeiten des Designs zu beschreiben.

\section{Listen}

Irgend ein Text. Irgend ein Text. Irgend ein Text. Irgend ein Text. Irgend ein Text. Irgend ein Text. Irgend ein Text. Irgend ein Text. Irgend ein Text. Irgend ein Text.

Irgend ein Text. Irgend ein Text. Irgend ein Text. Irgend ein Text. Irgend ein Text. Irgend ein Text. Irgend ein Text. Irgend ein Text. Irgend ein Text. Irgend ein Text. \footnote{bla}
\begin{itemize}
	\item bla
	\item bla
	\item bla
\end{itemize}
Irgend ein Text. Irgend ein Text. Irgend ein Text. Irgend ein Text. Irgend ein Text. Irgend ein Text. Irgend ein Text. Irgend ein Text. Irgend ein Text. Irgend ein Text.

Irgend ein Text. Irgend ein Text. Irgend ein Text. Irgend ein Text. Irgend ein Text. Irgend ein Text. Irgend ein Text. Irgend ein Text. Irgend ein Text. Irgend ein Text. 

\begin{itemize}
	\item bla
	\item bla
	\item bla
\end{itemize}

Irgend ein Text. Irgend ein Text. Irgend ein Text. Irgend ein Text. Irgend ein Text. Irgend ein Text. Irgend ein Text. Irgend ein Text. Irgend ein Text. Irgend ein Text. Irgend ein Text. Irgend ein Text. Irgend ein Text. Irgend ein Text. Irgend ein Text. Irgend ein Text. Irgend ein Text. Irgend ein Text. Irgend ein Text. Irgend ein Text. 

\begin{itemize}
	\item Irgend ein Text. Irgend ein Text. Irgend ein Text. Irgend ein Text. Irgend ein Text. Irgend ein Text. Irgend ein Text. Irgend ein Text. Irgend ein Text. Irgend ein Text. Irgend ein Text. Irgend ein Text. Irgend ein Text. Irgend ein Text. Irgend ein Text. Irgend ein Text. Irgend ein Text. Irgend ein Text. Irgend ein Text. Irgend ein Text.
	
	Irgend ein Text. Irgend ein Text. Irgend ein Text. Irgend ein Text. Irgend ein Text. Irgend ein Text. Irgend ein Text. Irgend ein Text. Irgend ein Text. Irgend ein Text. Irgend ein Text. Irgend ein Text. Irgend ein Text. Irgend ein Text. Irgend ein Text. Irgend ein Text. Irgend ein Text. Irgend ein Text. Irgend ein Text. Irgend ein Text.
	
	\item bla
	\item bla
\end{itemize}


\section{Definitionen und Co.}
Definitionen sind blau hinterlegt.
Sätze und Lemmata sind grün hinterlegt und Sätze zusätzlich dunkelgrün gerahmt.
Die Rahmung dient zum stärkeren hervorheben.
Beweise sind weder gerahmt, noch hinterlegt. Das qed-Symbol muss mittels \texttt{\small\textbackslash qed} selbst gesetzt werden.
ToDos sind rot, damit sie gut auffallen.

%\subsection{Anwendung}
%\subsubsection{Definitionen, Sätze und Beweise}
%{\ttfamily\small
%\textbackslash begin\{mydef / myTheo / myProof\}[...]\\
%\ \ ...\\
%\textbackslash end\{mydef / myTheo / myProof\}
%}
%
%\subsubsection{ToDo}
%{\ttfamily\small
%\textbackslash todo\{...\}
%}
%
\subsection{Beispiel}
\begin{mydef}[Graph]Ein Graph $G$ ist ein 2-Tupel $G=(V,E)$. Dabei ist
 $V$ eine endliche Menge von Knoten und
 $E \subseteq \{\{u,v\} \ |\ u,v \in V \wedge u \neq v\}$ die Menge an Kanten.

Eine Kante $\{u,v\}$ wird abgekürzt durch $uv$.
\end{mydef}

\begin{Theorem}
Graphen sind toll!
\end{Theorem}

\begin{Lemma}\label{Lem:NurTolleSachen}
Ich schreibe nur über tolle Sachen.
\end{Lemma}

\begin{Proof}
Angenommen, Graphen wären nicht toll. Dann würde ich diese Arbeit nicht schreiben. $\overset{Lem. \ref{Lem:NurTolleSachen}}{\Rightarrow}$ Annahme falsch. $\Rightarrow$ Graphen sind toll! \qed
\end{Proof}

\todo{Ein gutes Beispiel für den \textbackslash todo-Befehl finden.}

\section{Listings}
Listings sehen noch voll doof aus und können kein UTF8.
\begin{lstlisting}[caption={McGregor-Algorithmus},label={lst:McGregor}]
Procedure 3+1(i) //  $\color{darkgreen}i\geq 1; i\in \mathbb{N}$
    While (i > 1)
        If i Mod 2 = 0 Then
            i = i / 2
        Else
            i = 3 * i + 1
        End If
    End While
End Procedure
\end{lstlisting}

\todo{UTF8 für Listings}

\todo{Listings in hübsch :)

Eventuell einrahmen und/oder hinterlegen. Dann kann eventuell auch auf float verzichtet werden.}


%\begin{framed}\ttfamily%\bfseries \boldmath
%Procedure BlaBla(i) {\color{darkgreen}// $i\geq 1; i\in \mathbb{N}$\\}
%\ \ \ \ While (i > 1)\\
%\ \ \ \ \ \ \ \ If i Mod 2 = 0 Then\\
%\ \ \ \ \ \ \ \ \ \ \ \ i = i / 2\\
%\ \ \ \ \ \ \ \ Else\\
%\ \ \ \ \ \ \ \ \ \ \ \ i = 3 * i + 1\\
%\ \ \ \ \ \ \ \ End If\\
%\ \ \ \ End While\\
%End Procedure
%\end{framed}

\section{Join und Cojoin}
Join (\textbackslash join) und Cojoin (\textbackslash cojoin) wurden mit TikZ umgesetzt, da \textbackslash textcircled doof aussieht :)

\subsection{Code}
{\ttfamily\small
\textbackslash newcommand\{\textbackslash cirBase\}[1]\\
\{\textbackslash tikz[baseline,line width=.1ex]\{\textbackslash node [draw,anchor=base,inner sep=.2ex,circle] \{\#1\};\}\}

\textbackslash DeclareRobustCommand\{\textbackslash mathCir\}[1]\{\\
\ \ \textbackslash ifmmode \textbackslash ,\textbackslash cirBase\{\#1\}\textbackslash ,\\
\ \ \textbackslash else\\
\ \ \ \ \textbackslash cirBase\{\#1\}\\
\ \ \textbackslash fi\\
\}

\textbackslash newcommand\{\textbackslash join\}\{\textbackslash mathCir\{1\}\}\\
\textbackslash newcommand\{\textbackslash cojoin\}\{\textbackslash mathCir\{0\}\}
}%\normalfont

\subsection{Vergleich mit \textbackslash textcircled}
\subsubsection{\textbackslash textcircled\{1\}}

Bla \textcircled{1} bla.

{\huge\bfseries Groß \textcircled{1} und fett mit Euro \textcircled{€}.}

Mathemodus $G_1 \textcircled{1} G_2$


\subsubsection{\textbackslash join}

Bla \join bla.

{\huge\bfseries Groß \join und fett mit Euro \cirBase{€}. \footnote{mit \textbackslash cirBase\{€\}}}

Mathemodus $G_1\join G_2$

\appendix%\parskip 2pt

%\chapter{Definitionen}
%{\setlength{\parskip}{2pt}\listtheorems{DefBox}}
%\todo{Einheitliche Quellen:\\
%- einheitlich Namen von Autoren, Journals usw. -- nicht mal abgekürzt und mal ausgeschrieben\\
%- Keine Quellen doppelt}
%

\bibliography{quellen}

\listoffigures

%\printindex

\chapter*{Erklärung}
Hiermit versichere ich, dass ich die vorliegende Arbeit selbstständig und 
nur unter Benutzung der angegebenen Hilfsmittel verfasst habe.

\vspace{4\baselineskip}

Rostock, den \today



\end{document}