\usepackage[ngerman]{babel}
%\usepackage{blindtext}
%\usepackage{fontspec}% provides font selecting commands
%\usepackage{xunicode}% provides unicode character macros
\usepackage{xltxtra} % provides some fixes/extras 
%\usepackage{mathpazo}
\defaultfontfeatures{Mapping=tex-text}
\setmainfont{Cambria}
\setsansfont{Calibri}
\setmonofont{Consolas}

% Verknüpfungen in der PDF
\usepackage{hyperref}

%%%%%%%%%%%%%%%%%%%%%%%%%%%%%%%%%%%%%%%%%%%%%%%%%%%%%%%%%%%%%%
% Quellcode
\usepackage{listings}
\lstdefinelanguage{blub}{sensitive=false,morekeywords={for,each,mod,if,to,else,while,end,then,Procedure,until,new,in,do}}
\lstset{%inputencoding=utf8/latin1,
	basicstyle=\ttfamily\small,
	keywordstyle=\color{blue},
	commentstyle=\color{darkgreen},
	numbers=none,
	breaklines=true,
	showstringspaces=false,
	tabsize=4,
	captionpos=b,
	float=htp,
	frame=TB,
	language=blub,
	mathescape=true,
	morecomment=[l]{//},
}



%%%%%%%%%%%%%%%%%%%%%%%%%%%%%%%%%%%%%%%%%%%%%%%%%%%%%%%%%%%%
% (Farbige) Rahmen
\usepackage{framed}
\setlength{\fboxrule}{.5pt}
\setlength{\fboxsep}{4pt}


\newenvironment{DefFrame}{
    \def\FrameCommand{\fcolorbox{clLight80Blue!33}{clLight80Blue!33}}
    \MakeFramed{\advance\hsize-\width \FrameRestore}
  }
  {\endMakeFramed}

\newenvironment{ToDoFrame}{
    \def\FrameCommand{\fcolorbox{clRed}{clLight40Red!33}}
    \MakeFramed{\advance\hsize-\width \FrameRestore}
  }
  {\endMakeFramed}

\newenvironment{TheoFrame}{
    \def\FrameCommand{\fcolorbox{clDark25Green}{clLight40Green!33}}
    \MakeFramed{\advance\hsize-\width \FrameRestore}
  }
  {\endMakeFramed}

\newenvironment{LemFrame}{
    \def\FrameCommand{\fcolorbox{clLight40Green!33}{clLight40Green!33}}
    \MakeFramed{\advance\hsize-\width \FrameRestore}
  }
  {\endMakeFramed}


%%%%%%%%%%%%%%%%%%%%%%%%%%%%%%%%%%%%%%%%%%%%%%%%%%%%%%%%
% Blöcke für Definitionen, Sätze und Beweise
%\usepackage[framed,standard,hyperref]{ntheorem}
\usepackage[framed,hyperref]{ntheorem}

\theoremstyle{plain}
\theoremheaderfont{\rmfamily\bfseries}
\theorembodyfont{\normalfont}
\theoremseparator{}
\theoremprework{}
\theorempreskipamount 0pt
\theorempostskipamount 0pt

\newtheorem*{myProof}{Beweis}
\newtheorem{DefBox}{Definition}[chapter]
\newtheorem{TheoBox}{Satz}
\newtheorem{LemBox}{Lemma}

\newenvironment{mydef}%[1][]%
{\begin{DefFrame}\begin{DefBox}%[#1]
}%
{\end{DefBox}\end{DefFrame}}
 
\newenvironment{myTheo}%
{\begin{TheoFrame}\begin{TheoBox}}%
{\end{TheoBox}\end{TheoFrame}}
 
\newenvironment{Lemma}%
{\begin{LemFrame}\begin{LemBox}}%
{\end{LemBox}\end{LemFrame}} 



\makeatletter % Anpssung der Theorem-Konstrukte: \endtrivlist                          wird früher verwendet, um den erzeugten Platz am Ende des Theorems zu entfernen. Dient primär dazu, damit es in einem gefärbten Frame besser aussieht.
\gdef\@thm#1#2#3{%
   \if@thmmarks
     \stepcounter{end\InTheoType ctr}%
   \fi
   \renewcommand{\InTheoType}{#1}%
   \if@thmmarks
     \stepcounter{curr#1ctr}%
     \setcounter{end#1ctr}{0}%
   \fi
   \refstepcounter{#2}%
   \theorem@prework
   \thm@topsepadd \theorempostskipamount   % cf. latex.ltx: \@trivlist
   \ifvmode \advance\thm@topsepadd\partopsep\fi
   \trivlist                          
   \@topsep \theorempreskipamount
   \@topsepadd \thm@topsepadd        % used by \@endparenv
   \advance\linewidth -\theorem@indent
   \advance\@totalleftmargin \theorem@indent
   \parshape \@ne \@totalleftmargin \linewidth
   \@ifnextchar[{\@ythm{#1}{#2}{#3}}{\@xthm{#1}{#2}{#3}}}\endtrivlist %%%%%%%%%%%%%%%% neu
\gdef\@endtheorem{%
  %\endtrivlist                          %%%%%%%%%%%%%%%%%%%%%%%%%%%%%%%%%%%%%%%%%%%%%%%%%%%
  \csname\InTheoType @postwork\endcsname
  }

\makeatother


%%%%%%%%%%%%%%%%%%%%%%%%%%%%%%%%%%%%%%%%%%%%%%%%%%%
% Bilder und Abbildungen

\usepackage{graphicx} % Bilder
\usepackage{subfig} % Abbildungen aus mehreren Bildern

% Rahmen für Abbildungen in einen Befehl kapseln
\newcommand{\floatimage}[3]{\begin{figure}[htb]
\centering
#3
\caption{#2}
\label{#1}
\end{figure}}

% Mathe Symbole
\usepackage{amssymbExp}
\usepackage{amsmath}
\DeclareRobustCommand{\qed}{%
  \ifmmode \square%\mathqed
  \else
    \leavevmode\unskip\penalty9999 \hbox{}\nobreak\hfill
    \quad\hbox{$\square$}%
  \fi
}

% ToDo-Befehl
\newcommand{\todo}[1]{
    \begin{ToDoFrame}\itshape
        {\rmfamily\normalsize\bfseries ToDo:} #1
    \end{ToDoFrame}
}

% Abstand nach einem Absatz
\newcommand{\parspace}{\parskip 10pt}

%%%%%%%%%%%%%%%%%%%%%%%%%%%%%%%%%%%%%%%%%%%%%%%%%%%%%%%%%%%%%%%%%
% Farben
\usepackage[table]{xcolor}
\definecolor{clDefBG}{rgb}{.776,.850,.941}
\definecolor{clSecTxt}{rgb}{.122,.286,.490}
\definecolor{clChapTxt}{rgb}{.090,.212,.365}%{.06,.14,.24}

\definecolor{clLight80Blue}{rgb}{.776,.850,.941}
\definecolor{clLight60Blue}{rgb}{.553,.584,.886}
\definecolor{clLight40Blue}{rgb}{.329,.553,.831}
\definecolor{clBlue}{rgb}{.122,.286,.490}
\definecolor{clDark25Blue}{rgb}{.090,.212,.365}
\definecolor{clDark50Blue}{rgb}{.059,.141,.243}

\definecolor{clLight80Green}{rgb}{.886,.894,.906}
\definecolor{clLight60Green}{rgb}{.843,.890,.737}
\definecolor{clLight40Green}{rgb}{.765,.839,.608}
\definecolor{clGreen}{rgb}{.608,.733,.349}
\definecolor{clDark25Green}{rgb}{.463,.573,.235}
\definecolor{clDark50Green}{rgb}{.310,.380,.157}

%\definecolor{clLight80Green}{rgb}{.886,.894,.906}
%\definecolor{clLight60Green}{rgb}{.843,.890,.737}
\definecolor{clLight40Red}{rgb}{.851,.588,.580}
\definecolor{clRed}{rgb}{.753,.314,.302}
%\definecolor{clDark25Green}{rgb}{.463,.573,.235}
%\definecolor{clDark50Green}{rgb}{.310,.380,.157}

% Schrift
%\usepackage{fontenc} 
%\usepackage{lmodern}
%\usepackage{textcomp}

% Computer Modern Sans Serif als Standard-Schrift 
\renewcommand*\familydefault{\sfdefault} %% Only if the base font of the document is to be sans serif

\makeatletter
\renewcommand{\section}{\@startsection {section}{1}{\z@}%
                                   {-3.5ex \@plus -1ex \@minus -.2ex}%
                                   {2.3ex \@plus.2ex}%
                                   {\rmfamily\Large\bfseries\color{clBlue}}}
\renewcommand{\subsection}{\@startsection{subsection}{2}{\z@}%
                                     {-3.25ex\@plus -1ex \@minus -.2ex}%
                                     {1.5ex \@plus .2ex}%
                                     {\rmfamily\large\bfseries\color{clBlue}}}
\renewcommand{\subsubsection}{\@startsection{subsubsection}{3}{\z@}%
                                     {-3.25ex\@plus -1ex \@minus -.2ex}%
                                     {1.5ex \@plus .2ex}%
                                     {\rmfamily\normalsize\bfseries\color{clBlue}}}


%---------------------------------------------
% \Chapter ändern
%
\def\@makechapterhead#1{%
    {\parspace
     \parbox[t][25\p@]{\linewidth}
     {\parindent \z@ \raggedright \rmfamily
     \ifnum \c@secnumdepth >\m@ne \vspace{\fill}
         \normalsize \bfseries \textcolor{clChapTxt}{\@chapapp\space \thechapter}
     \fi}
     \par\nobreak  \vskip 5\p@
     \interlinepenalty\@M   \raggedright \rmfamily
     \Huge \bfseries \textcolor{clChapTxt}{#1}\par\nobreak
     \vskip 40\p@
    }}
\def\@makeschapterhead#1{%
    {\parspace
     \parbox[t][25\p@]{\linewidth}{}
     \parindent \z@ \raggedright
     \par\nobreak  \vskip 5\p@
     \interlinepenalty\@M
     \Huge \rmfamily \bfseries  \textcolor{clChapTxt}{#1}\par\nobreak
     \vskip 40\p@
  }}
%---------------------------------------------
  
%---------------------------------------------
% Anderen Abstract
%
\newenvironment{abstractFrame}
{
  \null\vfil
  \@beginparpenalty\@lowpenalty
}
{\par\null}

\newenvironment{deAbstract}
{
  \selectlanguage{ngerman}
  \begin{center}%
    {\bfseries\rmfamily \abstractname\vspace{-.5em}\vspace{\z@}}%
  \end{center}% 
}
{\par\vfil}

\newenvironment{enAbstract}
{%
  \selectlanguage{english}
  \begin{center}%
    {\bfseries\rmfamily \abstractname\vspace{-.5em}\vspace{\z@}}%
  \end{center}% 
}%
{\par\vfil}
%---------------------------------------------


\renewcommand{\tableofcontents}{%
    \if@twocolumn
      \@restonecoltrue\onecolumn
    \else
      \@restonecolfalse
    \fi
    \chapter*{\contentsname
        \@mkboth{%
           \MakeUppercase\contentsname}{\MakeUppercase\contentsname}}%
    {\parskip 2pt\@starttoc{toc}}%
    \if@restonecol\twocolumn\fi
    }
    
\renewcommand{\listoffigures}{%
    \if@twocolumn
      \@restonecoltrue\onecolumn
    \else
      \@restonecolfalse
    \fi
    \chapter*{\listfigurename}%
      \@mkboth{\MakeUppercase\listfigurename}%
              {\MakeUppercase\listfigurename}%
    {\parskip 2pt\@starttoc{lof}}%
    \if@restonecol\twocolumn\fi
    }

\makeatother

% Zeilenabstand
\linespread{1.15}
\parspace

% Einrücken bei neuen Absatz
\parindent 0pt

% Linksbündig
\raggedright

% Abstand zwischen Überschrift und nachfolgendem Absatz veringern
\usepackage{titlesec}
\titlespacing{\section}{0pt}{*2}{*-1.3}
\titlespacing{\subsection}{0pt}{*2}{*-1.3}
\titlespacing{\subsubsection}{0pt}{*2}{*-1.3}

% URLs in den Quellen erkennen
\usepackage{url}

% Literatur- und Abbildungsverzeichnis mit ins Inhaltsverzeichnis
\usepackage[nottoc]{tocbibind}

% Literaturverzeichnis
\usepackage[numbers]{natbib}
\bibliographystyle{natdin}



 
