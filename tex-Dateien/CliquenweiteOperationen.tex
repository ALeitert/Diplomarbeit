\begin{longtable}{r>{\raggedright\arraybackslash}X@{}}
    $\odot_i$ & Es wird ein neuer Knoten erstellt und mit dem Label~$i$ versehen.\par
\\    
    $G_1 \oplus G_2$ & Wurden bereits zwei Graphen $G_1$ und $G_2$ (mit disjunkten Knotenmengen) erzeugt, werden sie nun als ein Graph $G$ betrachtet. Es werden jedoch keine neuen Kanten erstellt. Somit ist $G$ nicht zusammenhängend.\par
\\
    $\eta_{i,j}(G)$ & Bei dieser Operation werden zwei Knoten~$u$ und $v$ miteinander verbunden, wenn $u$ das Label~$i$ und $v$ das Label~$j$ hat. Diese Operation betrifft alle Knoten im Graphen mit den entsprechenden Labels. Außerdem gilt, dass $i \neq j$ sein muss. Es ist die einzige Möglichkeit, um Kanten zu erzeugen.\par
\\    
    $\rho_{i \rightarrow j}(G)$ & Vorhandene Labels können umbenannt werden. Dabei erhalten alle Knoten mit dem Label~$i$ das Label~$j$. %Es ist nicht möglich nur einen einzelnen Knoten umzubenennen.
\end{longtable}