% Verknüpfungen in der PDF
\usepackage{url}
\usepackage{hyperref}




%%%%%%%%%%%%%%%%%%%%%%%%%%%%%%%%%%%%%%%%%%%%%%%%%%%%%%%%%%%%%%
% Quellcode
\usepackage{listings}
\lstdefinelanguage{blub}{sensitive=false,morekeywords={for,each,mod,if,to,else,while,end,then,Procedure,Function,until,new,in,do,and,or}}
\lstset{%inputencoding=utf8/latin1,
	basicstyle=\ttfamily\small,
	keywordstyle=\color{blue},
	commentstyle=\color{darkgreen},
	numbers=none,
	breaklines=true,
	showstringspaces=false,
	tabsize=4,
	captionpos=b,
	float=htp,
	language=blub,
%	backgroundcolor=\color{clLight80Orange},
	mathescape=true,
	frame=single,
	rulecolor=\color{clBlue},
	morecomment=[l]{//},
}



%%%%%%%%%%%%%%%%%%%%%%%%%%%%%%%%%%%%%%%%%%%%%%%%%%%%%%%%%%%%
% (Farbige) Rahmen
\usepackage{framed}
\setlength{\fboxrule}{.5pt}
\setlength{\fboxsep}{4pt}


\newenvironment{CodeFrame}{
    \def\FrameCommand{\fcolorbox{clBlue}{white!0}}
    \MakeFramed{\advance\hsize-\width \FrameRestore}
  }
  {\endMakeFramed}

\newenvironment{ProofFrame}{
    \setlength{\fboxrule}{0pt}
    \setlength{\fboxsep}{0pt}
    \def\FrameCommand{\fcolorbox{white!0}{white!0}}
    \MakeFramed{\advance\hsize-\width \FrameRestore}
  }
  {\endMakeFramed}

\newenvironment{DefFrame}{
    \def\FrameCommand{\fcolorbox{clLight80Blue!33}{clLight80Blue!33}}
    \MakeFramed{\advance\hsize-\width \FrameRestore}
  }
  {\endMakeFramed}

\newenvironment{AlgoFrame}{
    \def\FrameCommand{\fcolorbox{clLight40Orange}{clLight80Orange!50}}
    \MakeFramed{\advance\hsize-\width \FrameRestore}
  }
  {\endMakeFramed}

\newenvironment{ToDoFrame}{
    \def\FrameCommand{\fcolorbox{clRed}{clLight40Red!33}}
    \MakeFramed{\advance\hsize-\width \FrameRestore}
  }
  {\endMakeFramed}

\newenvironment{TheoFrame}{
    \def\FrameCommand{\fcolorbox{clDark25Green}{clLight40Green!33}}
    \MakeFramed{\advance\hsize-\width \FrameRestore}
  }
  {\endMakeFramed}

\newenvironment{LemFrame}{
    \def\FrameCommand{\fcolorbox{clLight40Green!33}{clLight40Green!33}}
    \MakeFramed{\advance\hsize-\width \FrameRestore}
  }
  {\endMakeFramed}


%%%%%%%%%%%%%%%%%%%%%%%%%%%%%%%%%%%%%%%%%%%%%%%%%%%%%%%%
% Blöcke für Definitionen, Sätze und Beweise
\usepackage[framed,standard,hyperref]{ntheorem}

\theoremstyle{plain}
\theoremheaderfont{\rmfamily\bfseries\boldmath}
\theorembodyfont{\normalfont}
\theoremseparator{}
\theoremprework{}
\theorempreskipamount 0pt
\theorempostskipamount 0pt

\newtheorem{TheoBox}{Satz}[chapter]
\newtheorem{LemBox}{Lemma}[chapter]
\newtheorem*{LemBoxS}{Lemma}

\newtheorem*{myProof}{Beweis}

\theoremstyle{break}
\theoremheaderfont{\rmfamily\bfseries\boldmath}
\theorembodyfont{\normalfont}
\theoremseparator{}
\theoremprework{\vspace*{11pt}}
\theorempreskipamount 0pt
\theorempostskipamount 0pt

\theoremstyle{break}
\theoremheaderfont{\rmfamily\bfseries\boldmath}
\theorembodyfont{\normalfont}
\theoremseparator{}
\theoremprework{}
\theorempreskipamount 0pt
\theorempostskipamount 0pt

\newtheorem{DefBox}{Definition}[chapter]
\newtheorem{AlgoBox}{Algorithmus}[chapter]

\newenvironment{mydef}%[1][]%
{\begin{DefFrame}\begin{DefBox}%[#1]
}%
{\end{DefBox}\end{DefFrame}}
 
\newenvironment{Algorithm}%[1][]%
{\begin{AlgoFrame}\begin{AlgoBox}%[#1]
}%
{\end{AlgoBox}\end{AlgoFrame}}
 
\renewenvironment{Theorem}%
{\begin{TheoFrame}\begin{TheoBox}}%
{\end{TheoBox}\end{TheoFrame}} 

\renewenvironment{Lemma}%
{\begin{LemFrame}\begin{LemBox}}%
{\end{LemBox}\end{LemFrame}} 

\newenvironment{LemmaS}%
{\begin{LemFrame}\begin{LemBoxS}}%
{\end{LemBoxS}\end{LemFrame}} 

\renewenvironment{Proof}%
{\begin{myProof}\ \nopagebreak\\}%
{\end{myProof}} 

\makeatletter % Anpssung der Theorem-Konstrukte: \endtrivlist                          wird früher verwendet, um den erzeugten Platz am Ende des Theorems zu entfernen. Dient primär dazu, damit es in einem gefärbten Frame besser aussieht.
\gdef\@thm#1#2#3{%
   \if@thmmarks
     \stepcounter{end\InTheoType ctr}%
   \fi
   \renewcommand{\InTheoType}{#1}%
   \if@thmmarks
     \stepcounter{curr#1ctr}%
     \setcounter{end#1ctr}{0}%
   \fi
   \refstepcounter{#2}%
   \theorem@prework
   \thm@topsepadd \theorempostskipamount   % cf. latex.ltx: \@trivlist
   \ifvmode \advance\thm@topsepadd\partopsep\fi
   \trivlist                          
   \@topsep \theorempreskipamount
   \@topsepadd \thm@topsepadd        % used by \@endparenv
   \advance\linewidth -\theorem@indent
   \advance\@totalleftmargin \theorem@indent
   \parshape \@ne \@totalleftmargin \linewidth
   \@ifnextchar[{\@ythm{#1}{#2}{#3}}{\@xthm{#1}{#2}{#3}}}\endtrivlist %%%%%%%%%%%%%%%% neu
\gdef\@endtheorem{%
  %\endtrivlist                          %%%%%%%%%%%%%%%%%%%%%%%%%%%%%%%%%%%%%%%%%%%%%%%%%%%
  \csname\InTheoType @postwork\endcsname
  }

\makeatother


%%%%%%%%%%%%%%%%%%%%%%%%%%%%%%%%%%%%%%%%%%%%%%%%%%%
% Bilder und Abbildungen

\usepackage[font=footnotesize,labelfont=bf,justification=raggedright,hang]{caption}
%\renewcommand{\captionfont}{\footnotesize\sffamily}
%\renewcommand{\captionlabelfont}{\bfseries}
\usepackage{graphicx} % Bilder
\usepackage[format=default,justification=raggedright]{subfig} % Abbildungen aus mehreren Bildern

% Rahmen für Abbildungen in einen Befehl kapseln
\newcommand{\floatimage}[3]{\begin{figure}[htb]
\centering
#3
\caption{#2}
\label{#1}
\end{figure}}

% Mathe Symbole
\usepackage{amssymb}
\usepackage{amsmath}
\DeclareRobustCommand{\qed}{%
  \ifmmode \square%\mathqed
  \else
    \leavevmode\newline\unskip\penalty9999 \hbox{}\nobreak\hfill
    \quad\hbox{$\square$}%
  \fi
}

% ToDo-Befehl
\newcommand{\todo}[1]{
    \begin{ToDoFrame}\itshape
        {\rmfamily\normalsize\bfseries ToDo:} #1
    \end{ToDoFrame}
}

% Abstand nach einem Absatz
\newcommand{\parspace}{\setlength{\parskip}{11pt}}

%%%%%%%%%%%%%%%%%%%%%%%%%%%%%%%%%%%%%%%%%%%%%%%%%%%%%%%%%%%%%%%%%
% Farben
\usepackage[table]{xcolor}
\definecolor{clDefBG}{rgb}{.776,.850,.941}
\definecolor{clSecTxt}{rgb}{.122,.286,.490}
\definecolor{clChapTxt}{rgb}{.090,.212,.365}%{.06,.14,.24}

\definecolor{clLight80Blue}{rgb}{.776,.850,.941}
\definecolor{clLight60Blue}{rgb}{.553,.584,.886}
\definecolor{clLight40Blue}{rgb}{.329,.553,.831}
\definecolor{clBlue}{rgb}{.122,.286,.490}
\definecolor{clDark25Blue}{rgb}{.090,.212,.365}
\definecolor{clDark50Blue}{rgb}{.059,.141,.243}

\definecolor{clLight80Green}{rgb}{.886,.894,.906}
\definecolor{clLight60Green}{rgb}{.843,.890,.737}
\definecolor{clLight40Green}{rgb}{.765,.839,.608}
\definecolor{clGreen}{rgb}{.608,.733,.349}
\definecolor{clDark25Green}{rgb}{.463,.573,.235}
\definecolor{clDark50Green}{rgb}{.310,.380,.157}

%\definecolor{clLight80Green}{rgb}{.886,.894,.906}
%\definecolor{clLight60Green}{rgb}{.843,.890,.737}
\definecolor{clLight40Red}{rgb}{.851,.588,.580}
\definecolor{clRed}{rgb}{.753,.314,.302}
%\definecolor{clDark25Green}{rgb}{.463,.573,.235}
%\definecolor{clDark50Green}{rgb}{.310,.380,.157}

\definecolor{clViolet}{HTML}{8064A2}

\definecolor{clLight80Orange}{HTML}{FDEADA}
\definecolor{clLight60Orange}{HTML}{FBD5B5}
\definecolor{clLight40Orange}{HTML}{FAC08F}
\definecolor{clOrange}{HTML}{F79646}
\definecolor{clDark25Orange}{HTML}{E36C09}
\definecolor{clDark50Orange}{HTML}{974806}

\definecolor{clAqua}{HTML}{4BACC6}

% Schrift
\usepackage[T1]{fontenc} 
\usepackage{lmodern}
\usepackage{textcomp}

% Computer Modern Sans Serif als Standard-Schrift 
%\renewcommand*\familydefault{\sfdefault} %% Only if the base font of the document is to be sans serif

\makeatletter
\renewcommand{\section}{\@startsection {section}{1}{\z@}%
                                   {-3.5ex \@plus -1ex \@minus -.2ex}%
                                   {2.3ex \@plus.2ex}%
                                   {\rmfamily\Large\bfseries\boldmath\color{clBlue}}}
\renewcommand{\subsection}{\@startsection{subsection}{2}{\z@}%
                                     {-3.25ex\@plus -1ex \@minus -.2ex}%
                                     {1.5ex \@plus .2ex}%
                                     {\rmfamily\large\bfseries\boldmath\color{clBlue}}}
\renewcommand{\subsubsection}{\@startsection{subsubsection}{3}{\z@}%
                                     {-3.25ex\@plus -1ex \@minus -.2ex}%
                                     {1.5ex \@plus .2ex}%
                                     {\rmfamily\normalsize\bfseries\boldmath\color{clBlue}}}
\renewcommand\paragraph{\@startsection{paragraph}{4}{\z@}%
                                    {3.25ex \@plus1ex \@minus.2ex}%
                                    {-1em}%
                                    {\rmfamily\normalsize\boldmath\bfseries}}




%---------------------------------------------
% \Part ändern
%

\renewcommand\part{%
  \if@openright
    \cleardoublepage
  \else
    \clearpage
  \fi
  \thispagestyle{empty}%
  \if@twocolumn
    \onecolumn
    \@tempswatrue
  \else
    \@tempswafalse
  \fi
  \null\vspace*{\fill}
  \secdef\@part\@spart}

\def\@part[#1]#2{%
    \ifnum \c@secnumdepth >-2\relax
      \refstepcounter{part}%
      \addcontentsline{toc}{part}{\thepart\hspace{1em}#1}%
    \else
      \addcontentsline{toc}{part}{#1}%
    \fi
    \markboth{}{}%
    {\centering
     \interlinepenalty \@M
     \normalfont\raggedleft
     \color{clChapTxt}\rule{\linewidth}{1pt}
     \ifnum \c@secnumdepth >-2\relax
       \normalsize\bfseries\boldmath \partname\nobreakspace\thepart
       \\
       %\vskip 20\p@
     \fi
     \Huge \bfseries\boldmath #2\par
     \rule[\baselineskip]{\linewidth}{1pt}
     }%
    \@endpart}
    
\def\@spart#1{%
    {\centering
     \interlinepenalty \@M
     \normalfont\raggedleft
     \color{clChapTxt}\rule{\linewidth}{1pt}
     \Huge \bfseries\boldmath #1\par
     \rule[\baselineskip]{\linewidth}{1pt}}%
    \@endpart}
    
\def\@endpart{
    \vspace*{\fill}
    \vspace*{\fill}
    \vspace*{\fill}
    \newpage
              \if@twoside
               \if@openright
                \null
                \thispagestyle{empty}%
                \newpage
               \fi
              \fi
              \if@tempswa
                \twocolumn
              \fi}
              

%---------------------------------------------
% \Chapter ändern
%
\renewcommand\chapter{\if@openright\cleardoublepage\else\clearpage\fi
                    \thispagestyle{plain}%
                    \global\@topnum\z@
                    \@afterindentfalse
                    \secdef\@chapter\@schapter}
\def\@makechapterhead#1{%
    {\parspace
     \parbox[t][25\p@]{\linewidth}
     {\parindent \z@ \raggedright \rmfamily
     \ifnum \c@secnumdepth >\m@ne \vspace{\fill}
         \normalsize \bfseries \textcolor{clChapTxt}{\@chapapp\space \thechapter}
     \fi}
     \par\nobreak  \vskip 5\p@
     \interlinepenalty\@M   \raggedright \rmfamily
     \Huge \bfseries\boldmath \textcolor{clChapTxt}{#1}\par\nobreak
     \vskip 40\p@
    }}
\def\@makeschapterhead#1{%
    {\parspace
     \parbox[t][25\p@]{\linewidth}{}
     \parindent \z@ \raggedright
     \par\nobreak  \vskip 5\p@
     \interlinepenalty\@M
     \Huge \rmfamily \bfseries\boldmath  \textcolor{clChapTxt}{#1}\par\nobreak
     \vskip 40\p@
  }}



%---------------------------------------------
% Kopfzeile

\setlength{\headheight}{16.5pt}

\if@twoside
  \def\ps@headings{%
    \def\@oddfoot{\tikz{
        \node[inner xsep=0pt, anchor=base west] (chapNode) at (0,0) {};
        \node[inner xsep=0pt, anchor=base east] (pageNode) at (\textwidth,0) {\thepage};
        %\draw[clChapTxt,semithick] ($(current bounding box.north west)+(0,1pt)$) -- ($(current bounding box.north east)+(0,0pt)$);
        %\draw[clChapTxt,semithick] (0,1em) -- ($(\linewidth,1em)-(0.0pt,0)$);
        \draw[clChapTxt,semithick] (0.5\textwidth,1em) -- (current bounding box.north west)
                                   (0.5\textwidth,1em) -- (current bounding box.north east);
    }}
    \def\@evenfoot{\tikz{
        \node[inner xsep=0pt, anchor=base west] (chapNode) at (0,0) {\thepage};
        \node[inner xsep=0pt, anchor=base east] (pageNode) at (\textwidth,0) {};
        %\draw[clChapTxt,semithick] ($(current bounding box.north west)+(0,1pt)$) -- ($(current bounding box.north east)+(0,0pt)$);
        %\draw[clChapTxt,semithick] (0,1em) -- ($(\linewidth,1em)-(0.0pt,0)$);
        \draw[clChapTxt,semithick] (0.5\textwidth,1em) -- (current bounding box.north west)
                                   (0.5\textwidth,1em) -- (current bounding box.north east);
    }}
      
   %   \def\@evenhead{\thepage\hfil\slshape\leftmark}%
    \def\@evenhead{\tikz{
        \node[inner xsep=0pt, anchor=base west] (chapNode) at (0,0) {};
        \node[inner xsep=0pt, anchor=base east] (pageNode) at (\textwidth,0) {\rmfamily\small\textcolor{clChapTxt}\leftmark};
        \draw[clChapTxt,semithick] (current bounding box.south west) -- (current bounding box.south east);
    }}%
    \def\@oddhead{\tikz{
        \node[inner xsep=0pt, anchor=base west] (chapNode) at (0,0) {\rmfamily\small\textcolor{clChapTxt}\rightmark};
        \node[inner xsep=0pt, anchor=base east] (pageNode) at (\textwidth,0) {};
        \draw[clChapTxt,semithick] (current bounding box.south west) -- (current bounding box.south east);
    }}%
      \let\@mkboth\markboth
    \def\chaptermark##1{%
      \markboth {{%
        \ifnum \c@secnumdepth >\m@ne
            \@chapapp\ \thechapter. \ %
        \fi
        ##1}}{}}%
    \def\sectionmark##1{%
      \markright {{%
        \ifnum \c@secnumdepth >\z@
          \thesection. \ %
        \fi
        ##1}}}}
\else
  \def\ps@headings{%
    %\let\@oddfoot\@empty
    \def\@oddfoot{\tikz{
        \node[inner xsep=0pt, anchor=base west] (chapNode) at (0,0) {};
        \node[inner xsep=0pt, anchor=base east] (pageNode) at (\textwidth,0) {\textcolor{black}\thepage};
        %\draw[clChapTxt,semithick] ($(current bounding box.north west)+(0,1pt)$) -- ($(current bounding box.north east)+(0,0pt)$);
        %\draw[clChapTxt,semithick] (0,1em) -- ($(\linewidth,1em)-(0.0pt,0)$);
        \draw[clChapTxt,semithick] (0.5\textwidth,1em) -- (current bounding box.north west)
                                   (0.5\textwidth,1em) -- (current bounding box.north east);
    }}
    \def\@oddhead{\tikz{
        \node[inner xsep=0pt, anchor=base west] (chapNode) at (0,0) {\rmfamily\small\textcolor{clChapTxt}\rightmark};
        \node[inner xsep=0pt, anchor=base east] (pageNode) at (\textwidth,0) {};
        \draw[clChapTxt,semithick] (current bounding box.south west) -- (current bounding box.south east);
    }}%
    \let\@mkboth\markboth
    \def\chaptermark##1{%
      \markright {%
        \ifnum \c@secnumdepth >\m@ne
            \@chapapp\ \thechapter. \ %
        \fi
        ##1}}}
\fi



\if@twoside
  \def\ps@plain{%
    %\let\@oddfoot\@empty
    \def\@evenfoot{\tikz{
        \node[inner xsep=0pt, anchor=base west] (chapNode) at (0,0) {\textcolor{black}\thepage};
        \node[inner xsep=0pt, anchor=base east] (pageNode) at (\textwidth,0) {};
        \draw[clChapTxt,semithick] (0.5\textwidth,1em) -- (current bounding box.north west)
                                   (0.5\textwidth,1em) -- (current bounding box.north east);
    }}
    \def\@evenhead{\tikz{
        \node[inner xsep=0pt, anchor=base west] (chapNode) at (0,0) {};
        \node[inner xsep=0pt, anchor=base east] (pageNode) at (\textwidth,0) {};
        \draw[clChapTxt,semithick] (current bounding box.south west) -- (current bounding box.south east);
    }}%
    \def\@oddfoot{\tikz{
        \node[inner xsep=0pt, anchor=base west] (chapNode) at (0,0) {};
        \node[inner xsep=0pt, anchor=base east] (pageNode) at (\textwidth,0) {\textcolor{black}\thepage};
        \draw[clChapTxt,semithick] (0.5\textwidth,1em) -- (current bounding box.north west)
                                   (0.5\textwidth,1em) -- (current bounding box.north east);
    }}
    \def\@oddhead{\tikz{
        \node[inner xsep=0pt, anchor=base west] (chapNode) at (0,0) {};
        \node[inner xsep=0pt, anchor=base east] (pageNode) at (\textwidth,0) {};
        \draw[clChapTxt,semithick] (current bounding box.south west) -- (current bounding box.south east);
    }}%
    \let\@mkboth\markboth
    \def\chaptermark##1{%
      \markright {%
        \ifnum \c@secnumdepth >\m@ne
            \@chapapp\ \thechapter. \ %
        \fi
        ##1}}}
\else
  \def\ps@plain{%
    %\let\@oddfoot\@empty
    \def\@oddfoot{\tikz{
        \node[inner xsep=0pt, anchor=base west] (chapNode) at (0,0) {};
        \node[inner xsep=0pt, anchor=base east] (pageNode) at (\textwidth,0) {\textcolor{black}\thepage};
        \draw[clChapTxt,semithick] (0.5\textwidth,1em) -- (current bounding box.north west)
                                   (0.5\textwidth,1em) -- (current bounding box.north east);
    }}
    \def\@oddhead{\tikz{
        \node[inner xsep=0pt, anchor=base west] (chapNode) at (0,0) {};
        \node[inner xsep=0pt, anchor=base east] (pageNode) at (\textwidth,0) {};
        \draw[clChapTxt,semithick] (current bounding box.south west) -- (current bounding box.south east);
    }}%
    \let\@mkboth\markboth
    \def\chaptermark##1{%
      \markright {%
        \ifnum \c@secnumdepth >\m@ne
            \@chapapp\ \thechapter. \ %
        \fi
        ##1}}}
\fi

%---------------------------------------------
% Inhaltsverzeichnis ändern, damit es auch Mathe-Symbole fett macht
%
\renewcommand*\l@part[2]{%
  \ifnum \c@tocdepth >-2\relax
    \addpenalty{-\@highpenalty}%
    \addvspace{2.25em \@plus\p@}%
    \setlength\@tempdima{3em}%
    \begingroup
      \parindent \z@ \rightskip \@pnumwidth
      \parfillskip -\@pnumwidth
      {\leavevmode
       \large \bfseries\boldmath\color{clBlue}\rule{\linewidth}{1pt} #1\hfil \hb@xt@\@pnumwidth{\hss #2}\\\rule[.5em]{\linewidth}{1pt}}\par \vspace*{-\baselineskip} %\par
       %\large \bfseries\boldmath\color{clBlue} #1\hfil \hb@xt@\@pnumwidth{\hss #2}}\par
       \nobreak
         \global\@nobreaktrue
         \everypar{\global\@nobreakfalse\everypar{}}%
    \endgroup
  \fi}
\renewcommand*\l@chapter[2]{%
  \ifnum \c@tocdepth >\m@ne
    \addpenalty{-\@highpenalty}%
    \vskip 1.0em \@plus\p@
    \setlength\@tempdima{1.5em}%
    \begingroup
      \parindent \z@ \rightskip \@pnumwidth
      \parfillskip -\@pnumwidth
      \leavevmode \bfseries\boldmath\color{clBlue}
      \advance\leftskip\@tempdima
      \hskip -\leftskip
      #1\nobreak\hfil \nobreak\hb@xt@\@pnumwidth{\hss #2}\par
      \penalty\@highpenalty
    \endgroup
  \fi}
%---------------------------------------------

  
%---------------------------------------------
% Anderen Abstract
%
\newenvironment{abstractFrame}
{
  \if@openright\cleardoublepage\else\clearpage\fi 
  \null\vfil
  \@beginparpenalty\@lowpenalty
}
{\par\null}

\newenvironment{deAbstract}
{
  \selectlanguage{ngerman}
  %\begin{center}%
  %  {\bfseries\rmfamily \abstractname\vspace{-.5em}\vspace{\z@}}%
  %\end{center}% 
  \subsection*{\abstractname}
}
{\par\vfil}

\newenvironment{enAbstract}
{%
  \selectlanguage{english}
  %\begin{center}%
  %  {\bfseries\rmfamily \abstractname\vspace{-.5em}\vspace{\z@}}%
  %\end{center}% 
  \subsection*{\abstractname}

}%
{\par\vfil}
%---------------------------------------------


\renewcommand{\tableofcontents}{%
    \if@twocolumn
      \@restonecoltrue\onecolumn
    \else
      \@restonecolfalse
    \fi
    \chapter*{\contentsname
        \@mkboth{%
           \contentsname}{\contentsname}}%
    {\setlength{\parskip}{2pt}\@starttoc{toc}}%
    \if@restonecol\twocolumn\fi
    }

\renewenvironment{thebibliography}[1]
     {\chapter*{\bibname\ blub}\addcontentsline{toc}{chapter}{\bibname\ blub}%
      \@mkboth{\bibname}{\bibname}%
      \list{\@biblabel{\@arabic\c@enumiv}}%
           {\settowidth\labelwidth{\@biblabel{#1}}%
            \leftmargin\labelwidth
            \advance\leftmargin\labelsep
            \@openbib@code
            \usecounter{enumiv}%
            \let\p@enumiv\@empty
            \renewcommand\theenumiv{\@arabic\c@enumiv}}%
      \sloppy
      \clubpenalty4000
      \@clubpenalty \clubpenalty
      \widowpenalty4000%
      \sfcode`\.\@m}
     {\def\@noitemerr
       {\@latex@warning{Empty `thebibliography' environment}}%
      \endlist}
    
\renewcommand{\listoffigures}{%
    \if@twocolumn
      \@restonecoltrue\onecolumn
    \else
      \@restonecolfalse
    \fi
    \chapter*{\listfigurename}\addcontentsline{toc}{chapter}{\listfigurename}%
      \@mkboth{\listfigurename}%
              {\listfigurename}%
    {\setlength{\parskip}{2pt}\@starttoc{lof}}%
    \if@restonecol\twocolumn\fi
    }
\renewcommand\listoftables{%
    \if@twocolumn
      \@restonecoltrue\onecolumn
    \else
      \@restonecolfalse
    \fi
    \chapter*{\listtablename}\addcontentsline{toc}{chapter}{\listfigurename}%
      \@mkboth{%
          \listtablename}%
         {\listtablename}%
    {\setlength{\parskip}{2pt}\@starttoc{lot}}%
    \if@restonecol\twocolumn\fi
    }

\renewcommand*\l@figure{\@dottedtocline{1}{0em}{2.3em}}
\renewcommand*\l@table{\@dottedtocline{1}{0em}{2.3em}}

\makeatother

% Zeilenabstand
\linespread{1.15}
\parspace

% Einrücken bei neuen Absatz
\parindent 0pt

% Linksbündig
\raggedright

% Abstand zwischen Überschrift und nachfolgendem Absatz veringern
\usepackage{titlesec}
\titlespacing{\section}{0pt}{*2}{*-1.3}
\titlespacing{\subsection}{0pt}{*2}{*-1.3}
\titlespacing{\subsubsection}{0pt}{*2}{*-1.3}

% URLs in den Quellen erkennen
%\usepackage{url}
%\renewcommand\UrlFont{\ttfamily}


% Literatur- und Abbildungsverzeichnis mit ins Inhaltsverzeichnis
%\usepackage[nottoc,notlof,notlot]{tocbibind}

% Literaturverzeichnis
\usepackage[numbers]{natbib}
\bibliographystyle{natdin}

\makeatletter
\renewenvironment{theindex}
               {
                %
                \begin{multicols}{2}[\chapter*{\indexname}\addcontentsline{toc}{chapter}{\indexname}]
                \@mkboth{\indexname}%
                        {\indexname}%
                \thispagestyle{plain}\parindent\z@
                \parskip\z@ \@plus .3\p@\relax
                \columnseprule \z@
                \columnsep 35\p@
                \let\item\@idxitem}
               {\end{multicols}}
               
\renewcommand\bibsection{%
      \chapter*{\bibname}\addcontentsline{toc}{chapter}{\bibname}
      \@mkboth{\bibname}{\bibname}%
    }
\makeatother



%\show\labelitemi
 \newlength{\enumItemWidth}
\renewenvironment{itemize}
    {\begin{list}
        {\settowidth{\enumItemWidth}{\labelenumi}\hspace*{-\enumItemWidth}\labelitemi}
        {\setlength{\topsep}{0.5em}
         \setlength{\itemsep}{0pt}
         \setlength{\parskip}{0pt}
         }
    }
    {\end{list}}
\renewenvironment{enumerate}
    {\begin{list}
        {\settowidth{\enumItemWidth}{\labelenumi}\hspace*{-\enumItemWidth}\labelenumi}
        {\usecounter{enumi}
         \setlength{\topsep}{0.5em}
         \setlength{\itemsep}{0pt}
         \setlength{\parskip}{0pt}
         }
    }
    {\end{list}}
\renewcommand{\theenumi}{(\roman{enumi})}
\renewcommand{\labelenumi}{\theenumi} 
\renewcommand{\theenumii}{(\alph{enumii})}
\renewcommand{\labelenumii}{\theenumii} 

\newenvironment{codeLine}
    {\begin{list}
        {\settowidth{\enumItemWidth}{(\arabic{enumi})}\hspace*{-\enumItemWidth}(\arabic{enumi})}
        {\usecounter{enumi}
         \setlength{\topsep}{0.5em}
         \setlength{\itemsep}{0pt}
         \setlength{\parskip}{0pt}
         }
    }
    {\end{list}}

\newenvironment{innerCodeLine}
    {\begin{list}
        {\settowidth{\enumItemWidth}{\labelenumii}\hspace*{-\enumItemWidth}\labelenumii}
        {\usecounter{enumii}
         \setlength{\topsep}{0.5em}
         \setlength{\itemsep}{0pt}
         \setlength{\parskip}{0pt}
         }
    }
    {\end{list}}

        

 