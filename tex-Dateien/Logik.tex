\chapter{Prädikatenlogik}

bla bla

\section{Logik erster Stufe}\index{FOL|see{Prädikatenlogik erster Stufe}}\index{Prädikatenlogik!erster Stufe}

bla bla

Die Definitionen in diesem Abschnit basieren auf den entsprechenden Definitionen in \cite{kreuzer2006logik} und \cite{wikiD:FOL}.

\subsection{Syntax}
Die Syntax der Prädikatenlogik unterteilt sich in drei Ebenen: Symbole, Terme und Formeln. Dabei bilden Symbole die unterste Ebene und Formeln die höchste.

\subsubsection{Symbole}
Symbole sind die Elemente für jede prädikatenlogische Aussage. Sie teilen sich in fünf Teilmengen ein: Quantoren, Variablen, Funktionssymbole, Prädikatensymbole und logische Symbole.

\begin{mydef}[Symbole]
    Die Symbole der Prädikatenlogik ergeben sich aus den paarweise disjunkten Mengen
   \begin{itemize}
        \item der Variablen $\mV$,
        \item der Funktionssymbole $\mF$,
        \item der Prädikatensymbole $\mP$,
        \item der Quantoren $\{\forall,\exists\}$ und
        \item der logischen Symbole $\{\land,\lor,\lnot\}$.
    \end{itemize}
\end{mydef}

\subsubsection{Terme}
Terme werden aus Variablen und Funktionssymbolen oder bereits vorhanden Termen gebildet.
\begin{mydef}[Terme]
    Es seien $v$ eine Variable, $f$ ein Funktionssymbol und $t_1,\ldots,t_k$ ($k\geq 1$) Terme. Dann gilt:
    \begin{itemize}
       \item  $v$ und $f$ sind Terme.
       \item  $f(t_1,\ldots,t_k)$ ist ein Term.
    \end{itemize}
    Im ersten Fall sagt man, dass $f$ eine konstante (oder $0$-stellige) Funktion ist. Im zweiten Fall spricht man von einer $k$-stelligen Funktion.
\end{mydef}

\subsubsection{Formeln}
Aus Symbolen und Termen lassen sich nun Formeln bilden. Genau wie bei Termen erfolgt dies wieder induktiv.
\begin{mydef}[Formeln]
    Es seien $v$ eine Variable, $P$ ein Prädikatensymbol, $Q$ ein Quantor, $t_1,\ldots,t_k$ ($k\geq 1$) Terme und $F$ und $G$ Formeln. Dann sind die folgenden Ausdrücke Formeln:
    \begin{itemize}
       \item  $P(t_1,\ldots,t_k)$
       \item  $Qv\ F$
       \item  $\lnot\,F$, $(F\land G)$, $(F\lor G)$
    \end{itemize}
    Im ersten Fall sagt man, dass $P$ ein $k$-stelliges Prädikat ist. Prädikate können nicht $0$-stellig sein.
\end{mydef}

\subsubsection{Anmerkungen}
In dieser Arbeit wird davon ausgegangen, dass die Stelligkeit von Prädikaten- und Funktionssymbolen eindeutig ist. Im Allgemeinen ist das aber nicht nötig. Hat ein Prädikaten- oder Funktionssymbol $S$ verschiedene Stelligkeiten (z. B. $j$ und $k$), kann es durch die Symbole $S_j$ und $S_k$ ersetzt werden. Aufgrund der Syntax ist die Ersetzung für jedes Vorkommen von $S$ eindeutig.
\todo{Überprüfen, ob nicht doch irgendwo in der Arbeit mehrdeutig}

Folgende Notation wird in dieser Arbeit verwendet:
\begin{itemize}
    \item  kleine lateinische Buchstaben für Variablen und $k$-stellige Funktionen ($k\geq 1$)
    \item  kleine grichische Buchstaben für konstante Funktionen
    \item  große lateinische Buchstaben für Prädikate
\end{itemize}

\todo{Werden Konstanten überhaubt benötigt?}

\subsection{Semantik}

\begin{mydef}[Struktur]
    Es sei $F$ eine Formel in FOL. Des Weiteren sei $P$ ein $k$-stelliges Prädikatensymbol, $f$ ein $k$-stelliges Funktionssymbol und $v$ eine Variable in $F$.
    
    Eine zu $F$ passende Struktur $\mS$ ist ein 4-Tupel $\mS=(\mU,\varphi,\psi,\chi)$ aus einer nicht leeren Menge $\mU$ (\emph{Universum} \index{Universum} genannt) und den Abbildungen $\varphi$, $\psi$ und $\chi$. Dabei gilt:
    \begin{itemize}
        \item  $\varphi(f):\mU^k\rightarrow\mU$
        \item  $\psi(P)\subseteq\mU^k$
        \item  $\chi(v)\in\mU$
    \end{itemize}
\end{mydef}

\section{Logik zweiter Stufe}\index{SOL}\index{Prädikatenlogik!zweiter Stufe}
Quantoren über prädikate

\begin{mydef}[Formeln in SOL]
    Es seien $v$ eine Variable, $P$ ein Prädikatensymbol, $Q$ ein Quantor, $t_1,\ldots,t_k$ ($k\geq 1$) Terme und $F$ und $G$ Formeln. Dann sind die folgenden Ausdrücke Formeln:
    \begin{itemize}
       \item  $P(t_1,\ldots,t_k)$
       \item  $Qv\ F$ und $QP\ F$
       \item  $\lnot\,F$, $(F\land G)$ und $(F\lor G)$
    \end{itemize}
\end{mydef}

\url{http://www2.informatik.hu-berlin.de/logik/lehre/WS07-08/Logik/HU/kap8.pdf}
@book\{gurski2010exakte,\\
  title=\{Exakte Algorithmen für schwere Graphenprobleme\},\\
  author=\{Gurski, F. and Rothe, I. and Rothe, J. and Wanke, E.\},\\
  isbn=\{978-3-642-04499-1\},\\
  year=\{2010\},\\
  publisher=\{Springer Berlin Heidelberg\}\\
\}

\section{Monadische Logik}
Nur einstellige Prädikate

\section{Anwendung bei Graphen}